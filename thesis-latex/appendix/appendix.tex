\chapter{Additional Material}

\section{Installing Alpha}

\paragraph{Using a pre-built binary file}
The Alpha ASP solver is written in Java. In order to run it, a Java Runtime Environment (JRE) compatible with Java 11 (i.e. Version 11 or newer) is required. Alpha builds can be obtained from the Github page\footnote{\url{https://github.com/alpha-asp/Alpha/releases}}.

The artifact required to run Alpha is the bundled jar file containing the commandline application, i.e. \texttt{alpha-cli-app-\$\{version\}-bundled.jar}.

In order to solve an ASP program stored in file \texttt{program.asp}, run the jar file as demonstrated in Listing~\ref{lst:appendix-run-alpha}.
\begin{lstlisting}[style=code, label={lst:appendix-run-alpha}, caption={Solving an ASP program using Alpha.}]
$>java -jar /path/to/alpha/
	alpha-cli-app-x.y.z-bundled.jar -i program.asp
\end{lstlisting}	

In order to run the Alpha jar from any location in the file system without always having to enter the full path to the jar, one can create a launcher script for Alpha. This launcher script then needs to be placed in a directory that is included in the system's \texttt{PATH} environment variable (e.g. \texttt{/usr/local/bin} on most Linux systems). Listing~\ref{lst:appendix-launcher-linux} shows a launcher script suitable for a Linux system (i.e. in the Bash scripting language) which assumes the Alpha jar for Alpha version 0.8.0 to be stored in \texttt{/usr/local/bin}.
\begin{lstlisting}[style=asp-code, label={lst:appendix-launcher-linux}, caption={Bash launcher script for Alpha.}]
#!/bin/bash

java -jar /usr/local/bin/alpha-cli-app-0.8.0-bundled.jar $@
\end{lstlisting}	

Listing~\ref{lst:appendix-launcher-windoze} shows a Windows-compatible script which assumes that the Alpha jar is stored in \texttt{C:/Program Files/alpha-asp}.
\begin{lstlisting}[style=asp-code, label={lst:appendix-launcher-windoze}, caption={Windows-Batch launcher script for Alpha.}]
java -jar "C:\Program Files\alpha-asp\alpha-cli-app-0.8.0-SNAPSHOT-bundled.jar" %*
\end{lstlisting}	

Using a central launcher script, Alpha can be called through the name of the launcher script from any working directory. This is demonstrated in Listing~\ref{lst:appendix-use-launcher}, where \texttt{alpha-solver} is assumed as name of the launcher script.
\begin{lstlisting}[style=code, label={lst:appendix-use-launcher}, caption={Solving an ASP program using Alpha through a launcher script.}]
$>alpha-solver -i program.asp
\end{lstlisting}

\paragraph{Building Alpha from source code}
As an alternative to using a pre-built binary package, Alpha can also be built from sources. For this, a Java Development Kit (JDK) compatible with Java 11 is required. In order to build Alpha, clone the git repository\footnote{\url{https://github.com/alpha-asp/Alpha.git}} and start the build through the Gradle Wrapper script as demonstrated in Listing~\ref{lst:appendix:build-alpha}. The actual executable jar is built using the gralde task \emph{bundledJar}, but to make sure all automated tests succeed as expected, is it recommended to run a full build (using taks \emph{build}) first. After the \emph{bundledJar} task has completed, the executable file is located in the build directory of the \texttt{alpha-cli-app} module.
\begin{lstlisting}[style=code, label={lst:appendix-build-alpha}, caption={Building Alpha from source code.}]
$>git clone https://github.com/alpha-asp/Alpha.git
$>cd Alpha
$>./gradlew build
$>./gradlew bundledJar
$>java -jar alpha-cli-app/build/libs/
	alpha-cli-app-0.8.0-bundled.jar -i program.asp
\end{lstlisting}	

\section{Examples}
\begin{example}[Fibonacci-Numbers using external atoms]
\label{ex:user-supplied-externals}
The following code snippet demonstrates how to run a program which includes user-supplied external atoms using Alpha. Since the Alpha Commandline-App currently does not support loading Atom Definitions from jar files, the solver is directly called from Java code in this example.\\
\\
In this example, we use an external predicate definition \texttt{fibonacci\_number/2} to efficiently calculate Fibonacci numbers. The actual ASP program generates all Fibonacci numbers up to index 40 that are even. The ASP program is shown in Listing~\ref{lst:user-supplied-externals-asp}, while Listing~\ref{lst:user-supplied-externals-java} shows the Java implementation of the external predicate.
\begin{lstlisting}[style=asp-code, label={lst:user-supplied-externals-asp}, caption={ASP program to find even Fibonacci numbers.}]
%% Find even fibonacci numbers up to F(40).
fib(N, FN) :- &fibonacci_number[N](FN), N = 0..40.
even_fib(N, FN) :- fib(N, FN), FN \ 2 = 0.    
\end{lstlisting}    
The Java implementation of the \texttt{fibonacci\_number/2} predicate makes use of \emph{Binet's Formula}\footnote{\url{https://en.wikipedia.org/wiki/Fibonacci\_sequence\#Binet's\_formula}} to efficiently compute Fibonacci numbers using a closed form expression of the sequence.
\begin{lstlisting}[style=java, label={lst:user-supplied-externals-java}, caption={Fibonacci number computation in Java.}]
public class CustomExternals {

	private static final double GOLDEN_RATIO = 1.618033988749894;
	private static final double SQRT_5 = Math.sqrt(5.0);

	/**
	 * Calculates the n-th Fibonacci number using a variant of Binet's Formula
	 */
	$$@Predicate(name="fibonacci_number")
	public static Set<List<ConstantTerm<Integer>>> fibonacciNumber(int n) {
		return Set.of(List.of(Terms.newConstant(binetRounding(n))));
	}

	public static int binetRounding(int n) {
		return (int) Math.round(Math.pow(GOLDEN_RATIO, n) / SQRT_5);
	}

}    
\end{lstlisting}    
The Alpha solver is invoked from a simple Java method, which uses Alpha's API to first compile the external atom definition, and then run the solving process for the parsed ASP program. This is illustrated in Listing~\ref{lst:user-supplied-externals-main}
\begin{lstlisting}[style=java, label={lst:user-supplied-externals-main}]
public class CustomExternalsApp {

	public static void main(String[] args) throws IOException {
		Alpha alpha = AlphaFactory.newAlpha();
		Map<String, PredicateInterpretation> customExternals =
			Externals.scan(CustomExternals.class);
		String aspCode =
			Files.readString(
				Paths.get("src/main/resources/
					customExternals.asp"));
				InputProgram program =
					alpha.readProgramString(aspCode,
				customExternals);
		alpha.solve(program).forEach(as -> {
			System.out.println("Answer set:\n" + as);
		});
	}

}
\end{lstlisting}
\end{example}

\begin{example}[Module Parsing]
\label{ex:alpha-module-parsing}	
Listing~\ref{lst:apdx-alpha-module-grammar} shows the actual ANTLR-grammar used in Alpha to parse module definitions. Parsed Module Definitions are stored in instances of the \texttt{Module} type. Listing~\ref{lst:apdx-alpha-module-def} shows the corresponding Java interface definition.
\begin{lstlisting}[style=asp-code, label={lst:apdx-alpha-module-grammar}, caption={ANTLR grammar for module definitions}]
directive_module: 
	SHARP DIRECTIVE_MODULE id 
		PAREN_OPEN module_signature PAREN_CLOSE 
			CURLY_OPEN statements CURLY_CLOSE;

module_signature : 
	predicate_spec ARROW CURLY_OPEN 
		('*' | predicate_specs) CURLY_CLOSE;	

predicate_specs: 
	predicate_spec (COMMA predicate_specs)?;

predicate_spec: id '/' NUMBER;
\end{lstlisting}
Note how the \texttt{module\_signature} rule permits the shorthand \texttt{*} instead of an output predicate list.
This is used as a shorthand to express the list of all predicates derived by any rule in the module (i.e. emit full, unfiltered answer sets).	
\begin{lstlisting}[style=java, label={lst:apdx-alpha-module-def}, caption={Java interface specifying Alpha's internal representation of a module definition.}]
public interface Module {

	String getName();

	Predicate getInputSpec();

	Set<Predicate> getOutputSpec();

	InputProgram getImplementation();

}	
\end{lstlisting}	
\end{example}	

\section{An XML-DOM parser in ASP}
\label{sec:appendix-xml-dom-parsing}

The module \texttt{xml\_dom} which is used in Listing~\ref{lst:results-xml-graphcol-dom-unpack} implements a generic XML-parser that generates a DOM-representation based on external atoms for regex evaluation.

Input is encoded as a list of strings, each one representing a line of input. At first, all input is tokenized in order to find opening and closing XML tags. Listing~\ref{lst:results-xml-parse-tokenize} shows the Evolog code used for tokenization in the \texttt{xml\_dom} module.

\begin{lstlisting}[style=asp-code, label={lst:results-xml-parse-tokenize}, caption={Tokenizing XML input.}]
token_regex(tag_open, "(<(\w+)( (\w+=\\"\w+\\"))*>)").
token_regex(tag_close, "(</\w+>)").
opening_tag_name_regex("<(\w+)( (\w+=\\"\w+\\"))*>").
closing_tag_name_regex("</(\w+)>").
attribute_regex("(\w+=\\"\w+\\")").
attribute_name_regex("(\w+)=\\"\w+\\"").
attribute_value_regex("\w+=\\"(\w+)\\"").

% Unwrap lines
line_element(E, TAIL) :- input_lines(lst(E, TAIL)).
line_element(E, TAIL) :- line_element(_, lst(E, TAIL)).
line(LINE_NO, LINE) :- line_element(line(LINE_NO, LINE), _).

% Match basic tokens
token(t(TOK, VALUE), LINE_NO, FROM, TO) :- 
    line(LINE_NO, LINE), token_regex(TOK, REGEX), 
    &regex_matches[REGEX, LINE](VALUE,FROM, TO). 
\end{lstlisting}

Regular expressions to match opening and closing XML tags are represented as facts the predicate \texttt{token\_regex/2}. The external atom \texttt{regex\_matches/5} is used to determine whether a line matches a token regex, as well as the respective parts of the string where the match occurs.

Once all tokens denoting opening and closing tags - up until that point ignoring actual tag names and relative positions of tags to each other - the next step is extracting actual tag names and pairs of opening and closing tags of the same name. This part of the parsing process is demonstrated in Listing~\ref{lst:results-xml-parse-tag-pairs}. First, a relative ordering of tokens is established using the \texttt{token\_before/8} predicate, which is derived by the rules on lines 2 through 11. Furthermore, we extract the name of each tag by using a variant of the tag open/close regexes with capture groups such that the first group is only the actual name of the tag, see lines 14 to 29. A matching pair of opening and closing tags in a well-formed XML document is characterized as an opening and a closing tag of the same name, with no opening tag of that name in between. In order to find these pairs, we first derive all matching pairs which do have another opening tag between them as instances of predicate \texttt{tag\_opening\_between/7}. The actual tag pairs making up well-formed XML elements are the derived using the rule on line 54 as those pairs of matching tags for which no corresponding instance of \texttt{tag\_opening\_between/7} exists.

\begin{lstlisting}[style=asp-code, label={lst:results-xml-parse-tag-pairs}, caption={Finding pairs of matching opening and closing XML tags.}]
% Establish a token ordering
token_before(TOK1, TOK1_LINE_NO, TOK1_FROM, TOK1_TO, 
        TOK2, TOK2_LINE_NO, TOK2_FROM, TOK2_TO) :- 
	token(TOK1, TOK1_LINE_NO, TOK1_FROM, TOK1_TO), 
	token(TOK2, TOK2_LINE_NO, TOK2_FROM, TOK2_TO), 
	TOK1_LINE_NO = TOK2_LINE_NO, TOK1_TO < TOK2_FROM.
token_before(TOK1, TOK1_LINE_NO, TOK1_FROM, TOK1_TO, 
        TOK2, TOK2_LINE_NO, TOK2_FROM, TOK2_TO) :- 
	token(TOK1, TOK1_LINE_NO, TOK1_FROM, TOK1_TO), 
	token(TOK2, TOK2_LINE_NO, TOK2_FROM, TOK2_TO), 
	TOK1_LINE_NO < TOK2_LINE_NO.

% Extract tag names
tag_opening(TAG_NAME, t(tag_open, TOK_VALUE), 
        LINE_NO, TAG_FROM, TAG_TO) :- 
	token(
            t(tag_open, TOK_VALUE), 
            LINE_NO, TAG_FROM, TAG_TO),
	&regex_matches
        [TAG_NAME_REGEX, TOK_VALUE](TAG_NAME, _, _),
	opening_tag_name_regex(TAG_NAME_REGEX).
tag_closing(TAG_NAME, t(tag_close, TOK_VALUE), 
        LINE_NO, TAG_FROM, TAG_TO) :- 
	token(
            t(tag_close, TOK_VALUE), 
            LINE_NO, TAG_FROM, TAG_TO),
	&regex_matches
        [TAG_NAME_REGEX, TOK_VALUE](TAG_NAME, _, _),
	closing_tag_name_regex(TAG_NAME_REGEX).

% We have an opening/closing tag pair if tag name is the same, 
% opening tag is before closing tag, 
% and no opening tag of same name between.
tag_opening_between(TAG_NAME, OPEN_LINE_NO, OPEN_FROM, OPEN_TO, 
        CLOSE_LINE_NO, CLOSE_FROM, CLOSE_TO) :- 
	tag_opening(TAG_NAME, OPEN_TOK, 
        OPEN_LINE_NO, OPEN_FROM, OPEN_TO),
	tag_closing(TAG_NAME, CLOSE_TOK, 
        CLOSE_LINE_NO, CLOSE_FROM, CLOSE_TO),
	token(OPEN_TOK, OPEN_LINE_NO, OPEN_FROM, OPEN_TO),
	token(CLOSE_TOK, CLOSE_LINE_NO, CLOSE_FROM, CLOSE_TO),
	token(INTM_OPEN_TOK, INTM_OPEN_LINE_NO, 
        INTM_OPEN_FROM, INTM_OPEN_TO),
	tag_opening(TAG_NAME, INTM_OPEN_TOK, 
        INTM_OPEN_LINE_NO, INTM_OPEN_FROM, INTM_OPEN_TO),
	token_before(
        OPEN_TOK, OPEN_LINE_NO, OPEN_FROM, OPEN_TO, 
        INTM_OPEN_TOK, INTM_OPEN_LINE_NO, 
        INTM_OPEN_FROM, INTM_OPEN_TO),
	token_before(
        INTM_OPEN_TOK, INTM_OPEN_LINE_NO, INTM_OPEN_FROM, 
        INTM_OPEN_TO, CLOSE_TOK, CLOSE_LINE_NO, 
        CLOSE_FROM, CLOSE_TO).
element(TAG_NAME, OPEN_LINE_NO, OPEN_FROM, OPEN_TO, 
        CLOSE_LINE_NO, CLOSE_FROM, CLOSE_TO) :- 
	tag_opening(TAG_NAME, OPEN_TOK, 
        OPEN_LINE_NO, OPEN_FROM, OPEN_TO),
	tag_closing(TAG_NAME, CLOSE_TOK, 
        CLOSE_LINE_NO, CLOSE_FROM, CLOSE_TO),
	token(OPEN_TOK, OPEN_LINE_NO, OPEN_FROM, OPEN_TO),
	token(CLOSE_TOK, CLOSE_LINE_NO, CLOSE_FROM, CLOSE_TO),
	token_before(OPEN_TOK, OPEN_LINE_NO, OPEN_FROM, OPEN_TO, 
        CLOSE_TOK, CLOSE_LINE_NO, CLOSE_FROM, CLOSE_TO),
	not tag_opening_between(
        TAG_NAME, OPEN_LINE_NO, OPEN_FROM, OPEN_TO, 
        CLOSE_LINE_NO, CLOSE_FROM, CLOSE_TO).
\end{lstlisting}    

Having parsed all XML elements, the next step is to determine which of these are terminal elements in the sense that they only contain text, as opposed to elements which consist of further XML elements. This is achieved using the code form Listing~\ref{lst:results-xml-parse-element-types}.

\begin{lstlisting}[style=asp-code, label={lst:results-xml-parse-element-types}, caption={Parsing details of XML elements}]
% Any string between two tags with no other opening 
% or closing tags between is content
any_tag_opening_between(TAG_NAME, OPEN_LINE_NO, OPEN_FROM, OPEN_TO, CLOSE_LINE_NO, CLOSE_FROM, CLOSE_TO) :- 
	element(TAG_NAME, OPEN_LINE_NO, OPEN_FROM, OPEN_TO, CLOSE_LINE_NO, CLOSE_FROM, CLOSE_TO),
	token(OPEN_TOK, OPEN_LINE_NO, OPEN_FROM, OPEN_TO),
	token(CLOSE_TOK, CLOSE_LINE_NO, CLOSE_FROM, CLOSE_TO),
	token_before(OPEN_TOK, OPEN_LINE_NO, OPEN_FROM, OPEN_TO, t(tag_open, INTM_TOK), INTM_LINE_NO, INTM_FROM, INTM_TO),
	token_before(t(tag_open, INTM_TOK), INTM_LINE_NO, INTM_FROM, INTM_TO, CLOSE_TOK, CLOSE_LINE_NO, CLOSE_FROM, CLOSE_TO).
any_tag_closing_between(TAG_NAME, OPEN_LINE_NO, OPEN_FROM, OPEN_TO, CLOSE_LINE_NO, CLOSE_FROM, CLOSE_TO) :- 
	element(TAG_NAME, OPEN_LINE_NO, OPEN_FROM, OPEN_TO, CLOSE_LINE_NO, CLOSE_FROM, CLOSE_TO),
	token(OPEN_TOK, OPEN_LINE_NO, OPEN_FROM, OPEN_TO),
	token(CLOSE_TOK, CLOSE_LINE_NO, CLOSE_FROM, CLOSE_TO),
	token_before(OPEN_TOK, OPEN_LINE_NO, OPEN_FROM, OPEN_TO, t(tag_close, INTM_TOK), INTM_LINE_NO, INTM_FROM, INTM_TO),
	token_before(t(tag_close, INTM_TOK), INTM_LINE_NO, INTM_FROM, INTM_TO, CLOSE_TOK, CLOSE_LINE_NO, CLOSE_FROM, CLOSE_TO). 
any_tag_between(TAG_NAME, OPEN_LINE_NO, OPEN_FROM, OPEN_TO, CLOSE_LINE_NO, CLOSE_FROM, CLOSE_TO) :-
	any_tag_opening_between(TAG_NAME, OPEN_LINE_NO, OPEN_FROM, OPEN_TO, CLOSE_LINE_NO, CLOSE_FROM, CLOSE_TO),
	any_tag_closing_between(TAG_NAME, OPEN_LINE_NO, OPEN_FROM, OPEN_TO, CLOSE_LINE_NO, CLOSE_FROM, CLOSE_TO).       
terminal(TAG_NAME, OPEN_LINE_NO, OPEN_FROM, OPEN_TO, CLOSE_LINE_NO, CLOSE_FROM, CLOSE_TO) :- 
	element(TAG_NAME, OPEN_LINE_NO, OPEN_FROM, OPEN_TO, CLOSE_LINE_NO, CLOSE_FROM, CLOSE_TO),
	not any_tag_between(TAG_NAME, OPEN_LINE_NO, OPEN_FROM, OPEN_TO, CLOSE_LINE_NO, CLOSE_FROM, CLOSE_TO).
% Extract content between tags (currently assumes start and end tags to be on the same line)
element_text(ELEMENT_NAME, FROM_POS, TO_POS, TEXT) :-
	terminal(ELEMENT_NAME, LINE_NO, OPEN_FROM, OPEN_TO, LINE_NO, CLOSE_FROM, CLOSE_TO),
	line(LINE_NO, FULL_LINE),
	&string_substring[FULL_LINE, OPEN_TO, CLOSE_FROM](TEXT),
	FROM_POS = pos(LINE_NO, OPEN_FROM),
	TO_POS = pos(LINE_NO, CLOSE_TO).
\end{lstlisting} 

In Listing~\ref{lst:results-xml-parse-element-types}, the rule on line 18 derives whether an XML element is a \emph{terminal} element in that is does not enclose any additional elements other than plain text. In order to determine this, we check if there are any opening or closing tags between the opening and closing tags of the element in question, see lines 3 through 15. For all terminal elements, we parse the text content of each individual element by taking a substring of the line containing the opening and closing tags ranging from after the last character of the opening tag to before the first character of the closing tag. (Note that this implementation assumes that content within a tag does not contain newline characters in order to keep the code a bit shorter.)

Every XML tag can also have any number of attributes. The parsing logic for attributes is shown in Listing~\ref{lst:results-xml-parse-attributes}. First, attributes are detected by checking every string that has been matched to be an opening XML tag against regular expressions matching attribute name an values, see the rule on line 2. Attributes derived using this approach are then associated with the corresponding \texttt{element/7} instance through the predicate \texttt{element\_attribute/8}. 

\begin{lstlisting}[style=asp-code, label={lst:results-xml-parse-attributes}, caption={Parsing XML attributes.}]
%% Extract attributes
tag_opening_attribute(TAG_NAME, t(tag_open, TOK_VALUE), LINE_NO, FROM, TO, attribute(NAME, VALUE)) :-
	tag_opening(TAG_NAME, t(tag_open, TOK_VALUE), LINE_NO, FROM, TO),
	&regex_matches[ATTRIBUTE_REGEX, TOK_VALUE](ATTRIBUTE_STR, _, _),
	attribute_regex(ATTRIBUTE_REGEX),
	&regex_matches[ATTRIBUTE_VALUE_REGEX, ATTRIBUTE_STR](VALUE, _, _), 
	attribute_value_regex(ATTRIBUTE_VALUE_REGEX),
	&regex_matches[ATTRIBUTE_NAME_REGEX, ATTRIBUTE_STR](NAME, _, _),
	attribute_name_regex(ATTRIBUTE_NAME_REGEX).

element_attribute(TAG_NAME, OPEN_LINE_NO, OPEN_FROM, OPEN_TO, CLOSE_LINE_NO, CLOSE_FROM, CLOSE_TO, attribute(NAME, VALUE)) :-
	element(TAG_NAME, OPEN_LINE_NO, OPEN_FROM, OPEN_TO, CLOSE_LINE_NO, CLOSE_FROM, CLOSE_TO),
	tag_opening_attribute(TAG_NAME, t(tag_open, _), OPEN_LINE_NO, OPEN_FROM, OPEN_TO, attribute(NAME, VALUE)).
\end{lstlisting} 

The final step in actual parsing of the XML content is to derive the complete hierarchy of all XML tags, i.e. determine which tags are direct children of another. This is done using the code from Listing~\ref{lst:results-xml-parse-hierarchy}. First, we establish elements that enclose other elements (i.e. opening and closing tag of the enclosed element between opening and closing tags of the enclosing element) using the rule on line 4. An element $e_1$ is not the parent element of an element $e_2$ enclosed by $e_1$ if there is a third element $e_i$ which is enclosed by $e_1$, but also encloses $e_2$. If, for any element encosing another element such an intermediate element exists, the rule from line 18 derives this as an instance of \texttt{element\_not\_parent/6}. An element $e_1$ \emph{is} the parent element of $e_2$ if it encloses $e_2$ and is not known to be not the parent. Element pairs fulfilling this definition are derived by the rule on line 22 and expressed as instances of \texttt{element\_parent/6}.

\begin{lstlisting}[style=asp-code, label={lst:results-xml-parse-hierarchy}, caption={Parsing XML attributes.}]
% Derive parent elements
% An element encloses another if the opening and closing tag of the enclosed 
% element lie fully between opening and cloising tags of the enclosing element
element_encloses(ENC, ENC_POS_FROM, ENC_POS_TO, LOWER, LOWER_POS_FROM, LOWER_POS_TO) :- 
	element(LOWER, LOWER_OPEN_LINE_NO, LOWER_OPEN_FROM, LOWER_OPEN_TO, LOWER_CLOSE_LINE_NO, LOWER_CLOSE_FROM, LOWER_CLOSE_TO),
	element(ENC, ENC_OPEN_LINE_NO, ENC_OPEN_FROM, ENC_OPEN_TO, ENC_CLOSE_LINE_NO, ENC_CLOSE_FROM, ENC_CLOSE_TO),
	token(LOWER_OPEN_TOK, LOWER_OPEN_LINE_NO, LOWER_OPEN_FROM, LOWER_OPEN_TO),
	token(LOWER_CLOSE_TOK, LOWER_CLOSE_LINE_NO, LOWER_CLOSE_FROM, LOWER_CLOSE_TO),
	token(ENC_OPEN_TOK, ENC_OPEN_LINE_NO, ENC_OPEN_FROM, ENC_OPEN_TO),
	token(ENC_CLOSE_TOK, ENC_CLOSE_LINE_NO, ENC_CLOSE_FROM, ENC_CLOSE_TO),
	token_before(ENC_OPEN_TOK, ENC_OPEN_LINE_NO, ENC_OPEN_FROM, ENC_OPEN_TO, LOWER_OPEN_TOK, LOWER_OPEN_LINE_NO, LOWER_OPEN_FROM, LOWER_OPEN_TO),
	token_before(LOWER_CLOSE_TOK, LOWER_CLOSE_LINE_NO, LOWER_CLOSE_FROM, LOWER_CLOSE_TO, ENC_CLOSE_TOK, ENC_CLOSE_LINE_NO, ENC_CLOSE_FROM, ENC_CLOSE_TO),
	ENC_POS_FROM = pos(ENC_OPEN_LINE_NO, ENC_OPEN_FROM),
	ENC_POS_TO = pos(ENC_CLOSE_LINE_NO, ENC_CLOSE_TO),
	LOWER_POS_FROM = pos(LOWER_OPEN_LINE_NO, LOWER_OPEN_FROM),
	LOWER_POS_TO = pos(LOWER_CLOSE_LINE_NO, LOWER_CLOSE_TO).
% An element E1 is parent of element E2 if E1 encloses E2 and there is no other element that encloses E2 and is enclosed by E1.   
element_not_parent(E, E_POS_FROM, E_POS_TO, P, P_POS_FROM, P_POS_TO) :- 
	element_encloses(P, P_POS_FROM, P_POS_TO, E, E_POS_FROM, E_POS_TO),
	element_encloses(P, P_POS_FROM, P_POS_TO, I, I_POS_FROM, I_POS_TO),
	element_encloses(I, I_POS_FROM, I_POS_TO, E, E_POS_FROM, E_POS_TO).
element_parent(E, E_POS_FROM, E_POS_TO, P, P_POS_FROM, P_POS_TO) :- 
	element_encloses(P, P_POS_FROM, P_POS_TO, E, E_POS_FROM, E_POS_TO),
	not element_not_parent(E, E_POS_FROM, E_POS_TO, P, P_POS_FROM, P_POS_TO).
\end{lstlisting}         

Last but not least, the \texttt{xml\_dom} module has a number of rules to project away positional information of tags which are not deemed to be of interest to the calling component. Listing~\ref{lst:results-xml-parse-projection} shows all rules used to achieve this. First, each parsed XML element is assigned a unique identifier represented using the predicate \texttt{dom\_element\_id/2} by the rule on line 1. Identifier terms are obtained from the \emph{enumeration literal} (see Section~\ref{subsubsec:alpha-arch-enum-resolution}) \texttt{id\_enum/3}. Based on the thereby established relation, versions of the predicates \texttt{element/7}, \texttt{element\_parent/6}, \texttt{element\_text/4}, \texttt{element\_attribute/8} in which the individual terms making up the respective element's position within the document are replaced with the corresponding identifier.

\begin{lstlisting}[style=asp-code, label={lst:results-xml-parse-projection}, caption={Deriving simplified representations of parsed XML content.}]
dom_element_id(e(NAME, POS_FROM, POS_TO), ID) :- 
	element(NAME, START_LINE_NO, START_FROM, START_TO, END_LINE_NO, END_FROM, END_TO),
	POS_FROM = pos(START_LINE_NO, START_FROM),
	POS_TO = pos(END_LINE_NO, END_TO),
	id_enum(element_ids, e(NAME, POS_FROM, POS_TO), ID).

dom_element(NAME, ID) :-
	dom_element_id(e(NAME, _, _), ID).	

dom_element_child(NAME, ID, CHILD_ID) :-
	dom_element(NAME, ID),
	dom_element_id(e(NAME, POS_FROM, POS_TO), ID),	
	dom_element_id(e(CHILD_NAME, CHILD_POS_FROM, CHILD_POS_TO), CHILD_ID),
	element_parent(CHILD_NAME, CHILD_POS_FROM, CHILD_POS_TO, NAME, POS_FROM, POS_TO).

dom_element_text(NAME, ID, TEXT) :-
	dom_element(NAME, ID),
	dom_element_id(e(NAME, POS_FROM, POS_TO), ID),
	element_text(NAME, POS_FROM, POS_TO, TEXT).	

dom_element_attribute(NAME, ID, ATTR_NAME, ATTR_VALUE) :-
	dom_element(NAME, ID),
	dom_element_id(e(NAME, pos(START_LINE_NO, START_FROM), pos(END_LINE_NO, END_TO)), ID),
	element_attribute(NAME, START_LINE_NO, START_FROM, _, END_LINE_NO, _, END_TO, attribute(ATTR_NAME, ATTR_VALUE)).	
\end{lstlisting}    

\section{Serializing graph-colorings to XML}
\label{sec:appendix-graphcol-to-xml}

Graph colorings calculated in the application from Chapter~\ref{chap:application-experiment} are converted to XML strings using the rules shown in Listing~\ref{lst:appendix-xml-graphcol-serialization}

\begin{lstlisting}[style=asp-code, label={lst:appendix-xml-graphcol-serialization}, caption={Generating XML strings for graph colorings.}]
#module threecol_to_xml (coloring/1 => {coloring_xmlstr/1}) {
    coloring_entry(vertex(ID, COL), TAIL) :- coloring(lst(col(ID, COL), TAIL)).
    coloring_entry(vertex(ID, COL), TAIL) :- coloring_entry(_, lst(col(ID, COL), TAIL)).
    coloring_entry_pred(ID, ID_PRED) :- coloring_entry(vertex(ID, _), lst(col(ID_PRED, _), _)).
    coloring_entry_has_pred(ID) :- coloring_entry_pred(ID, _).
    coloring_entry_is_pred(ID) :- coloring_entry_pred(_, ID).
    coloring_entry_first(ID) :- coloring_entry(vertex(ID, _), _), not coloring_entry_has_pred(ID).
    coloring_entry_last(ID) :- coloring_entry(vertex(ID, _), _), not coloring_entry_is_pred(ID).

    coloring_tag_open("<coloring>").
    coloring_tag_close("</coloring>").

    coloring_part(ID, STR) :- 
        coloring_entry(vertex(ID, COL), _), 
        &stdlib_string_concat["<vertex><id>", ID](S1),
        &stdlib_string_concat[S1, "</id><color>"](S2),
        &stdlib_string_concat[S2, COL](S3),
        &stdlib_string_concat[S3, "</color></vertex>"](STR).

    coloring_xmlstr_upto(STR, ID) :- 
        coloring_entry_first(ID), 
        coloring_part(ID, FIRST_PARTSTR),
        coloring_tag_open(OPENSTR),
        &stdlib_string_concat[OPENSTR, FIRST_PARTSTR](STR).

    coloring_xmlstr_upto(STR, ID) :- 
        coloring_entry_pred(ID, ID_PRED),
        coloring_xmlstr_upto(PREV_STR, ID_PRED),
        coloring_part(ID, PARTSTR),
        &stdlib_string_concat[PREV_STR, PARTSTR](STR).
        
    coloring_xmlstr(STR) :- 
        coloring_entry_last(ID),
        coloring_xmlstr_upto(PREV_STR, ID),
        coloring_tag_close(CLOSESTR),
        &stdlib_string_concat[PREV_STR, CLOSESTR](STR).	    
} 
\end{lstlisting}  
