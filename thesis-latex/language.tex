The Evolog language extends (non-disjunctive) ASP as defined in the ASP-Core2 standard~\cite{asp-core2} with facilities to communicate with and influence the "outside world" (e.g. read and write files, capture user input, etc.) as well as program modularization and reusability features, namely \emph{actions} and \emph{modules}.

\section{Actions in Evolog}
\label{sec:evolog-actions}

Actions allow for an ASP program to encode operations with \emph{side-effects} while maintaining fully declarative semantics. Actions are modelled in a functional style loosely based on the concept of monads as used in Haskell~\todo{cite something here!}. Intuitively, to maintain declarative semantics, actions need to behave as pure functions, meaning the result of executing an action (i.e. evaluating the respective function) must be reproducible for each input value across all executions. On first glance, this seems to contradict the nature of IO operations, which inherently depend on some state, e.g. the result of evaluating a function $getFileHandle(f)$ for a file $f$ will be different depending on whether $f$ exists, is readable, etc. However, at any given point in time - in other words, in a given state of the world - the operation will have exactly one result (i.e. a file handle or an error will be returned). A possible solution to making state-dependent operations behave as functions is therefore to make the state of the world at the time of evaluation part of the function's input. A function $f(x)$ is then turned into $f'(s, x)$  where $s$ represents a specific world state. The rest of this section deals with formalizing this notion of actions.

\todo{Define non-disjunctive ASP-Core2 in detail in preliminaries. Give detailed definition of all "standard ASP" elements referenced here!}

\subsection{Syntax}
\label{subsec:evolog-actions-syntax}

\begin{definition}[Action Rule, Action Program]
\label{def:action-rule-syntax}
An \emph{action rule} $R$ is of form
\[
	a_H : @t_{act} = act_{res} \leftarrow l_1,\ldots,l_n.
\]
where
\begin{itemize}
	\item $a_H$ is an atom called \emph{head atom},
	\item $t_{act}$ is a functional term called \emph{action term},
	\item $act_{res}$ is a term called \emph{(action-)result} term
	\item and $l_1,\ldots,l_n$ are literals constituting the \emph{body} of $R$.
\end{itemize}
An \emph{action program} $P$ is a set of (classic ASP-)rules and action rules.
\end{definition}

\subsection{Semantics}
\label{subsec:evolog-actions-semantics}

To properly define the semantics of an action program according to the intuition outlined at the start of this section, we first need to formalize our view of the "outside world" which action rules interact with. We call the world in which we execute a program a \emph{frame} - formally, action programs are always evaluated \emph{with respect to a given frame}. The behavior of actions is specified in terms of \emph{action functions}. The semantics (i.e. interpretations) of action functions in a program are defined by the respective frame.

\subsubsection{Action Rule Expansion}
\label{subsubsec:evolog-actions-semantics-expansion}

To get from the practical-minded action syntax from Definition~\ref{def:action-rule-syntax} to the formal representation of an action as a function of some state and an input, we use the helper construct of an action rule's \emph{expansion} to bridge the gap. Intuitively, the expansion of an action rule is a syntactic transformation that results in a more verbose version of the original rule called \emph{application rule} and a second rule only dependent on the application rule called \emph{projection rule}. A (ground) application rule's head atom uniquely identifies the ground instance of the rule that derived it. As one such atom corresponds to one action executed, we call a ground instance of an application rule head in an answer set an \emph{action witness}. 

\todo{define (classic ASP) grounding and substitutions in preliminaries}

\begin{definition}[Action Rule Expansion]
\label{def:action-rule-expansion}
Given a non-ground action rule $R$ with head atom $a_H$, action term $f_{act}(i_1,\ldots,i_n)$ and body  $B$ consisting of literals $l_1,\ldots,l_m$, the expansion of $R$ is a pair of rules consisting of an \emph{application rule} $R_{app}$ and \emph{projection rule} $R_{proj}$. $R_{app}$ is defined as
\[
	a_{res}(f_{act}, S, I, f_{act}(S, I)) \leftarrow l_1,\ldots,l_n.
\]
where $S$ and $I$ and function terms called \emph{state-} and \emph{input-}terms, respectively.
An action rule's state term has the function symbol $\mathit{state}$ and terms $fn(l_1),\ldots,fn(l_m)$, with the expression $fn(l)$ for a literal $l$ denoting a function term representing $l$. The (function-)term representation of a literal $p(t_1,\ldots,t_n)$ with predicate symbol $p$ and terms $t_1,\ldots,t_n$ uses $p$ as function symbol. For a negated literal $\mathit{not}~p(t_1,\ldots,t_n)$, the representing function term is $not(p(t_1,\ldots,p_n))$. The action input term is a "wrapped" version of all arguments of the action term, i.e. for action term $f_{act}(t_1,\ldots,t_n)$, the corresponding input term is $input(t_1,\ldots,t_n)$. The term $f_{act}(S, I)$ is called \emph{action application term}. \\
The projection rule $R_{proj}$ is defined as
\[
	a_H \leftarrow a_{res}(f_{act}, S, I, v_{res}).
\]
where $a_H$ is the head atom of the initial action rule $R$ and the (sole) body atom is the action witness derived by $R_{app}$, with the application term $f_{act}(S, I)$ replaced by a variable $v_{res}$ called \emph{action result variable}.
\end{definition}

Looking at the head of an action application rule of format $a_{res}(f_{act}, S, I, \mathit{t_{app}})$ with action $f_{act}$, state term $S$, input term $I$ and application term $t_{app}$, the intuitive reading of this atom is "The result of action function $f_{act}$ applied to state $S$ and input $I$ is $t_{app}$", i.e. the action application term $t_{app}$ is not a regular (uninterpreted) function term as in regular ASP, but an actual function call which is resolved using an interpretation function provided by a \emph{frame} during grounding.

%% NOTE: Action func. term is uninterpreted in nonground view! Application formally happens during grounding, i.e. action func is applied in grounding, ground version has action result.

\subsubsection{Grounding of Action Rules}
\label{subsubsec:evolog-actions-semantics-grounding}

Grounding, in the context of answer set programming, generally refers to the conversion of a program with variables into a semantically equivalent, variable-free, version. Action application terms as introduced in Definition \ref{def:action-rule-expansion} can be intuitively read as variables, in the sense that they represent the result of applying the respective action function. Consequently, all action application terms are replaced with the respective (ground) result terms defined in the \emph{frame} with respect to which the program is grounded.

\begin{definition}[Frame]
\label{def:evolog-frame}
Given an action program $P$ containing action application terms $A = \{a_1,\ldots,a_n\}$, a frame $F$ is an interpretation function such that, for each application term $f_{act}(S, I) \in A$ where $S \in H_{U}(P)^{*}$ and $I \in H_{U}(P)^{*}$, $F(f_{act}): H_{U}(P)^{*} \times H_{U}(P)^{*} \mapsto H_{U}(P)$.
\end{definition}

\begin{definition}[Grounding of action rules]
\label{def:evolog-grounding}
Grounding of Evolog rules (and programs) always happens \emph{with respect to a frame}. Given a frame $F$, an expanded action rule $r_a$ and a (grounding) substitution $\sigma$ over all body variables of application rule $r_{a_{app}}$, during grounding, every ground action application term $t_{app}\sigma$ resulting from applying substitution $\sigma$ is replaced with its interpretation according to $F$.
\end{definition}

Example \ref{ex:action-rule-expansion} demonstrates the expansion of an action rule as well as a compatible example frame for the respective action.

\begin{example}[Expansion and Frame]
\label{ex:action-rule-expansion}
Consider following Evolog Program $P$ which contains an action rule with action $a$:
\begin{align*}
	&p(a).~q(b).~r(c). \\
	&h(X, R) : @a(X, Z) = R \leftarrow p(X), q(Y), r(Z).
\end{align*}
The expansion of $R$ is:
\begin{align*}
	a_{res}(a, \mathit{state}(p(X), q(Y), r(Z)), \mathit{input}(X, Z), a(\mathit{state}(p(X), q(Y), r(Z)), \mathit{input}(X, Z))) \leftarrow& \\
	p(X), q(Y), r(Z).& \\
	h(X, R) \leftarrow a_{res}(a, \mathit{state}(p(X), q(Y), r(Z)), \mathit{input}(X, Z), R).&
\end{align*}
Furthermore, consider following frame $F$:
\[
	F(a) = \{a(\mathit{state}(p(a), q(b), r(c)), \mathit{input}(a, c)) \mapsto \mathit{success}(a, c)\}
\]
which assigns the result $\mathit{success}(a, c)$ to the action application term (i.e. function call $a(\mathit{state}(p(a), q(b), r(c)), \mathit{input}(a, c)))$. \\

Then, the ground program $P_{grnd}$ after action rule expansion is
\begin{align*}
	p(a).~q(b).~r(c).& \\
	a_{res}(a, \mathit{state}(p(a), q(b), r(c)), \mathit{input}(a, c), \mathit{success}(a, c)) \leftarrow p(a).~q(b).~r(c).& \\
	h(a, \mathit{success}(a, c)) \leftarrow a_{res}(a, \mathit{state}(p(a), q(b), r(c)), \mathit{input}(a, c), \mathit{success}(a, c)).&
\end{align*}
The sole model of $P$ with respect to frame $F$ is 
\begin{align*}
	M = \{p(a), q(b), r(c),& \\
	a_{res}(a, \mathit{state}(p(a), q(b), r(c)), \mathit{input}(a, c), \mathit{success}(a, c))& \\ h(a, \mathit{success}(a, c))\}&
\end{align*}
\end{example}

\subsubsection{Evolog Models}
\label{subsubsec:evolog-action-semantics-models}

Having introduced action rule expansions as well as frames, we now use these to extend the stable model semantics to Evolog programs.

\begin{definition}[Supportedness of Actions]
\label{def:evolog-supported-action}
Let $r_{app}$ be a non-ground action application rule with head $H = a_{res}(f_{act}, S, I, f_{act}(S, I))$, $F$ a frame, and $H_{grnd} = a_{res}(f_{act}, S_{grnd}, I_{grnd}, r)$ a ground instance of $H$ with $r$ being an arbitrary ground term.
Then, $H_{grnd}$ is \emph{supported by $F$}, if and only if $F$ contains a mapping of form $f_{act}(S_{grnd}, I_{grnd}) \mapsto r$, i.e. $r$ is a valid result of action function $f_{act}$ with arguments $S_{grnd}$ and $I_{grnd}$ according to frame $F$. 
We call a ground instance of an action rule \emph{supported by a frame} if the head of the corresponding application rule in the rule's expansion is supported by that frame.
\end{definition}

\begin{definition}[Evolog-Reduct]
\label{def:evolog-reduct}
Given a ground Evolog program $P$, a frame $F$ and a set of ground atoms $A$, the \emph{Evolog Reduct} of $P$ with respect to $F$ and $A$ $P_{F}^{A}$ is obtained from $P$ as follows:
\begin{enumerate}
	\item Remove all rules $r$ from $P$ that are "blocked", i.e. $A \not\models l$ for some negative body literal $l \in b^{-}(r)$.
	\item Remove all action application rules from $P$ which are not supported by $F$.
	\item Remove the negative body from all other rules.
\end{enumerate}
\end{definition}

Note that the reduct outlined in Definition \ref{def:evolog-reduct} extends the classic GL-reduct (see Definition \ref{def:prelims-asp-semantics-gl-reduct}) just by adding a check on action supportedness.

\begin{definition}[Evolog Model]
A herbrand interpretation $I$ of an Evolog Program is an \emph{Evolog Model} ("answer set") of an Evolog program $P$ with respect to a frame $F$ if and only if it is a minimal classical model of its Evolog-Reduct $P_{F}^{A}$. We denote the set of Evolog Models of a program $P$ as $\mathit{EM}(P)$.
\end{definition}

\subsection{Restrictions on Program Structure}
\label{subsec:evolog-actions-restrictions}

% Following goals are listed in introduction
%     \item declarative programs, i.e. order in which actions occur in code does not affect semantics,
%     \item actions behaving in a functional fashion, i.e. an action always gives the same result for the
%      same input. Especially, actions have to be idempotent in the sense that, for an ASP rule that is
%      associated with some action, the result of the action never changes, no matter how often the rule
%      fires.
%     \item transparent action execution, i.e. every action that is executed during evaluation of a program must be reflected in an answer set.

While the action semantics outlined so far addresses the requirements for both declarativity and functionality of actions outlined in the introduction (see \ref{goals:actions}), we haven't yet addressed the demand for transparency, i.e. that every action that is executed must be reflected in an aswer set of the respective program. With just the semantics outlined in Section \ref{subsec:evolog-actions-semantics}, it would be possible to write programs such as the one shown in Example \ref{ex:unsat-with-sideeffects} where an action rule can fire, but the program is unsatisfiable.

\begin{example}[Unsatisfiable Program with Side-effects]
\label{ex:unsat-with-sideeffects}
The program below contains an action rule that can fire (because $p(a)$ is true), but is also unsatisfiable due to the constraint in the last line.
\begin{align*}
	&p(a). \\
	&q(X) \leftarrow p(X). \\
	&act\_done(X, R) : @act(X) = R \leftarrow p(X). \\
	&\leftarrow q(X), act\_done(X, \_).
\end{align*}
\end{example}

This kind of programs raises some hard problems for implementations - given the contract that every side-effect of (i.e. action executed by) a program must be reported in an answer set, a solver evaluating the program from Example~\ref{ex:unsat-with-sideeffects} would have to "retract" action $act/1$ after finding that the program is unsatisfiable. Since it is generally not possible to "take back" side-effects (e.g. when some message is sent over a network broadcast to an unknown set of recipients), the only practical way to deal with this is to impose some conditions on programs with actions. Definition~\ref{def:evolog-actions-transparency} details this notion and introduces \emph{transparency} as a necessary condition for an Evolog program to be considered valid.

\begin{definition}[Action Transparency]
\label{def:evolog-actions-transparency}
An action rule $r_a$ of an Evolog program $P$ is \emph{transparent} if, for every expanded ground instance $gr_a$ it holds that, if $gr_a$ fires, then the head $h(gr_{a_{app}})$ of the repective application rule $gr_{a_{app}}$ is contained in an answer set of $P$. \\

For an Evolog program to be valid, all its action rules must be transparent.
\end{definition}

It follows from Definition \ref{def:evolog-actions-transparency} that only satisfiable programs can be transparent. Furthermore, note that Definition~\ref{def:evolog-actions-transparency} aims to be as permissive as possible in terms of program structure. In general, it can not be assumed that transparency of all rules in a program can be guaranteed up-front. Implementations may therefore impose further restrictions on what is considered a valid Evolog program. \todo{It might make sense to introduce the restriction P = CBP(P) for a valid program already here (anything more permissive cannot be linted without a theorem prover!)}


\section{Program Modularization in Evolog}
\label{sec:evolog-modules}

Modules aim to introduce new ways to re-use and unit-test to \gls{asp} code by enabling programmers to put frequently used sub-programs into \emph{modules} which can be used in multiple programs.

Conceptually, an Evolog Module is a special kind of external atom (see~\ref{def:prelims-asp-syntax-ext-atom}) that refers to an ASP solver being called with some program and input.

\subsection{Syntax}
\label{subsec:evolog-modules-syntax}

Definition~\ref{def:module-definition} outlines the syntax for an Evolog module, i.e. a reusable sub-program the can be used in other programs.

\begin{definition}[Module Definition]
\label{def:module-definition}
The following EBNF describes a \emph{module definition} in Evolog.
\begin{lstlisting}[style=code, label={lst:module-def-grammar}]
	module_definition : 
		"#module" ID "(" input_predicate "=>" 
				output_predicates ")" 
			"{" statements "}";
		
	input_predicate : predicate;
	output_predicates : 
		("*" | ("{" predicate ("," predicate)* "}"));
	predicate: ID "/" INTEGER;
	statements : (fact | classic-rule | constraint)*;
\end{lstlisting} 
In the context of the above grammar,
\begin{itemize}
	\item \texttt{ID} are identifiers according to Definition~\ref{def:prelims-asp-syntax-id},
	\item \texttt{INTEGER} are integers according to Definition~\ref{def:prelims-asp-syntax-int},
	\item and \texttt{classical-rule} refers to a regular, i.e. non-evolog, ASP rule according to Definition~\ref{def:prelims-asp-syntax-rule}
\end{itemize}
\end{definition}

Intuitively, a module definition establishes a name for the sub-program in question, and specifies how input and output are translated to and from the module program. Definition \ref{def:module-atom} introduces \emph{module atoms}, i.e. atoms in ASP rules that refer to modules.

\begin{definition}[Module Atom]
\label{def:module-atom}
Given a \emph{module name} $\mathit{mod}$, a positive integer $k$, \emph{input terms} $t_1,\ldots,t_n$ and \emph{output terms} $t_{n+1},\ldots,t_m$, the expression
\[
	\#\mathit{mod}\{k\}[t_1,\ldots,t_n](t_{n+1},\ldots,t_m)
\]
is called a \emph{module atom}. Syntacticaly, module atoms are regular atoms where $\#\mathit{mod}$ is the predicate symbol and $t_1,\ldots,t_m$ are argument terms. The integer $k$ is called the \emph{answer set limit}. It denotes the number of answer sets of the module program to return. If $k$ is omitted, all answer sets are returned.
\end{definition}

Example \ref{ex:3col-module} shows an encoding of the graph 3-coloring problem as an Evolog module.

\begin{example}[3-Coloring Module]
\label{ex:3col-module}
The following module definition encodes the 3-coloring problem for a graph $G = (V, E)$ as an Evolog module. Note that the module program assumes its input to be a single fact of form \texttt{graph(V, E)}, where \texttt{V} and \texttt{E} hold the lists of vertices and edges of $G$, respectively.
\begin{lstlisting}[style=asp-code, label={lst:3col-module}]
#module 3col(graph/2 => {col/2})  {
	% Unwrap input
	vertex_element(V, TAIL) :- graph(list(V, TAIL), _).
	vertex_element(V, TAIL) :- 
		vertex_element(_, list(V, TAIL)).
	vertex(V) :- vertex_element(V, _).
	edge_element(E, TAIL) :- graph(_, list(E, TAIL)).
	edge_element(E, TAIL) :- 
		edge_element(_, list(E, TAIL)).
	edge(V1, V2) :- edge_element(edge(V1, V2), _).
	
	% Make sure edges are symmetric
	edge(V2, V1) :- edge(V1, V2).
	
	% Guess colors
	red(V) :- vertex(V), not green(V), not blue(V).
	green(V) :- vertex(V), not red(V), not blue(V).
	blue(V) :- vertex(V), not red(V), not green(V).
	
	% Filter invalid guesses
	:- vertex(V1), vertex(V2), 
		edge(V1, V2), red(V1), red(V2).
	:- vertex(V1), vertex(V2), 
		edge(V1, V2), green(V1), green(V2).
	:- vertex(V1), vertex(V2), 
		edge(V1, V2), blue(V1), blue(V2).
	
	col(V, red) :- red(V).
	col(V, blue) :- blue(V).
	col(V, green) :- green(V).
}	
\end{lstlisting}	
\end{example}	

\subsection{Semantics}
\label{subsec:evolog-modules-semantics}

Intuitively, a module atom is a specialized external atom, and thus follows the general semantics of \emph{fixed interpretation literals} (see~\ref{def:prelims-asp-semantics-fixedinterpretation-literals}). Rather than mapping to some annotated Java Method as is the case with regular external atoms, a module atom is interpreted by calling an ASP solver.

In order to make input terms of a module atom available to the module implementation program as facts, a translation step is necessary. Definition~\ref{def:module-input-translation} describes this process in more detail.

\begin{definition}[Module Input Translation]
\label{def:module-input-translation}
Given a module definition $M$ with input predicate $p$ of arity $k$ and implementation program $P_M$ and a ground module atom $a_m$ referencing $M$ with input terms $t_1,\ldots,t_k$, then the ASP program that is solved to interpret $a_m$ is obtained by adding a fact $p(t_1,\ldots,t_k)$ to $P_M$. We denote this conversion of input terms to an atom as the \emph{"factification"} function $\mathit{fct(t_1,\ldots,t_k)}$.
\end{definition}	

Similar to module input translation, it is also necessary, to translate answer sets of an instantiated module program into output terms of the corresponding module atom. Every answer set gets converted into one set of terms, each - together with the inout terms - constituting one ground instance of the module atom. Predicates get converted into function symbols, and atoms become list items. Definition~\ref{def:asp-list-term} introduces the notion of a list term, which is used to represent lists of terms in ASP.

\begin{definition}[List Term]
\label{def:asp-list-term}
A list term is defined inductively as follows:
\begin{itemize}
		\item The constant denoting the empty list, $\mathit{empty\_list}$ is a list term.
		\item If $t$ is a term and $l$ is a list term, then $s_{lst}(t, l)$ is a list term.
\end{itemize}	
In the above, $s_{lst}$ is a reserved function symbol denoting list terms.
\end{definition}	

% \begin{definition}[List Aggregation]
% \label{def:asp-list-aggregation}
% Given a list term $lst$, a term $e$, and literal $lit$, the atom 
% \[
% 	$lst$ = \#\mathit{list}\{ e : lit\}
% \]
% is called a \emph{list aggregation atom}. The atom is true if and only if $lst$ is a list term containing all terms $e$ for which $lit$ is true. 
% \end{definition}

\begin{definition}[Module Output Translation]
\label{def:module-output-translation}
Given a module definition with output predicates $p_1,\ldots,p_n$, implementation program $P_M$ and an  answer sets $A$ of $P_M$, the output terms of a corresponding module atom are obtained by converting the answer set into a set of terms. For every output predicate specified in the module definition, a function term is created with the predicate symbol as function symbol and a single list term as argument. The list term is constructed by collecting all atoms of the respective predicate in $A$ into a list.
We denote this conversion of an answer set as the \emph{"termification"} function $\mathit{trm(A)}$.
\end{definition}

Using the conversions from Definitions~\ref{def:module-input-translation} and~\ref{def:module-output-translation}, we can now define the interpretation function for module atoms.

\begin{definition}[Module Atom Interpretation]
\label{def:module-atom-interpretation}
Given (non-ground) module atom $a_M$ with input terms $t_1,\ldots,t_k$ and output terms $t_1,\ldots,t_m$, a module definition $M$ with input predicate $p$ and output predicates $p_1,\ldots,p_m$, implementation program $P_M$, and ground substitution $\sigma$, the ground instance $grnd(a_M)$ obtained by applying $\sigma$ to $a_M$ is true iff $\{\exists A \in AS(P_M \cup fct(\sigma(t_1,\ldots,t_k))): trm(A) = \sigma(t_{1},\ldots,t_m) \}$, i.e. the module atom is true if and only if there exists an answer set of the module program together with the factified input terms that, when termified, equals the output terms of the module atom.
\end{definition}	

\begin{corollary}[Fixed Module Interpretation]
\label{cor:fixed-module-interpretation}
It follows from Definition~\ref{def:module-atom-interpretation} that module literals are \emph{fixed interpretation literals} according to Definition~\ref{def:prelims-asp-semantics-fixedinterpretation-literals}, where the interpretation function is the calculation described in Definition~\ref{def:module-atom-interpretation}. Since the implementation of a module does not change at runtime, the answer sets of the module program, i.e. the output terms of a corresponding module atom, only depend on the input terms supplied by a specific ground instance. The truth value of a given module literal $l$ is therefore independent of the rest of the program $l$ occurs in.
\end{corollary}

Example~\ref{ex:bin-packing-module} demonstrates the use of modules to find solutions to an application of the bin-packing problem.

\begin{example}[Bin-packing Module]
\label{ex:bin-packing-module}
Suppose a group of computer scientists $G = \{\mathit{bob}, \mathit{alice}, \mathit{dilbert}, \mathit{claire}, \mathit{cate}, \mathit{bill}, \mathit{carl}, \mathit{mary}\}$ plan to attend a conference in the neighboring town of Buzzwordville, which is 150 km distant. They have a number of cars at their disposal - Claire's camper van which seats 7 people, Dilbert's sedan seating 5, and Bob's roadster with 2 seats. Each car has different fuel efficiency, and the group wants to minimize the amount of fuel used.\\
To complicate matters even more, there are a number of additional constraints:
\begin{itemize}
	\item For insurance reasons, each car must be driven by its owner.
	\item Alice and Carl are a couple and must travel together.
	\item Dilbert and Mary are not on speaking terms and must not travel together.
\end{itemize}
It is intuitively clear, that the problem at hand is a specialization of the bin-packing problem (with differently sized bins). We therefore introduce the Evolog module \texttt{bin\_packing(instance\/2) \=\> \{item\_packed\/2\}}, which calculates solutions to the bin-packing decision problem, i.e. assignments of items to bins, such that all items are packed, and no bin capacity is exceeded. Listing~\ref{lst:bin-packing-module} shows the module in question.
\begin{lstlisting}[style=asp-code, label={lst:bin-packing-module}, caption={Bin-packing module}]
#module bin_packing(instance/2 => {item_packed/2}) {
	% Unpack input lists.
	bin_element(E, TAIL) :- 
		instance(lst(E, TAIL), _).
	bin_element(E, TAIL) :- 
		bin_element(_, lst(E, TAIL)).
	bin(B, S) :- 
		bin_element(bin(B, S), _).
	
	item_element(E, TAIL) :- 
		instance(_, lst(E, TAIL)).
	item_element(E, TAIL) :- 
		item_element(_, lst(E, TAIL)).
	item(I, S) :- 
		item_element(item(I, S), _).
	% For every item, guess an assignment to each bin.
	{ item_packed(I, B) : bin(B, _) } :- item(I, _).
	% An item may only be assigned to one bin at a time
	:- item_packed(I, B1), 
		item_packed(I, B2), 
		B1 != B2.
	% We must not exceed the capacity of any bin.
	capacity_used(B, C) :- 
		C = #sum{S : item(I, S), 
		item_packed(I, B)}, bin(B, _).
	:- capacity_used(B, C), 
		C > S, 
		bin(B, S).
	% Every item must be packed.
	item_packed_somewhere(I) :- 
		item_packed(I, _).
	:- item(I, _), 
	not item_packed_somewhere(I).
}	
\end{lstlisting}

Since the implementation from~\ref{lst:bin-packing-module} only solves the basic decision problem, we need additional code to represent the constraints in this specific variant of the problem and derive a cost value for each valid assignment. Listing~\ref{lst:bin-packing-constraints} shows a validation module that accepts a bin assignment and verifies it against the additional constraints.
\begin{lstlisting}[style=asp-code, label={lst:bin-packing-constraints}, caption={Custom Constraints}]
#module assignment_valid(assignment/1 => {assignment/1}) {
	assignment_element(lst(E, TAIL), E, TAIL) :- 
		assignment(lst(E, TAIL)).
	assignment_element(ASSGN, E, TAIL) :- 
		assignment_element(ASSGN, _, lst(E, TAIL)).
	assigned(I, B, A) :- 
		assignment_element(A, item_packed(I, B), _).
	
	owner_of(claire, van).
	owner_of(bob, roadster).
	owner_of(dilbert, sedan).
	
	% For insurance reasons, each car must be 
	% driven by its owner, i.e. for each car, 
	% if it is used, the owner must be assigned to it.
	car_in_use(C) :- assigned(_, C, A).
	:- car_in_use(C), owner_of(P, C), 
		assigned(P, C2, A), C2 != C.
	
	% Alice and Carl are a couple and 
	% must travel together.
	:- assigned(alice, C1, A), 
		assigned(carl, C2, A), C1 != C2.
	
	% Dilbert and Mary are not on speaking 
	% terms and must not travel together.
	:- assigned(dilbert, C, A), assigned(mary, C, A).
}	
\end{lstlisting}	
Finally, we calculate the cost of driving one kilometer for each assignment, and collect all assignments with the lowest possible cost value using predicate \texttt{optimal\_cost\_assignment/2}.
Listing~\ref{lst:bin-packing-main} shows the full program, excluding the module definitions given in Listings~\ref{lst:bin-packing-module} and~\ref{lst:bin-packing-constraints}.
\begin{lstlisting}[style=asp-code, label={lst:bin-packing-main}, caption={Main Program}]
person(bob).
person(alice).
person(dilbert).
person(claire).
person(cate).
person(bill).
person(carl).
person(mary).

car(van, 7).
car(roadster, 2).
car(sedan, 5).

distance(150).
cost_per_km(van, 5).
cost_per_km(roadster, 8).
cost_per_km(sedan, 3).

owner_of(claire, van).
owner_of(bob, roadster).
owner_of(dilbert, sedan).

% Create input instance for bin-packing module
instance(BINS, ITEMS) :- 
	BINS = #list{bin(C, S) : car(C, S)}, 
	ITEMS = #list{item(P, 1) : person(P)}.

% Calculate assignments and unfold list terms
assignment_validation_output(ASSGN_VALID) :- 
	instance(BINS, ITEMS), 
	#bin_packing[BINS, ITEMS](ASSGN), 
	#assignment_vali[ASSGN](ASSGN_VALID).
assignment(A) :- 
	assignment_validation_output(
		lst(assignment(A), lst_empty)).
assignment_element(lst(E, TAIL), E, TAIL) :- 
	assignment(lst(E, TAIL)).
assignment_element(ASSGN, E, TAIL) :- 
	assignment_element(ASSGN, _, lst(E, TAIL)).
assigned(I, B, A) :- 
	assignment_element(A, item_packed(I, B), _).

% Calculate the total travelling cost for each assignment
assignment_cost_per_km(A, C) :- 
	C = #sum{ CKM : cost_per_km(CAR, CKM), 
	assigned(_, CAR, A) }, 
	assignment(A).

% Find the minimum cost per km over all eligible 
% assignments and assignments with that cost
optimal_cost_per_km(C) :- 
	C = #min{CKM : assignment_cost_per_km(_, CKM)}.
optimal_cost_assignment(C, A) :- 
	optimal_cost_per_km(C), assignment_cost_per_km(A, C).	
\end{lstlisting}
\end{example}

The program from Listing~\ref{lst:bin-packing-main} has a single answer set, with 16 instances of \texttt{optimal\_cost\_assignment/2}, i.e. 16 possible assignments of people to cars which minimize the cost of a driven kilometer (Note that we use the notation $[a, b, c]$ for a list term holding terms $a, b, c$, and write terms of form $item\_packed(PERSON, CAR)$ as tuples $(PERSON, CAR)$, for brevity):
\begin{itemize}
	\item $optimal\_cost\_assignment(8, [(alice, sedan), (bill, sedan), (bob, sedan), (carl, sedan),\\(cate, sedan), (claire, van), (dilbert, sedan), (mary, van)])$
	\item $optimal\_cost\_assignment(8, [(alice, sedan), (bill, sedan), (bob, sedan), (carl, sedan),\\(cate, van), (claire, van), (dilbert, sedan), (mary, van)])$
	\item $optimal\_cost\_assignment(8, [(alice, sedan), (bill, sedan), (bob, van), (carl, sedan),\\(cate, sedan), (claire, van), (dilbert, sedan), (mary, van)])$
	\item $optimal\_cost\_assignment(8, [(alice, sedan), (bill, sedan), (bob, van), (carl, sedan),\\(cate, van), (claire, van), (dilbert, sedan), (mary, van)])$
	\item $optimal\_cost\_assignment(8, [(alice, sedan), (bill, van), (bob, sedan), (carl, sedan),\\(cate, sedan), (claire, van), (dilbert, sedan), (mary, van)])$
	\item $optimal\_cost\_assignment(8, [(alice, sedan), (bill, van), (bob, sedan), (carl, sedan),\\(cate, van), (claire, van), (dilbert, sedan), (mary, van)])$
	\item $optimal\_cost\_assignment(8, [(alice, sedan), (bill, van), (bob, van), (carl, sedan),\\(cate, sedan), (claire, van), (dilbert, sedan), (mary, van)])$
	\item $optimal\_cost\_assignment(8, [(alice, sedan), (bill, van), (bob, van), (carl, sedan),\\(cate, van), (claire, van), (dilbert, sedan), (mary, van)])$
	\item $optimal\_cost\_assignment(8, [(alice, van), (bill, sedan), (bob, sedan), (carl, van),\\(cate, sedan), (claire, van), (dilbert, sedan), (mary, van)])$
	\item $optimal\_cost\_assignment(8, [(alice, van), (bill, sedan), (bob, sedan), (carl, van),\\(cate, van), (claire, van), (dilbert, sedan), (mary, van)])$
	\item $optimal\_cost\_assignment(8, [(alice, van), (bill, sedan), (bob, van), (carl, van),\\(cate, sedan), (claire, van), (dilbert, sedan), (mary, van)])$
	\item $optimal\_cost\_assignment(8, [(alice, van), (bill, sedan), (bob, van), (carl, van),\\(cate, van), (claire, van), (dilbert, sedan), (mary, van)])$
	\item $optimal\_cost\_assignment(8, [(alice, van), (bill, van), (bob, sedan), (carl, van),\\(cate, sedan), (claire, van), (dilbert, sedan), (mary, van)])$
	\item $optimal\_cost\_assignment(8, [(alice, van), (bill, van), (bob, sedan), (carl, van),\\(cate, van), (claire, van), (dilbert, sedan), (mary, van)])$
	\item $optimal\_cost\_assignment(8, [(alice, van), (bill, van), (bob, van), (carl, van),\\(cate, sedan), (claire, van), (dilbert, sedan), (mary, van)])$
	\item $optimal\_cost\_assignment(8, [(alice, van), (bill, van), (bob, van), (carl, van),\\(cate, van), (claire, van), (dilbert, sedan), (mary, van)])$
\end{itemize}



\section{Relationship between Evolog- and Stable Model Semantics}

The extensions to the usual \gls{asp} programming language described in the previous sections extend the original formalism with a notion of an outside world (through frames in the context of which, actions can be expressed) as well as a grouping mechanism for sub-programs into modules.

\begin{theorem}[Extension]
\label{thm:extension}
Every ASP program $P$ is a valid and semantically equivalent Evolog program in the sense that - for any given frame $F$, the Evolog Models of $P$ are the same as its Answer Sets according to Stable Model Semantics~\ref{def:prelims-asp-semantics-answer-set}.
\end{theorem}

\begin{proof}
First, we assert that every "regular" ASP program P is also a syntactically valid Evolog program. This follows directly from the definitions - since Evolog only adds syntactic support for actions~\ref{def:action-rule-syntax}, but does not restrict regular ASP syntax, it follows that a syntactically correct ASP program is also a syntactically correct Evolog program. \\
Next, we show that for every regular ASP program $P$ and any frame $F$, the Evolog Models of $P$ are the same as its Stable Models, i.e. $\mathit{EM}(P)=\mathit{AS}(P)$: \\
Given any set of ground Atoms $A$ from the Herbrand Base $HB_P$ of $P$, $P$ is an Answer Set according to Stable Model Semantics if it is a minimal model of the GL-reduct~\ref{def:prelims-asp-semantics-gl-reduct} $P^{A}$ of $P$ w.r.t. $A$. We recall that $P^{A}$ is constructed as follows (see~\ref{def:prelims-asp-semantics-gl-reduct}):
\begin{itemize}
	\item remove from $P$ all rules $r$ that are "blocked", i.e. $A \not\models l$ for some literal $l \in b^{-}(r)$ 
	\item and remove the negative body of all other rules.
\end{itemize}
Furthermore, consider how the Evolog-Reduct of $P$ w.r.t. $A$ and (any arbitrary) Frame $F$, $P_{F}^{A}$, is constructed (see~\ref{def:evolog-reduct}):
\begin{itemize}
	\item Remove all rules $r$ from $P$ that are "blocked", i.e. $A \not\models l$ for some negative body literal $l \in b^{-}(r)$.
	\item Remove all action application rules from $P$ which are not supported by $F$.
	\item Remove the negative body from all other rules.
\end{itemize}
Since $P$ is a "regular", i.e. non-Evolog, ASP Program it contains no action application rules by definition. It is therefore clear, that for any set answer set candidate $A$ of $P$ and any Frame $F$, the GL-Reduct $P^{A}$ and Evolog-Reduct $P_{F}^{A}$ coincide. Any minimal model of the GL-reduct is therefore also a minimal model of the Evolog-Reduct (and vice-versa) and therefore the Evolog Models of P are identical with its Stable Models, i.e. $\mathit{EM}(P)=\mathit{AS}(P)$.
\end{proof}