% Copyright (C) 2014-2020 by Thomas Auzinger <thomas@auzinger.name>

\documentclass[draft,final]{vutinfth} % Remove option 'final' to obtain debug information.

% Load packages to allow in- and output of non-ASCII characters.
\usepackage{lmodern}        % Use an extension of the original Computer Modern font to minimize the use of bitmapped letters.
\usepackage[T1]{fontenc}    % Determines font encoding of the output. Font packages have to be included before this line.
\usepackage[utf8]{inputenc} % Determines encoding of the input. All input files have to use UTF8 encoding.

% Extended LaTeX functionality is enables by including packages with \usepackage{...}.
\usepackage{amsmath}    % Extended typesetting of mathematical expression.
\usepackage{amssymb}    % Provides a multitude of mathematical symbols.
\usepackage{mathtools}  % Further extensions of mathematical typesetting.
\usepackage{microtype}  % Small-scale typographic enhancements.
\usepackage[inline]{enumitem} % User control over the layout of lists (itemize, enumerate, description).
\usepackage{multirow}   % Allows table elements to span several rows.
\usepackage{booktabs}   % Improves the typesettings of tables.
\usepackage{subcaption} % Allows the use of subfigures and enables their referencing.
\usepackage[ruled,linesnumbered,algochapter]{algorithm2e} % Enables the writing of pseudo code.
\usepackage[usenames,dvipsnames,table]{xcolor} % Allows the definition and use of colors. This package has to be included before tikz.
\usepackage{nag}       % Issues warnings when best practices in writing LaTeX documents are violated.
\usepackage{todonotes} % Provides tooltip-like todo notes.
\usepackage{hyperref}  % Enables cross linking in the electronic document version. This package has to be included second to last.
\usepackage[acronym,toc]{glossaries} % Enables the generation of glossaries and lists fo acronyms. This package has to be included last.

% Define convenience functions to use the author name and the thesis title in the PDF document properties.
\newcommand{\authorname}{Forename Surname} % The author name without titles.
\newcommand{\thesistitle}{Title of the Thesis} % The title of the thesis. The English version should be used, if it exists.

% Set PDF document properties
\hypersetup{
    pdfpagelayout   = TwoPageRight,           % How the document is shown in PDF viewers (optional).
    linkbordercolor = {Melon},                % The color of the borders of boxes around crosslinks (optional).
    pdfauthor       = {\authorname},          % The author's name in the document properties (optional).
    pdftitle        = {\thesistitle},         % The document's title in the document properties (optional).
    pdfsubject      = {Subject},              % The document's subject in the document properties (optional).
    pdfkeywords     = {a, list, of, keywords} % The document's keywords in the document properties (optional).
}

\setpnumwidth{2.5em}        % Avoid overfull hboxes in the table of contents (see memoir manual).
\setsecnumdepth{subsection} % Enumerate subsections.

\nonzeroparskip             % Create space between paragraphs (optional).
\setlength{\parindent}{0pt} % Remove paragraph identation (optional).

\makeindex      % Use an optional index.
\makeglossaries % Use an optional glossary.
%\glstocfalse   % Remove the glossaries from the table of contents.

% Set persons with 4 arguments:
%  {title before name}{name}{title after name}{gender}
%  where both titles are optional (i.e. can be given as empty brackets {}).
\setauthor{Pretitle}{\authorname}{Posttitle}{female}
\setadvisor{Pretitle}{Forename Surname}{Posttitle}{male}

% For bachelor and master theses:
\setfirstassistant{Pretitle}{Forename Surname}{Posttitle}{male}
\setsecondassistant{Pretitle}{Forename Surname}{Posttitle}{male}
\setthirdassistant{Pretitle}{Forename Surname}{Posttitle}{male}

% For dissertations:
\setfirstreviewer{Pretitle}{Forename Surname}{Posttitle}{male}
\setsecondreviewer{Pretitle}{Forename Surname}{Posttitle}{male}

% For dissertations at the PhD School and optionally for dissertations:
\setsecondadvisor{Pretitle}{Forename Surname}{Posttitle}{male} % Comment to remove.

% Required data.
\setregnumber{0123456}
\setdate{01}{01}{2001} % Set date with 3 arguments: {day}{month}{year}.
\settitle{\thesistitle}{Titel der Arbeit} % Sets English and German version of the title (both can be English or German). If your title contains commas, enclose it with additional curvy brackets (i.e., {{your title}}) or define it as a macro as done with \thesistitle.
\setsubtitle{Optional Subtitle of the Thesis}{Optionaler Untertitel der Arbeit} % Sets English and German version of the subtitle (both can be English or German).

% Select the thesis type: bachelor / master / doctor / phd-school.
% Bachelor:
%\setthesis{bachelor}
%
% Master:
\setthesis{master}
\setmasterdegree{dipl.} % dipl. / rer.nat. / rer.soc.oec. / master
%
% Doctor:
%\setthesis{doctor}
%\setdoctordegree{rer.soc.oec.}% rer.nat. / techn. / rer.soc.oec.
%
% Doctor at the PhD School
%\setthesis{phd-school} % Deactivate non-English title pages (see below)

% For bachelor and master:
\setcurriculum{Logic and Computation}{Logic and Computation} % Sets the English and German name of the curriculum.

\begin{document}

\frontmatter % Switches to roman numbering.
% The structure of the thesis has to conform to the guidelines at
%  https://informatics.tuwien.ac.at/study-services

\addtitlepage{naustrian} % German title page (not for dissertations at the PhD School).
\addtitlepage{english} % English title page.
\addstatementpage

\begin{danksagung*}
\todo{Ihr Text hier.}
\end{danksagung*}

\begin{acknowledgements*}
\todo{Enter your text here.}
\end{acknowledgements*}

\begin{kurzfassung}
\todo{Ihr Text hier.}
\end{kurzfassung}

\begin{abstract}
\todo{Enter your text here.}
\end{abstract}

% Select the language of the thesis, e.g., english or naustrian.
\selectlanguage{english}

% Add a table of contents (toc).
\tableofcontents % Starred version, i.e., \tableofcontents*, removes the self-entry.

% Switch to arabic numbering and start the enumeration of chapters in the table of content.
\mainmatter

\chapter{Introduction}
\section{Answer Set Programming}

\gls{asp} is a formalism for declarative problem solving based on the Stable Model semantics introduced by Gelfond and Lifschitz in 1988~\cite{stable-models}. At its core, an \gls{asp} program is a collection of conditional rules along the lines of \emph{"if A holds true, then B must also hold"} as well as negative rules, so-called \emph{constraints}, which prohibit certain conditions, e.g. \emph{"If A has property p(A), then it cannot have property q(A)"}. Example \ref{ex:asp-first-intro} demonstrates the basic idea of \gls{asp} based on a program for course planning in a (very simplified) school setting.
\begin{example}
\label{ex:asp-first-intro}
Listing \ref{lst:school-planning} shows a knowledge base written in \gls{asp}, which encodes facts - things that are always true, such as \emph{"There is a subject called maths"} - and rules, e.g. \emph{A teacher T that is qualified to teach subject S, can be assigned to teach S}, about a simplified school domain.
\begin{lstlisting}[style=asp-code, caption=School planning in ASP, label={lst:school-planning}]
    % The curriculum consists of different courses.
    subject(german).
    subject(english).
    subject(maths).
    subject(biology).
    subject(history).

    % Each teacher can teach one or more subjects
    teacher(bob).
    can_teach(bob, english).
    can_teach(bob, maths).
    teacher(alice).
    can_teach(alice, maths).
    can_teach(alice, history).
    teacher(claire).
    can_teach(claire, german).
    can_teach(claire, history).
    teacher(joe).
    can_teach(joe, biology).
    can_teach(joe, history).

    % Assign subjects to teachers
    { teaches(T, S) } :- teacher(T), can_teach(T, S).
    % such that..
    % Every teacher teaches at least one subject
    :- teacher(T), not teaches(T, _).
    % Every subject is taught by exactly one teacher
    :- subject(S), not teaches(_, S).
    :- teaches(T1, S), teaches(T2, S), T1 != T2.
\end{lstlisting}    
Evaluating the above program using an \emph{answer set solver}, i.e. an interpreter for the \gls{asp} language, yields the following collection of so-called \emph{answer sets}:
\begin{itemize}
    \item $A_1 = \{teaches(bob,english), teaches(bob,maths), teaches(alice,history),\\teaches(claire,german), teaches(joe,biology)\}$
    \item $A_2 = \{teaches(bob,english), teaches(alice,maths), teaches(claire,german),\\teaches(claire,history), teaches(joe,biology)\}$
    \item $A_3 = \{teaches(bob,english), teaches(alice,maths), teaches(claire,german),\\teaches(joe,biology), teaches(joe,history)\}$
    \item $A_4 = \{teaches(bob,english), teaches(alice,maths), teaches(alice,history),\\teaches(claire,german), teaches(joe,biology)\}$
\end{itemize}
Intuitively, each answer set constitues a valid solution to the problem specified in the program in the sense that, if one adds the propositions form the answer set to the original program, all rules are fulfilled, while no constraint is violated. What sets \gls{asp} apart from many other logic programming formalisms is its \emph{multi-model semantics}, i.e. that it can express more than one solution to a problem as well as mutually exclusive solutions.
\end{example}    

As Example \ref{ex:asp-first-intro} illustrates, \gls{asp} offers a declarative and concise description language for complex problems, where formulating an imperative algorithm is non-trivial. Since its inception, \gls{asp} has been applied in a wide range of applications such as logistics~\cite{gioia-tauro}~\cite{train-scheduling}, automated music composition~\cite{blues-composition} and even spaceflight~\cite{space-shuttle}.
\todo{add more content here, e.g. from "answer set programming at a glance"}..

% First implementations 
% "blabla as example \ref{ex:bla} shows, ASP is well suited to all kinds of planning problems... it has successfully been used for ... more software-engineering-like tasks could also benefit from that kind of declarative brevity (hint: parsers, configuration, etc), but for ease of coding ,we don't want to write connectors all day...."

Apart from the established use of \gls{asp} as a formalization language for decision- or optimization problems, more recent developments in the field are increasingly targeted toward working with continuous external data in \gls{asp} programs. A prominent example from this direction of research is the stream reasoning framework LARS~\cite{lars}. Closely related to the concept of processing external data is the notion of actually influencing the outside world, for instance by writing data to a network buffer, from an \gls{asp} program, while still preserving declarative semantics. The DLV-extension Acthex~\cite{acthex} is an example of a system with basic action support. The goal of this thesis is to implement action support in the lazy-grounding solver Alpha~\cite{alpha}, along with a basic modularization concept, thus enabling the development of arbitrary programs fully within \gls{asp}.

\section{Actions and Modularization in ASP - Motivation}

\begin{example}
\label{ex:asp-binary-parsing}
Listing \ref{lst:binaryparsing} shows a simple \gls{asp} program that parses binary strings and calculates the corresponding decimal numbers. It uses external atoms implemented in Java for basic stirng operations: The predicate $str\_x\_xs$ takes inspiration from list syntax in Haskell and splits off the first character of a given string, e.g $str\_x\_xs["abc"]("a", "bc")$, while $stdlib_string_length$ and $stdlib_string_matches_regex$ test the length of a string and whether it matches some regular expression, respectively. 
\begin{lstlisting}[style=asp-code, caption=Parsing binary strings in ASP, label={lst:binaryparsing}]
    encoding_scheme("1", "0", "[01]+").

    % Helper - binstring_intm_decoded is the "internal" 
    % predicate we're using to recursively add up the 
    % bit values
    binstring_intm_decoded(START_STR, START_STR, 0) :- 
        binstr(START_STR). 
    
    % Handle the case where the current bit is set
    binstring_intm_decoded(START_STR, CURR_STR, CURR_VALUE) :- 
        binstring_intm_decoded(START_STR, LAST_STR, LAST_VALUE),
        \&str_x_xs[LAST_STR](HIGH_CODE, CURR_STR),
        \&stdlib_string_length[LAST_STR](LAST_LEN),
        CURR_VALUE = LAST_VALUE + 2 ** (LAST_LEN - 1),
        encoding_scheme(HIGH_CODE, _, REGEX),
        &stdlib_string_matches_regex[START_STR, REGEX].
    
    % Handle the case where the current bit is not set
    binstring_intm_decoded(START_STR, CURR_STR, CURR_VALUE) :- 
        binstring_intm_decoded(START_STR, LAST_STR, LAST_VALUE),
        \&str_x_xs[LAST_STR](LOW_CODE, CURR_STR),
        \&stdlib_string_length[LAST_STR](LAST_LEN),
        CURR_VALUE = LAST_VALUE,
        encoding_scheme(_, LOW_CODE, REGEX),
        \&stdlib_string_matches_regex[START_STR, REGEX].
    
    % These are the final values
    bin_number(BIN, DEC) :- 
        binstring_intm_decoded(BINSTR, "", DEC).    
\end{lstlisting}
\end{example}    

The program from Example \ref{ex:asp-binary-parsing} is a (very simple) example of a parser component that may be easier to implement in a declarative language than some (typically imperative) general purpose language such as Java or Python. With most current \gls{asp} solvers, if one wanted to write such a declarative parsing component, reading of parser inputs (typically from some file or stream) and writing of parser output would have to be done in some other language. Especially in applications where input and output is relatively simple, but parsing and data transformation logic is more involved, it would streamline application development to be able to write the whole application in one language. Furthermore, to re-use code parts - for example some generic parser as exhibited in Example \ref{ex:asp-binary-parsing} - one would currently have to keep that code in a separate \gls{asp} file and manually avoid conflicts in naming of predicates without any language-level support for code encapsulation and modularization. These considerations lead to the goals laid out in Section \ref{sec:problem-statement}. Section \ref{sec:state-of-the-art} gives an overview of the state of the art on action support and approaches to program modularizations in current \gls{asp} solving systems. Section \ref{sec:thesis-roadmap} gives an outline of the rest of the thesis.

\section{Problem Statement}
\label{sec:problem-statement}

\paragraph{Triggering actions from programs} \label{goals:actions}Most program flows follow a chain of events, each a consequence of its predecessor, e.g. "If there exists a file A, read it. If reading was successful, do something with the content. If the operation succeeds, write the result to file B". It is highly desirable to be able to write this kind of program in a declarative, logic-based language that can leverage the strengths of ASP for the "business logic" part. Specifically, the proposed action semantics should deliver
\begin{itemize}
    \item declarative programs, i.e. order in which actions occur in code does not affect semantics,
    \item actions behaving in a functional fashion, i.e. an action always gives the same result for the same input. Especially, actions have to be idempotent in the sense that, for an ASP rule that is associated with some action, the result of the action never changes, no matter how often the rule fires.
    \item transparent action execution, i.e. every action that is executed during evaluation of a program must be reflected in an answer set.
\end{itemize}

\paragraph{Program Modularization} While not formally connected, triggering actions from programs and modularization (i.e. plugable and re-usable sub-programs), intuitively complement each other in our current high-level design. Introducing a simple, easy-to-use module system is therefore the second goal of this work. It is, however, secondary in priority to definition and prototypical implementation of action support and may be reduced to a technical design draft if required due to time constraints.

\paragraph{Incremental Evaluation and Lazy Grounding} Experiences from existing systems for ASP application development such as ASAP~\cite{aspetris} or ACTHEX~\cite{acthex} show that, in order to achieve the evaluation performance necessary for use in real-world applications, ASP application code needs to be evaluated in an incremental fashion (rather than iteratively re-evaluating the whole program) whenever possible. The lazy-grounding architecture employed by ASP systems such as Alpha~\cite{alpha} offers an intuitive solution.

\section{State of the Art}
\label{sec:state-of-the-art}
%\todo{Mostly copied from proposal for now -- refine a bit!}
This work aims to blend action support with modularization in the context of lazy-grounding ASP solving - all three of these areas have seen a substantial amount of research in the past.

Both Clingo~\cite{clingo4} and DLVHEX, through the ACTHEX~\cite{acthex} extension, offer their own flavours of support for triggering actions from programs. While Clingo does not directly support actions as a dedicated feature - and therefore offers no strictly enforced semantics for this - similar behavior can be achieved using external functions and the reactive solving features first introduced in oClingo~\cite{oclingo}. ACTHEX has thoroughly defined semantics for actions. In the ACTHEX model, answer set search and action execution are separate steps, where executability of actions is only determined after answer sets are calculated. While this gives users a high degree of flexibility in working with actions, it does not directly lend itself to the idea of a general purpose language where program behavior may be influenced by continuous two-way communication between a program and its environment.

With regards to Modularity, i.e. the process of "assembling" an ASP program from smaller building blocks (i.e. modules), a comprehensive semantics for so-called ~\emph{nonmonotonic modular logic programs} has been introduced in~\cite{mlp-krennw} and~\cite{mlp-2009}. While it does not impose any restrictions on language constructs used in modules and recursion within and between modules, it also comes with rather high computational complexity and no easily available implementations so far. A more "lightweight" approach to modularization are \emph{Templates}~\cite{templates}. As the name implies, this purely syntactic approach aims to define isolated sub-programs that can be generically used to avoid code duplication throughout an application and is conceptually similar to the well-known generics in object-oriented languages such as C++ and Java. Templates are rewritten into regular ASP rules using an "explosion" algorithm which basically "instantiates" the template by generating the needed body atoms (and rules deriving them) wherever templates are used. While easy to implement and flexible, a potential disadvantage of this concept is that - due to its purely syntactical nature - programmers need to be on the watch for potential bugs arising from unintended cyclic dependencies or recursive use of templates (leading to potential non-termination of the explosion algorithm) themselves.
Yet another powerful toolset for modular application development is provided by Clingo's \emph{multi-shot-solving}~\cite{clingo-multishot} features. Clingo allows for parameterized sub-programs which are then repeatedly grounded in a process that is conceptually similar to the notion of module instantiation in ~\cite{modules-compositionality} and solved during solving of the overall program. However, as this "contextual grounding" needs to be programmatically controlled by an external application through Clingo's API, the inherent flexibility and usefulness for incremental solving of this approach is counterweighed by a high level of proficiency with and knowledge of the Clingo system necessary to leverage these capabilites.
The concept of \emph{lazy grounding}, i.e. interleaving of the - traditionally sequential - grounding and solving steps present in most prevalent ASP solvers, is relatively new. It has been spearheaded by the GASP~\cite{gasp} and ASPERIX~\cite{asperix-fw-chain} solvers which avoid calculating the full grounding of an input program by performing semi-naive bottom-up evaluation along the input's topologically sorted (non-ground) dependency graph. While efficient in terms of memory use, this approach cannot stand up to the solving performance of systems like DLV or Clingo which employ their knowledge of all possible ground rules to perform conflict-driven nogood learning (CDNL) as part of their solving algorithm to great effect. Alpha~\cite{alpha}, a more recent lazy-grounding solver, aims to bridge this gap in performance by employing CDNL-style solving techniques~\cite{lazy-cdnl} incrementally on partially ground program parts as part of its central ground-and-solve loop.

Conceptually, the common ingredient linking the - on first glance not directly connected - areas of actions in ASP, program modularization, and lazy grounding is a need for detailed static program analysis prior to solving, be it to detect potentially invalid action sequences, calculate module instantiation orders, or for up-front evaluation of stratified program parts in a lazy-grounding context. In addition, both actions and modularization can greatly benefit from - or even depend on - incremental evaluation facilities of a solver for efficient operation. Since lazy-grounding by its very definition embodies an incremental evaluation approach, it seems only natural to incorporate actions and modularization into a lazy-grounding solver's input language in order to provide ASP programmers with a powerful tool for application development. Alpha, with its good solving performance compared to other lazy-grounding systems, support for a large part of the current ASP-Core2 language standard, and active development status, appears the natural choice as the technical backbone of this work.

\section{Thesis Roadmap}
\label{sec:thesis-roadmap}

The core part of this work is the formal specification of the Evolog language in Chapter \ref{chap:evolog-language}, where we formally define Evolog's action semantics and modularization concept and make some observations on the relationship between Evolog programs versus regular \gls{asp} programs. Chapter \ref{chap:reference-implementation} gives and overview of how the formal specifications from Chapter \ref{chap:evolog-language} were implemented in the Alpha \gls{asp} solver, along with some examples of actual programs written in Evolog. Finally, Chapter \ref{chap:results} reflects on the experiences gained in using the implementation from Chapter \ref{chap:reference-implementation} for hands-on software development. We try to gauge the practical applicability of the implementation, highlight challenges yet to be adressed, and take a look at related work. % A short introduction to LaTeX.

\chapter{Additional Chapter}
\todo{Enter your text here.}

\backmatter

% Use an optional list of figures.
\listoffigures % Starred version, i.e., \listoffigures*, removes the toc entry.

% Use an optional list of tables.
\cleardoublepage % Start list of tables on the next empty right hand page.
\listoftables % Starred version, i.e., \listoftables*, removes the toc entry.

% Use an optional list of alogrithms.
\listofalgorithms
\addcontentsline{toc}{chapter}{List of Algorithms}

% Add an index.
\printindex

% Add a glossary.
\printglossaries

% Add a bibliography.
\bibliographystyle{alpha}
\bibliography{evolog}

\end{document}