% Copyright (C) 2014-2020 by Thomas Auzinger <thomas@auzinger.name>

\documentclass[draft,final]{vutinfth} % Remove option 'final' to obtain debug information.

% Load packages to allow in- and output of non-ASCII characters.
\usepackage{lmodern}        % Use an extension of the original Computer Modern font to minimize the use of bitmapped letters.
\usepackage[T1]{fontenc}    % Determines font encoding of the output. Font packages have to be included before this line.
\usepackage[utf8]{inputenc} % Determines encoding of the input. All input files have to use UTF8 encoding.

% Extended LaTeX functionality is enables by including packages with \usepackage{...}.
\usepackage{amsmath}    % Extended typesetting of mathematical expression.
\usepackage{amsthm}
\usepackage{listings}
\usepackage{amssymb}    % Provides a multitude of mathematical symbols.
\usepackage{mathtools}  % Further extensions of mathematical typesetting.
\usepackage{microtype}  % Small-scale typographic enhancements.
\usepackage[inline]{enumitem} % User control over the layout of lists (itemize, enumerate, description).
\usepackage{multirow}   % Allows table elements to span several rows.
\usepackage{booktabs}   % Improves the typesettings of tables.
\usepackage{subcaption} % Allows the use of subfigures and enables their referencing.
\usepackage[ruled,linesnumbered,algochapter]{algorithm2e} % Enables the writing of pseudo code.
\usepackage[usenames,dvipsnames,table]{xcolor} % Allows the definition and use of colors. This package has to be included before tikz.
\usepackage{nag}       % Issues warnings when best practices in writing LaTeX documents are violated.
\usepackage{todonotes} % Provides tooltip-like todo notes.
\usepackage{hyperref}  % Enables cross linking in the electronic document version. This package has to be included second to last.
\usepackage[acronym,toc]{glossaries} % Enables the generation of glossaries and lists fo acronyms. This package has to be included last.

% Define convenience functions to use the author name and the thesis title in the PDF document properties.
\newcommand{\authorname}{Michael Langowski} % The author name without titles.
\newcommand{\thesistitle}{Evolog - Actions and Modularization in Lazy-Grounding Answer Set Programming} % The title of the thesis. The English version should be used, if it exists.

% Set PDF document properties
\hypersetup{
    pdfpagelayout   = TwoPageRight,           % How the document is shown in PDF viewers (optional).
    linkbordercolor = {Melon},                % The color of the borders of boxes around crosslinks (optional).
    pdfauthor       = {\authorname},          % The author's name in the document properties (optional).
    pdftitle        = {\thesistitle},         % The document's title in the document properties (optional).
    pdfsubject      = {Subject},              % The document's subject in the document properties (optional).
    pdfkeywords     = {a, list, of, keywords} % The document's keywords in the document properties (optional).
}

\setpnumwidth{2.5em}        % Avoid overfull hboxes in the table of contents (see memoir manual).
\setsecnumdepth{subsection} % Enumerate subsections.

\nonzeroparskip             % Create space between paragraphs (optional).
\setlength{\parindent}{0pt} % Remove paragraph identation (optional).

\theoremstyle{definition}
\newtheorem{definition}{Definition}[section]

\newtheorem{theorem}{Theorem}[section]

\newtheorem{example}{Example}[section]

\lstdefinestyle{code}{
	basicstyle=\ttfamily
}

\lstdefinestyle{asp-code}{
	basicstyle=\ttfamily,
	frame=single,
	numbers=left,
    stepnumber=1
}



\makeindex      % Use an optional index.
\makeglossaries % Use an optional glossary.
%\glstocfalse   % Remove the glossaries from the table of contents.

% Set persons with 4 arguments:
%  {title before name}{name}{title after name}{gender}
%  where both titles are optional (i.e. can be given as empty brackets {}).
\setauthor{}{\authorname}{BSc.}{male}
\setadvisor{Prof. Dr.}{Thomas Eiter}{}{male}

% For bachelor and master theses:
\setfirstassistant{Dr.}{Antonius Weinzierl}{}{male}
%\setsecondassistant{Pretitle}{Forename Surname}{Posttitle}{male}
%\setthirdassistant{Pretitle}{Forename Surname}{Posttitle}{male}

% For dissertations:
%\setfirstreviewer{Pretitle}{Forename Surname}{Posttitle}{male}
%\setsecondreviewer{Pretitle}{Forename Surname}{Posttitle}{male}

% For dissertations at the PhD School and optionally for dissertations:
%\setsecondadvisor{Pretitle}{Forename Surname}{Posttitle}{male} % Comment to remove.

% Required data.
\setregnumber{01426581}
\setdate{01}{07}{2022} % Set date with 3 arguments: {day}{month}{year}.
\settitle{\thesistitle}{\thesistitle} % Sets English and German version of the title (both can be English or German). If your title contains commas, enclose it with additional curvy brackets (i.e., {{your title}}) or define it as a macro as done with \thesistitle.
\setsubtitle{}{} % Sets English and German version of the subtitle (both can be English or German).

% Select the thesis type: bachelor / master / doctor / phd-school.
% Bachelor:
%\setthesis{bachelor}
%
% Master:
\setthesis{master}
\setmasterdegree{dipl.} % dipl. / rer.nat. / rer.soc.oec. / master
%
% Doctor:
%\setthesis{doctor}
%\setdoctordegree{rer.soc.oec.}% rer.nat. / techn. / rer.soc.oec.
%
% Doctor at the PhD School
%\setthesis{phd-school} % Deactivate non-English title pages (see below)

% For bachelor and master:
\setcurriculum{Logic and Computation}{Logic and Computation} % Sets the English and German name of the curriculum.

\newcommand{\IDs}{\mathit{ID}}
\newcommand{\INTs}{\mathit{INT}}
\newcommand{\VARs}{\mathit{VAR}}

\newcommand{\NOT}{\mathit{not}}

\begin{document}

\frontmatter % Switches to roman numbering.
% The structure of the thesis has to conform to the guidelines at
%  https://informatics.tuwien.ac.at/study-services

\addtitlepage{naustrian} % German title page (not for dissertations at the PhD School).
\addtitlepage{english} % English title page.
\addstatementpage

\begin{danksagung*}
\todo{Ihr Text hier.}
\end{danksagung*}

\begin{acknowledgements*}
\todo{Enter your text here.}
\end{acknowledgements*}

\begin{kurzfassung}
\todo{Ihr Text hier.}
\end{kurzfassung}

\begin{abstract}
\todo{Enter your text here.}
\end{abstract}

% Select the language of the thesis, e.g., english or naustrian.
\selectlanguage{english}

% Add a table of contents (toc).
\tableofcontents % Starred version, i.e., \tableofcontents*, removes the self-entry.

% Switch to arabic numbering and start the enumeration of chapters in the table of content.
\mainmatter

\chapter{Introduction}
\section{Answer Set Programming}

\gls{asp} is a formalism for declarative problem solving based on the Stable Model semantics introduced by Gelfond and Lifschitz in 1988~\cite{stable-models}. At its core, an \gls{asp} program is a collection of conditional rules along the lines of \emph{"if A holds true, then B must also hold"} as well as negative rules, so-called \emph{constraints}, which prohibit certain conditions, e.g. \emph{"If A has property p(A), then it cannot have property q(A)"}. Example \ref{ex:asp-first-intro} demonstrates the basic idea of \gls{asp} based on a program for course planning in a (very simplified) school setting.
\begin{example}
\label{ex:asp-first-intro}
Listing \ref{lst:school-planning} shows a knowledge base written in \gls{asp}, which encodes facts - things that are always true, such as \emph{"There is a subject called maths"} - and rules, e.g. \emph{A teacher T that is qualified to teach subject S, can be assigned to teach S}, about a simplified school domain.
\begin{lstlisting}[style=asp-code, caption=School planning in ASP, label={lst:school-planning}]
    % The curriculum consists of different courses.
    subject(german).
    subject(english).
    subject(maths).
    subject(biology).
    subject(history).

    % Each teacher can teach one or more subjects
    teacher(bob).
    can_teach(bob, english).
    can_teach(bob, maths).
    teacher(alice).
    can_teach(alice, maths).
    can_teach(alice, history).
    teacher(claire).
    can_teach(claire, german).
    can_teach(claire, history).
    teacher(joe).
    can_teach(joe, biology).
    can_teach(joe, history).

    % Assign subjects to teachers
    { teaches(T, S) } :- teacher(T), can_teach(T, S).
    % such that..
    % Every teacher teaches at least one subject
    :- teacher(T), not teaches(T, _).
    % Every subject is taught by exactly one teacher
    :- subject(S), not teaches(_, S).
    :- teaches(T1, S), teaches(T2, S), T1 != T2.
\end{lstlisting}    
Evaluating the above program using an \emph{answer set solver}, i.e. an interpreter for the \gls{asp} language, yields the following collection of so-called \emph{answer sets}:
\begin{itemize}
    \item $A_1 = \{teaches(bob,english), teaches(bob,maths), teaches(alice,history),\\teaches(claire,german), teaches(joe,biology)\}$
    \item $A_2 = \{teaches(bob,english), teaches(alice,maths), teaches(claire,german),\\teaches(claire,history), teaches(joe,biology)\}$
    \item $A_3 = \{teaches(bob,english), teaches(alice,maths), teaches(claire,german),\\teaches(joe,biology), teaches(joe,history)\}$
    \item $A_4 = \{teaches(bob,english), teaches(alice,maths), teaches(alice,history),\\teaches(claire,german), teaches(joe,biology)\}$
\end{itemize}
Intuitively, each answer set constitues a valid solution to the problem specified in the program in the sense that, if one adds the propositions form the answer set to the original program, all rules are fulfilled, while no constraint is violated. What sets \gls{asp} apart from many other logic programming formalisms is its \emph{multi-model semantics}, i.e. that it can express more than one solution to a problem as well as mutually exclusive solutions.
\end{example}    

As Example \ref{ex:asp-first-intro} illustrates, \gls{asp} offers a declarative and concise description language for complex problems, where formulating an imperative algorithm is non-trivial. Since its inception, \gls{asp} has been applied in a wide range of applications such as logistics~\cite{gioia-tauro}~\cite{train-scheduling}, automated music composition~\cite{blues-composition} and even spaceflight~\cite{space-shuttle}.
\todo{add more content here, e.g. from "answer set programming at a glance"}..

% First implementations 
% "blabla as example \ref{ex:bla} shows, ASP is well suited to all kinds of planning problems... it has successfully been used for ... more software-engineering-like tasks could also benefit from that kind of declarative brevity (hint: parsers, configuration, etc), but for ease of coding ,we don't want to write connectors all day...."

Apart from the established use of \gls{asp} as a formalization language for decision- or optimization problems, more recent developments in the field are increasingly targeted toward working with continuous external data in \gls{asp} programs. A prominent example from this direction of research is the stream reasoning framework LARS~\cite{lars}. Closely related to the concept of processing external data is the notion of actually influencing the outside world, for instance by writing data to a network buffer, from an \gls{asp} program, while still preserving declarative semantics. The DLV-extension Acthex~\cite{acthex} is an example of a system with basic action support. The goal of this thesis is to implement action support in the lazy-grounding solver Alpha~\cite{alpha}, along with a basic modularization concept, thus enabling the development of arbitrary programs fully within \gls{asp}.

\section{Actions and Modularization in ASP - Motivation}

\begin{example}
\label{ex:asp-binary-parsing}
Listing \ref{lst:binaryparsing} shows a simple \gls{asp} program that parses binary strings and calculates the corresponding decimal numbers. It uses external atoms implemented in Java for basic stirng operations: The predicate $str\_x\_xs$ takes inspiration from list syntax in Haskell and splits off the first character of a given string, e.g $str\_x\_xs["abc"]("a", "bc")$, while $stdlib_string_length$ and $stdlib_string_matches_regex$ test the length of a string and whether it matches some regular expression, respectively. 
\begin{lstlisting}[style=asp-code, caption=Parsing binary strings in ASP, label={lst:binaryparsing}]
    encoding_scheme("1", "0", "[01]+").

    % Helper - binstring_intm_decoded is the "internal" 
    % predicate we're using to recursively add up the 
    % bit values
    binstring_intm_decoded(START_STR, START_STR, 0) :- 
        binstr(START_STR). 
    
    % Handle the case where the current bit is set
    binstring_intm_decoded(START_STR, CURR_STR, CURR_VALUE) :- 
        binstring_intm_decoded(START_STR, LAST_STR, LAST_VALUE),
        \&str_x_xs[LAST_STR](HIGH_CODE, CURR_STR),
        \&stdlib_string_length[LAST_STR](LAST_LEN),
        CURR_VALUE = LAST_VALUE + 2 ** (LAST_LEN - 1),
        encoding_scheme(HIGH_CODE, _, REGEX),
        &stdlib_string_matches_regex[START_STR, REGEX].
    
    % Handle the case where the current bit is not set
    binstring_intm_decoded(START_STR, CURR_STR, CURR_VALUE) :- 
        binstring_intm_decoded(START_STR, LAST_STR, LAST_VALUE),
        \&str_x_xs[LAST_STR](LOW_CODE, CURR_STR),
        \&stdlib_string_length[LAST_STR](LAST_LEN),
        CURR_VALUE = LAST_VALUE,
        encoding_scheme(_, LOW_CODE, REGEX),
        \&stdlib_string_matches_regex[START_STR, REGEX].
    
    % These are the final values
    bin_number(BIN, DEC) :- 
        binstring_intm_decoded(BINSTR, "", DEC).    
\end{lstlisting}
\end{example}    

The program from Example \ref{ex:asp-binary-parsing} is a (very simple) example of a parser component that may be easier to implement in a declarative language than some (typically imperative) general purpose language such as Java or Python. With most current \gls{asp} solvers, if one wanted to write such a declarative parsing component, reading of parser inputs (typically from some file or stream) and writing of parser output would have to be done in some other language. Especially in applications where input and output is relatively simple, but parsing and data transformation logic is more involved, it would streamline application development to be able to write the whole application in one language. Furthermore, to re-use code parts - for example some generic parser as exhibited in Example \ref{ex:asp-binary-parsing} - one would currently have to keep that code in a separate \gls{asp} file and manually avoid conflicts in naming of predicates without any language-level support for code encapsulation and modularization. These considerations lead to the goals laid out in Section \ref{sec:problem-statement}. Section \ref{sec:state-of-the-art} gives an overview of the state of the art on action support and approaches to program modularizations in current \gls{asp} solving systems. Section \ref{sec:thesis-roadmap} gives an outline of the rest of the thesis.

\section{Problem Statement}
\label{sec:problem-statement}

\paragraph{Triggering actions from programs} \label{goals:actions}Most program flows follow a chain of events, each a consequence of its predecessor, e.g. "If there exists a file A, read it. If reading was successful, do something with the content. If the operation succeeds, write the result to file B". It is highly desirable to be able to write this kind of program in a declarative, logic-based language that can leverage the strengths of ASP for the "business logic" part. Specifically, the proposed action semantics should deliver
\begin{itemize}
    \item declarative programs, i.e. order in which actions occur in code does not affect semantics,
    \item actions behaving in a functional fashion, i.e. an action always gives the same result for the same input. Especially, actions have to be idempotent in the sense that, for an ASP rule that is associated with some action, the result of the action never changes, no matter how often the rule fires.
    \item transparent action execution, i.e. every action that is executed during evaluation of a program must be reflected in an answer set.
\end{itemize}

\paragraph{Program Modularization} While not formally connected, triggering actions from programs and modularization (i.e. plugable and re-usable sub-programs), intuitively complement each other in our current high-level design. Introducing a simple, easy-to-use module system is therefore the second goal of this work. It is, however, secondary in priority to definition and prototypical implementation of action support and may be reduced to a technical design draft if required due to time constraints.

\paragraph{Incremental Evaluation and Lazy Grounding} Experiences from existing systems for ASP application development such as ASAP~\cite{aspetris} or ACTHEX~\cite{acthex} show that, in order to achieve the evaluation performance necessary for use in real-world applications, ASP application code needs to be evaluated in an incremental fashion (rather than iteratively re-evaluating the whole program) whenever possible. The lazy-grounding architecture employed by ASP systems such as Alpha~\cite{alpha} offers an intuitive solution.

\section{State of the Art}
\label{sec:state-of-the-art}
%\todo{Mostly copied from proposal for now -- refine a bit!}
This work aims to blend action support with modularization in the context of lazy-grounding ASP solving - all three of these areas have seen a substantial amount of research in the past.

Both Clingo~\cite{clingo4} and DLVHEX, through the ACTHEX~\cite{acthex} extension, offer their own flavours of support for triggering actions from programs. While Clingo does not directly support actions as a dedicated feature - and therefore offers no strictly enforced semantics for this - similar behavior can be achieved using external functions and the reactive solving features first introduced in oClingo~\cite{oclingo}. ACTHEX has thoroughly defined semantics for actions. In the ACTHEX model, answer set search and action execution are separate steps, where executability of actions is only determined after answer sets are calculated. While this gives users a high degree of flexibility in working with actions, it does not directly lend itself to the idea of a general purpose language where program behavior may be influenced by continuous two-way communication between a program and its environment.

With regards to Modularity, i.e. the process of "assembling" an ASP program from smaller building blocks (i.e. modules), a comprehensive semantics for so-called ~\emph{nonmonotonic modular logic programs} has been introduced in~\cite{mlp-krennw} and~\cite{mlp-2009}. While it does not impose any restrictions on language constructs used in modules and recursion within and between modules, it also comes with rather high computational complexity and no easily available implementations so far. A more "lightweight" approach to modularization are \emph{Templates}~\cite{templates}. As the name implies, this purely syntactic approach aims to define isolated sub-programs that can be generically used to avoid code duplication throughout an application and is conceptually similar to the well-known generics in object-oriented languages such as C++ and Java. Templates are rewritten into regular ASP rules using an "explosion" algorithm which basically "instantiates" the template by generating the needed body atoms (and rules deriving them) wherever templates are used. While easy to implement and flexible, a potential disadvantage of this concept is that - due to its purely syntactical nature - programmers need to be on the watch for potential bugs arising from unintended cyclic dependencies or recursive use of templates (leading to potential non-termination of the explosion algorithm) themselves.
Yet another powerful toolset for modular application development is provided by Clingo's \emph{multi-shot-solving}~\cite{clingo-multishot} features. Clingo allows for parameterized sub-programs which are then repeatedly grounded in a process that is conceptually similar to the notion of module instantiation in ~\cite{modules-compositionality} and solved during solving of the overall program. However, as this "contextual grounding" needs to be programmatically controlled by an external application through Clingo's API, the inherent flexibility and usefulness for incremental solving of this approach is counterweighed by a high level of proficiency with and knowledge of the Clingo system necessary to leverage these capabilites.
The concept of \emph{lazy grounding}, i.e. interleaving of the - traditionally sequential - grounding and solving steps present in most prevalent ASP solvers, is relatively new. It has been spearheaded by the GASP~\cite{gasp} and ASPERIX~\cite{asperix-fw-chain} solvers which avoid calculating the full grounding of an input program by performing semi-naive bottom-up evaluation along the input's topologically sorted (non-ground) dependency graph. While efficient in terms of memory use, this approach cannot stand up to the solving performance of systems like DLV or Clingo which employ their knowledge of all possible ground rules to perform conflict-driven nogood learning (CDNL) as part of their solving algorithm to great effect. Alpha~\cite{alpha}, a more recent lazy-grounding solver, aims to bridge this gap in performance by employing CDNL-style solving techniques~\cite{lazy-cdnl} incrementally on partially ground program parts as part of its central ground-and-solve loop.

Conceptually, the common ingredient linking the - on first glance not directly connected - areas of actions in ASP, program modularization, and lazy grounding is a need for detailed static program analysis prior to solving, be it to detect potentially invalid action sequences, calculate module instantiation orders, or for up-front evaluation of stratified program parts in a lazy-grounding context. In addition, both actions and modularization can greatly benefit from - or even depend on - incremental evaluation facilities of a solver for efficient operation. Since lazy-grounding by its very definition embodies an incremental evaluation approach, it seems only natural to incorporate actions and modularization into a lazy-grounding solver's input language in order to provide ASP programmers with a powerful tool for application development. Alpha, with its good solving performance compared to other lazy-grounding systems, support for a large part of the current ASP-Core2 language standard, and active development status, appears the natural choice as the technical backbone of this work.

\section{Thesis Roadmap}
\label{sec:thesis-roadmap}

The core part of this work is the formal specification of the Evolog language in Chapter \ref{chap:evolog-language}, where we formally define Evolog's action semantics and modularization concept and make some observations on the relationship between Evolog programs versus regular \gls{asp} programs. Chapter \ref{chap:reference-implementation} gives and overview of how the formal specifications from Chapter \ref{chap:evolog-language} were implemented in the Alpha \gls{asp} solver, along with some examples of actual programs written in Evolog. Finally, Chapter \ref{chap:results} reflects on the experiences gained in using the implementation from Chapter \ref{chap:reference-implementation} for hands-on software development. We try to gauge the practical applicability of the implementation, highlight challenges yet to be adressed, and take a look at related work. % A short introduction to LaTeX.

\chapter{Preliminaries}
\section{Answer Set Programming}

When speaking of \gls{asp}, we nowadays mostly refer to the language specfied by the ASP-Core2 standard~\cite{asp-core2}. It uses the \emph{stable model semantics} by Gelfond and Lifschitz~\cite{stable-models} as a formal basis and enhances it with support for advanced concepts such as disjunctive programs, aggregate literals and weak constraints. This chapter describes the input language supported by the Alpha solver, which will serve as the basis on which we will define the Evolog language.

\subsection{Syntax}
\label{subsec:prelims-asp-syntax}

\begin{definition}[Integer numeral]
\label{def:prelims-asp-syntax-int}
An \emph{integer numeral} in the context of an \gls{asp} program is a string matching the regular expression:
\begin{lstlisting}[style=code]
(-)?[0-9]+
\end{lstlisting}
The set of all valid integer numerals is denoted as $\INTs$.
\end{definition}

\begin{definition}[Identifier]
\label{def:prelims-asp-syntax-id}
An \emph{identifier} in the context of an \gls{asp} program is a string matching the regular expression:
\begin{lstlisting}[style=code]
[a-z][a-zA-Z0-9\_]*
\end{lstlisting}
The set of all valid identifiers is denoted as $\IDs$.
\end{definition}

\begin{definition}[Variable Name]
\label{def:prelims-asp-syntax-var}
A \emph{variable name} in the context of an \gls{asp} program is a string matching the regular expression:
\begin{lstlisting}[style=code]
[A-Z][a-zA-Z0-9\_]*
\end{lstlisting}
The set of all valid variable names is denoted as $\VARs$.
\end{definition}

\begin{definition}[Term]
\label{def:prelims-asp-syntax-term}
A \emph{term} is inductively defined as follows:
\begin{itemize}
	\item Any \emph{constant} $c \in (\INTs \cup \IDs)$ is a term.
	\item Any \emph{variable} $v \in \VARs$ is a term.
	\item Given terms $t_1, t_2$, any \emph{artihmetic expression} $t_1 \oplus t_2$ with $\oplus \in \{+, - , *, /, **\}$ is a term.
	\item Given terms $t_1, t_2$, any \emph{interval expression} $t_1 \ldots t_2$ is a term.
	\item For function symbol $f \in \IDs$ and argument terms $t_1, \ldots, t_n$, the \emph{functional expression} $f(t_1, \ldots, t_n)$ is a term.
\end{itemize}
\end{definition}

\begin{definition}[Subterms]
\label{def:prelims-asp-syntax-subterms}
Given a term $t$, the set of \emph{subterms} of $t$, $st(t)$, is defined as follows:
\begin{itemize}
	\item If $t$ is a \emph{constant} or \emph{variable}, $st(t) = \{t\}$.
	\item If $t$ is an \emph{arithmetic expression} $t_1 \oplus t_2$, $st(t) = st(t_1) \cup st(t_2)$.
	\item If $t$ is an \emph{interval expression} $t_1 \ldots t_2$, $st(t) = st(t_1) \cup st(t_2)$.
	\item If $t$ is a \emph{functional expression} with argument terms $t_1, \ldots, t_n$, $st(t) = st(t_1) \cup \ldots \cup st(t_n)$.
\end{itemize}
A term is called \emph{ground} if it is variable-free, i.e. none of its subterms is a variable.
\end{definition}

\begin{definition}[Basic Atom]
\label{def:prelims-asp-syntax-atom}
Given a predicate symbol $p \in \IDs$ and argument terms $t_1,\ldots,t_n$, the expression
\[
	p(t_1,\ldots,t_n)
\]
is called a \emph{atom}. An atom is ground if all of its argument terms are ground. A ground atom with predicate $p$ is called an \emph{instance} of $p$.
\end{definition}

\begin{definition}[Comparison Atom]
\label{def:prelims-asp-syntax-cmp-atom}
Given terms $t_1$ and $t_2$ and comparison operator $\odot$ where $\odot \in \{ <, \leq, =, \geq, >, \neq \}$, the expression
\[
	t_1 \odot t_2
\]
is called a \emph{comparison atom}. Syntactically, a comparison atom is a regular atom where the predicate symbol (i.e. comparison operator) is written in infix- rather than prefix-notation.
\end{definition}

\begin{definition}[External Atom]
\label{def:prelims-asp-syntax-ext-atom}
Given an \emph{external predicate name} $\mathit{ext}$, \emph{input terms} $t_1,\ldots,t_n$ and \emph{output terms} $t_{n+1},\ldots,t_m$, the expression
\[
	\&\mathit{ext}[t_1,\ldots,t_n](t_{n+1},\ldots,t_m)
\]
is called an \emph{external atom}. Syntacticaly, external atoms are regular atoms where $\&\mathit{ext}$ is the predicate symbol and $t_1,\ldots,t_m$ are argument terms.
\end{definition}

\begin{definition}[Literal]
\label{def:prelims-asp-syntax-literal}
A literal in \gls{asp} is an atom $a$ or ("default"-)negated atom $\NOT\ a$. Literals wrapping comparison- or external atoms are called \emph{fixed interpretation literals}. Given a literal $l$, the expression $pred(l)$ refers to the predicate of $l$ (e.g $pred(p(a)) = p/1$).
\end{definition}

\begin{definition}[Rule, Program]
\label{def:prelims-asp-syntax-rule}
A \emph{rule} is an expression of form
\[
	a_H \leftarrow b_1,\ldots,b_n.
\]
for $n \geq 0$, where the \emph{rule head} $a_H$ is an atom and the \emph{rule body} $b_1,\ldots,b_n$ is a set of literals. An \gls{asp} \emph{program} is a set of rules. A rule with an empty body is called a \emph{fact}. A rule is \emph{ground} if both its head atom and all of its body literals are ground. By the same reasoning, a program is ground if all of its rules are ground.\\
Given a rule $r$, we refer to the head of $r$ as $h(r)$ and the body of $r$ as $b(r)$. Furthermore, $b^+(r)$ is used to reference the set of \emph{positive body literals} of $r$, while $b^-(r)$ references the \emph{negative body literals}. 
\end{definition}

\begin{definition}[Constraint]
\label{def:prelims-asp-syntax-constraint}
A \emph{constraint} is a special form of rule, written as a rule with an empty head, i.e.
\[
	\leftarrow b_1,\ldots,b_n.
\]
It is syntactic sugar for
\[
	q \leftarrow b_1,\ldots,b_n, \NOT\ q.
\]
where $q$ is a propositional constant not occurring in any other rule in the program.
\end{definition}

\subsection{Semantics}
\label{subsec:prelims-asp-semantics}

\begin{definition}[Herbrand Universe]
\label{def:prelims-asp-semantics-hu}
The Herbrand Universe $HU_P$ of a Program $P$ is the set of all ground terms that can be constructed with respect to Definitions~\ref{def:prelims-asp-syntax-int},~\ref{def:prelims-asp-syntax-id} and \ref{def:prelims-asp-syntax-term}.
Note that most papers use stricter definitions of the Herbrand Universe where $HU_P$ consists only of terms constructible from constants occurring in $P$. The broader definition used here is chosen for ease of definition with respect to some of the extensions introduced in Section~\ref{sec:evolog-actions}.
\end{definition}

\begin{definition}[Herbrand Base]
\label{def:prelims-asp-semantics-hb}
The Herbrand Base $HB_P$ of a Program $P$ is the set of all ground atoms that can be constructed from the Herbrand Universe $HU_P$ according to definition~\ref{def:prelims-asp-syntax-atom}. 
\end{definition}

\begin{definition}[Herbrand Interpretation]
\label{def:prelims-asp-semantics-herbrand-interpretation}
A Herbrand Interpretation is a special form of first order interpretation where the domain of the interpretation is a Herbrand Universe and the interpretation of a term is the term itself, i.e. the corresponding element of $HU_P$. Intuitively, Herbrand Interpretations constitute listings of atoms that are true in a given program. Since the domain of a Herbrand Interpretation is always the Herbrand Universe $HU_P$, we only need to give a predicate interpretation for the predicates occurring in a program $P$ in oder to fully specify a Herbrand Interpretation. We can therefore denote Herbrand Interpretations as sets of atoms $I \subseteq HB_P$.
\end{definition}

\subsubsection{Grounding}
\label{subsubsec:prelims-grounding}
Given a program $P$ containing variables, \emph{grounding} refers to the process of converting $P$ into a semantically equivalent propositional, i.e. variable-free, program.

\begin{definition}[Substitution, adapted from~\cite{lazy-cdnl}]
\label{def:prelims-asp-semantics-substitution}
A substitution $\sigma: \VARs \mapsto (\IDs \cup \INTs)$ is a mapping from variables to constants. For a atom $a$, applying a substitution results in a substituted atom $a\sigma$ in which variables are replaced according to $\sigma$. Substitutions are applied to rules  by applying them to every individual atom or literal within the rule. By the same mechanism, we can apply substitutions to programs by applying the to all rules.
\end{definition}

\begin{definition}[Grounding]
\label{def:prelims-asp-semantics-grounding}
Given a rule $r$, the \emph{grounding} of $r$, $\mathit{grnd}(r)$, is a set of substitutions $S$, such that the set of ground rules resulting from applying the substitutions in $S$ is semantically equivalent to $r$. In a slight abuse of terminology, \emph{grounding} in this work also refers to the set of ground rules resulting from applying $S$ as well as the process of finding said set.
\end{definition}

\subsubsection{Stable Model Semantics}
\label{subsubsec:prelims-asp-semantics-stable-models}

\begin{definition}[Fixed interpretation literals]
\label{def:prelims-asp-semantics-fixedinterpretation-literals}	
Fixed interpretation literals, i.e. comparison- and external literals, respectively, are interpreted by means of a program-independent oracle function $f_O : H_{U}(P)^{*} \mapsto \{ \top, \bot \}$, i.e. a fixed interpretation literal with argument terms $t_1,\ldots,t_n$ has the same truth value under all interpretations.
\end{definition}

\begin{definition}[Truth of Atoms and Literals]
\label{def:prelims-asp-semantics-truth}
A positive ground literal $l$ with atom $a$ is true w.r.t. a Herbrand Interpretation $I$, i.e. $I \models l$ if
\begin{itemize}
	\item $a$ is a basic atom contained in $I$, i.e. $a \in I$,
	\item $a$ is a fixed interpretation literal with terms $t_1,\ldots,t_n$ and $f_O(t_1,\ldots,t_n) = \top$.
\end{itemize} 
For a negative ground literal $\NOT\ a$, the reverse holds, i.e. $I \models \NOT\ a$ if
\begin{itemize}
	\item $a$ is a basic atom not contained in $I$, i.e. $a \notin I$,
	\item $a$ is a fixed interpretation literal with terms $t_1,\ldots,t_n$ and $f_O(t_1,\ldots,t_n) = \bot$.
\end{itemize} 
A set of literals $L$ is true w.r.t. an interpretation $I$ if $I \models l$ holds for every literal $l \in L$. 
\end{definition}

\begin{definition}[Positive Logic Program]
\label{def:prelims-asp-semantics-positive-program}
A \emph{positive} logic program is a program according to Definition \ref{def:prelims-asp-syntax-rule}, where all rule bodies are positive, i.e. no rule body contains a negated atom.
\end{definition}

\begin{definition}[Immediate Consequence Operator, adapted from~\cite{asp-primer}]
\label{def:prelims-asp-semantics-immediate-consequence}
Given a Herbrand Interpretation $I$ and a ground positive logic program $P$, the immediate  consequence operator $T_P(I)$ defines a monotonic function $T_P: 2^{HB_P} \mapsto 2^{HB_P}$ such that
\[
	T_P(I) = \{h(r)\ |\ r \in P \land I \models b(r)\}
\]
i.e. the result set of applying $T_P$ with a Herbrand Interpretation $I$ contains the heads of all rules whose body is true under $I$.
\end{definition}

\begin{definition}[Least Model of positive logic programs]
\label{def:prelims-asp-semantics-least-model}
The least model $LM(P)$ of a (ground) positive logic program $P$ is the least fixpoint of the $T_P$ operator  of $P$, i.e. the set toward which the sequence $\langle T^{i}_{P} \rangle$, with $i \geq 0$, $T^{0}_P = \emptyset$ and $T^{i}_P = T_P(T^{i-1}_P)$ for $i \geq 1$, converges. The existence of said fixpoint and its characterisation as limit of $\langle T^{i}_{P} \rangle$ follow from the fixpoint theorems of Knaster, Tarski and Kleene, respectively.
\end{definition}

\begin{definition}[Gelfond-Lifschitz Reduct, adapted from~\cite{stable-models} and~\cite{asp-primer}]
\label{def:prelims-asp-semantics-gl-reduct}
Given a ground \gls{asp} program $P$ and a Herbrand Interpretation $I$, the \emph{Gelfond-Lifschitz-Reduct} ("GL-reduct") $P^{I}$ of $P$ with respect to $I$ is the program obtained by:
\begin{itemize}
	\item removing from $P$ all rules $r$ that are "blocked", i.e. $I \not\models l$ for some literal $l \in b^{-}(r)$ 
	\item and removing the negative body of all other rules.
\end{itemize}
Note that $P^{I}$ is a positive logic program.
\end{definition}

\begin{definition}[Answer Set~\cite{stable-models}~\cite{asp-primer}]
\label{def:prelims-asp-semantics-answer-set}
A Herbrand Interpretation $I$ of an \gls{asp} program $P$ is an \emph{answer set} or \emph{stable model} of $P$ iff it is the least model $LM(P^I)$ of the GL-reduct $P^I$ of $P$. We denote the set of Answer Sets of a program $P$ as $\mathit{AS}(P)$.
\end{definition}

\section{Lazy-grounding Answer-Set Solving}

The theoretical notion underpinning lazy-grounding answer-set solving is that of a \emph{Computation Sequence}, which is formalized in Definition~\ref{def:prelims-asp-semantics-compseq}. Intuitively, a computation sequence is a sequence of (ground) rules firing, such that, at the end, the atoms derived by the rules from the computation sequence (together with all facts of the program) constitute an answer set.

\begin{definition}[Computation Sequence, adapted from~\cite{lazy-cdnl} and~\cite{asperix-fw-chain}]
\label{def:prelims-asp-semantics-compseq}
Let $P$ be an \gls{asp} program and $S = (A_0,\ldots,A_{\infty})$ a sequence of assignments, i.e. herbrand interpretations denoted by a set of atoms assumed to be true, then $S$ is called a \emph{computation sequence} iff
\begin{itemize}
	\item $A_0 = \emptyset$
	\item $\forall i \geq 1: A_i \subseteq T_P(A_{i - 1})$, i.e. every $A_i$ is a consequence of its predecessor in the sequence,
	\item $\forall i \geq 1: A_{i - 1} \subseteq A_{i}$, i.e. S is monotonic,
	\item $A_{\infty} = \cup^{\infty}_{i = 0} A_i = T_P(A_{\infty})$, i.e. $S$ converges toward a fixpoint and
	\item $\forall i \geq 1: \forall a \ \in A_i \setminus A_{i - 1}, \exists r \in P: h(r) = a \land \forall j \geq i - 1: A_j \models a$, i.e. applicability of rules is persistent.
\end{itemize}
$A_{\infty}$ is an answer set of $P$ iff $S$ is a computation sequence. Note that there may exist an arbitrary number of computation sequences leading to the same answer set.
\end{definition}

It is easily observed that, in order to calculate computation sequences, one does not need a full grounding of the input program. Since facts are ground by definition, rules that only depend on facts can be grounded (and evaluated) based on facts alone. The same holds for rules depending on facts and rules from the previous step, etc. Example~\ref{ex:compseq-naive-calc} demonstrates how a computation sequence for a very simple positive program can be caluclated using a naive algorithm based on repeated application of the $T_P$-operator.

\begin{example}[Lazy-grounding a positive logic program]
\label{ex:compseq-naive-calc}
In order to illustrate the use of computation sequences for lazy-grounding, we consider the program $P$ in Listing~\ref{lst:compseq-naive-calc}. We denote the rule on line 2 as $r_1$ and the one on line 3 as $r_2$.
\begin{lstlisting}[style=asp-code, label={lst:compseq-naive-calc}, caption={A positive, non-ground program}]
p(a). p(b). p(c). q(b). q(c). q(d).
r(X) :- p(X), q(X).
t(X, Y) :- r(X), r(Y), X != Y.	
\end{lstlisting}
Starting from the set of facts on line 1, which we denote as $F$, we now look for rules that only depend on predicates in $F$ (i.e. body literals only have predicates for which we already know ground instances). The only applicable rule is $r_1$ for which we  can construct ground instances \texttt{r(b) :- p(b), q(b)} and \texttt{r(c) :- p(c), q(c)}. Continuing this process, now based on the set $F \cup \{r(b),r(c)\}$, we find ground instances \texttt{t(b, c) :- r(b), r(c), b != c} and \texttt{t(c, b) :- r(c), r(b), c != b} of $r_2$. We thus arrive at the following computation sequence:
\begin{align*}
	A_0 &= \{p(a), p(b), p(c), q(b), q(c), q(d)\},\\
	A_1 &= \{p(a), p(b), p(c), q(b), q(c), q(d), r(b), r(c)\},\\
	A_2 &= \{p(a), p(b), p(c), q(b), q(c), q(d), r(b), r(c), t(b, c), t(c, b)\}\\
\end{align*}	
The last element of the sequence, $A_2$ is the sole answer set of $P$.
\end{example}	

While the intuitive approach from Example~\ref{ex:compseq-naive-calc} clearly works for positive programs, things get more complicated when negation is involved. Consider the following program:
\begin{lstlisting}[style=asp-code]
	p(a). p(b). p(c). q(c).
	s(X) :- p(X), q(X).
	t(X) :- p(X), not s(X).
\end{lstlisting}
In order to arrive a the correct answer set $A = \{p(a), p(b), p(c), q(c), s(c), t(a), t(b)\}$, one has to first evaluate all rules that could potentially derive instances of $s/1$, before starting to evaluate the last rule, in order to end up with a valid computation sequence. Evaluation orders for this kind of programs can be calculated using the notion of \emph{stratification}~\cite{stratification} which is described in detail in Section~\ref{subsec:stratified-evaluation}.

\subsection{Structural Dependency Analysis and Stratified Evaluation}
\label{subsec:stratified-evaluation}

In a nutshell, a \emph{stratifiable} logic program is a program, for which we can calculate a partition into sub-programs, such that, when evaluating the sub-programs sequentially, one always ends up with a correct computation sequence. Definition~\ref{def:prelims-asp-semantics-stratification} formally characterizes stratifiable programs.

\begin{definition}[Stratifiable answer set program, as stated in~\cite{partial-eval}, adapted from\cite{stratification},~\cite{asp-primer}]
\label{def:prelims-asp-semantics-stratification}	
Given a program $P = \{r_{0}, ..., r_{n}\}$, $P$ is called \emph{stratifiable} iff there is a partition $P = P_{S_0} \cup\ ... \cup\ P_{S_n}$ so that for each $0 \le i \leq n$ the following holds:
\begin{itemize}
	\item For every positive body literal $l_{b}$ of every rule in $P_{S_i}$, every rule that derives instances of the predicate $\mathit{pred}(l_{b})$, i.e. where the predicate of the head atom $\mathit{pred}(a_H)$ is $\mathit{pred}(l_{b})$, is contained in some $P_{S_j}$ with $j \leq i$.
	\item For every negative body literal $\mathit{not}\ l_{b}$ of every rule in $P_{S_i}$, every rule that derivesinstances of the predicate $\mathit{pred}(l_{b})$, i.e. where the predicate of the head atom $\mathit{pred}(a_H)$is $\mathit{pred}(l_{b})$, is contained in some $P_{j}$ with $j < i$.
\end{itemize}
The individual subprograms $P_{S_i}$ are called strata. A \emph{stratification} $S = \{P_{S_0},\ldots, P_{S_n}\}$refers to the set of all strata making up a partition which satisfies the criteria above.
\end{definition}

The least model, i.e. the sole answer set, of a stratifiable program can be computed by sequentially applying the immediate consequence operator to every stratum, see Definition~\ref{def:prelims-asp-semantics-stratified-compseq}.

\begin{definition}[Least model of stratifiable programs, adapted from~\cite{asp-primer}]
\label{def:prelims-asp-semantics-stratified-compseq}
Let $P_{strat}$ be a stratifiable program for which a stratification $S = \{P_{S_0},\ldots,P_{S_n}\}$ exists and let $T_{P_{S_i}}$ be the immediate consequence operator for the sub-program defined by stratum $P_{S_i}$. Furthermore, for a model $M$, let $pr(M)$ be the program consisting of the facts representing the atoms in $M$.
Then the least model $LM(P_{strat})$ of $P_{strat}$ is defined as follows.
The sequence $\langle M_{S_i} \rangle,\ 0 \leq i \leq n$ with $M_{S_0} = \mathit{lfp}(T_{P_{S_0}})$ and $M_{S_i} = \mathit{lfp}(T_{(P_{S_{i}} \cup\ pr(M_{S_{i-1}}))})$ for all $1 \leq i \leq n$ defines the (iterative) least model for each stratum. The least model $LM(P_{strat})$ of program $P_{strat}$ is then the iterative least model of the highest stratum $M_{S_n}$, i.e. the end of the sequence $\langle M_{S_i} \rangle$.
\end{definition}

Since the evaluation mode described in Definition~\ref{def:prelims-asp-semantics-stratified-compseq} does not require any backtracking, i.e. the result of each step is part of the final result, we observe that the intermediate results of the immediate consequence operator yield a computation sequence for $P$. Example~\ref{ex:prelims-strat-eval} illustrates evaluation of a short stratifiable program.

\begin{example}[Evaluating a stratified program]
\label{ex:prelims-strat-eval}	
Consider the program $P$ from Listing~\ref{lst:prelims-strat-eval}.
\begin{lstlisting}[style=asp-code, label={lst:prelims-strat-eval}, caption={A stratifiable program.}]
p(a). p(b). p(c). q(c). q(d).

s(X) :- p(X), q(X).
t(X) :- p(X), not s(X).
u(X) :- q(X), not p(X).

v(X, Y) :- t(X), u(Y), X != Y.
\end{lstlisting}
A possible stratification $S$ of $P$ could look as follows:
\begin{align*}
	S_0 = \{&p(a).~p(b).~p(c).~q(c).~q(d). \\
			&s(X)~\leftarrow~p(X),~q(X).\} \\
	S_1 = \{&t(X)~\leftarrow~p(X),~not~s(X). \\
			&u(X)~\leftarrow~q(X),~not~p(X).\} \\
	S_2 = \{&v(X, Y)~\leftarrow~t(X),~u(Y),~X~!=~Y.\}
\end{align*}
In this case, the least model of $P$ can be calculated using the method from Definition~\ref{def:prelims-asp-semantics-stratified-compseq}. The individual results of the $T_P$ operator constitute a computation sequence.
\begin{align*}
	T_{P_{S_0}}^1 &= T_{P_{S_0}}(\emptyset) = \{p(a), p(b), p(c), q(c), q(d)\} \\
	T_{P_{S_0}}^2 &= T_{P_{S_0}}(T_{P_{S_0}}^1) = \{p(a), p(b), p(c), q(c), q(d), s(c)\} \\
	T_{P_{S_0}}^3 &= T_{P_{S_0}}(T_{P_{S_0}}^2) = \{p(a), p(b), p(c), q(c), q(d), s(c)\} = lfp(T_{P_{S_0}})\\
	\\
	T_{P_{E1}} &= T_{P_{S_1\cup pr(lfp(T_{P_{S_0}}))}}\\
	T_{P_{E1}}^1 &= T_{P_{E1}}(\emptyset) = \{p(a), p(b), p(c), q(c), q(d), s(c)\} \\
	T_{P_{E1}}^2 &= T_{P_{E1}}(T_{P_{E1}}^1) = \{p(a), p(b), p(c), q(c), q(d), s(c), t(a), t(b), u(d)\} \\
	T_{P_{E1}}^3 &= T_{P_{E1}}(T_{P_{E1}}^2) = \{p(a), p(b), p(c), q(c), q(d), s(c), t(a), t(b), u(d)\} = lfp(T_{P_{E1}}) \\
	\\
	T_{P_{E2}} &= T_{P_{S_2\cup pr(lfp(T_{P_{E1}}))}}\\
	T_{P_{E2}}^1 &= T_{P_{E2}}(\emptyset) = \{p(a), p(b), p(c), q(c), q(d), s(c), t(a), t(b), u(d)\} \\
	T_{P_{E2}}^2 &= T_{P_{E2}}(T_{P_{E2}}^1) = \{p(a), p(b), p(c), q(c), q(d), s(c), t(a), t(b), u(d), v(a, d), v(b, d)\} \\
	T_{P_{E2}}^3 &= T_{P_{E2}}(T_{P_{E2}}^2) = \{p(a), p(b), p(c), q(c), q(d), s(c), t(a), t(b), u(d), v(a, d), v(b, d)\} = lfp(T_{P_{E2}})\\
	\\
	LM(P) &= lfp(T_{P_{E2}})
\end{align*}	
\end{example}	

However, in general, ASP programs are not stratifiable. Example~\ref{ex:prelims-nonstrat} shows a typical non-stratifiable program.

\begin{example}[A non-stratifiable program]
\label{ex:prelims-nonstrat}
Consider a variant of the well-known 3-coloring problem for undirected graphs, where some vertices "stay blank", i.e. they are excluded from coloring. Essentially, prior to calculating actual colorings, we remove all ignored vertices and their corresponding edges from the graph and color only the resulting subgraph. Program $P_{col}$ in Listing~\ref{lst:prelims-3col-var} shows a possible encoding of this problem.
\begin{lstlisting}[style=asp-code, label={lst:prelims-3col-var}, caption={Graph 3-coloring with excluded vertices}]
vertex(a). vertex(b). vertex(c). vertex(d).
edge(a, b). edge(a, c). edge(a, d).
edge(b, c). edge(b, d). edge(c, d).
edge(X, Y) :- edge(Y, X).

exclude_vertex(d).
exclude_edge(V1, V2) :- edge(V1, V2), exclude_vertex(V1).
exclude_edge(V1, V2) :- exclude_edge(V2, V1).

coloring_vertex(V) :- vertex(V), not exclude_vertex(V).
coloring_edge(V1, V2) :- 
	edge(V1, V2), not exclude_edge(V1, V2).

% Guess colors
red(V) :- coloring_vertex(V), not green(V), not blue(V).
green(V) :- coloring_vertex(V), not red(V), not blue(V).
blue(V) :- coloring_vertex(V), not red(V), not green(V).

% Filter invalid guesses
:- coloring_vertex(V1), coloring_vertex(V2), 
	coloring_edge(V1, V2), red(V1), red(V2).
:- coloring_vertex(V1), coloring_vertex(V2), 
	coloring_edge(V1, V2), green(V1), green(V2).
:- coloring_vertex(V1), coloring_vertex(V2), 
	coloring_edge(V1, V2), blue(V1), blue(V2).
\end{lstlisting}
Clearly, $P_{col}$ is not stratifiable - the color assignment rules cyclically depend on each other through negated body literals. In other words, the rules on lines 15 to 17 are \emph{mutually exclusive}, i.e. a solver has to \emph{choose} which of the three should fire for a given ground instance of $coloring\_vertex/1$.
\end{example}

The program from Example~\ref{ex:prelims-nonstrat} can not be evaluated by iterative fixpoint-calculation as in Example~\ref{ex:prelims-strat-eval} because it is not stratifiable. However, stratified evaluation is still applicable for a part of the program. This partial evaluation approach uses \emph{splitting sets}, which are characterized in Definition~\ref{def:prelims-asp-semantics-splitting-set}. Intuitively, a splitting set $S$ is a set of atoms, such that if an atom $a$ is in $S$, also all atoms it depends on (i.e. body atoms of rules deriving $a$) are in $S$. By the splitting set theorem, a splitting sets lets us partition a program $P$ such that it can be evaluated incrementally, by first evaluating the bottom $B_U(P)$ and then the top $T_U(P)$. Since this section deals with non-ground programs specifically, we also give an adapted definition for splitting sets of nonground programs, see Definition~\ref{def:prelims-asp-semantics-nonground-splitting-set}. Example~\ref{ex:prelims-nonstrat-splitting-sets} applies the splitting set theorem to the program from Example~\ref{ex:prelims-nonstrat}.

\begin{definition}[Splitting Set, adapted from~\cite{splitting-sets}]
\label{def:prelims-asp-semantics-splitting-set}
Given a program $P$, a set of atoms $U$ is a \emph{splitting set} of $P$ if for every rule $r$, where $h(r) \in U$, also the atoms corresponding to all body literals of $r$ are in $U$. The set of rules corresponding to $U$, i.e. the rules definingthe atoms in $U$, is called \emph{bottom} of $P$ with respect to $U$, denoted as $B_U(P)$. Consequently, $P \setminus B_U(P)$is called \emph{top} of $P$, which is denoted as $T_U(P)$.\\
\end{definition}

\begin{theorem}[Splitting Set Theorem, adapted from~\cite{splitting-sets}]
Given a program $P$ and splitting set $U$ which splits $P$ into bottom $B_U(P)$ and top $T_U(P) = P \setminus B_U(P)$. Then a set of atoms $A$ is an answer set if and only if $A = X \cup Y$ for sets $X$ and $Y$ where
\begin{itemize}
	\item $X \in AS(B_U(P))$, i.e. $X$ is an answer set of bottom $B_U(P)$ and
	\item $Y \in AS(T_U(P) \cup X)$, i.e. $Y$ is an answer set of the program resulting from adding $X$ as facts to $T_U(P)$ 
\end{itemize}	
\end{theorem}	

\begin{definition}[Nonground splitting set]
\label{def:prelims-asp-semantics-nonground-splitting-set}
Given a non-ground program $P$, we denote splitting sets as sets of predicates rather than atoms. 
For a given predicate $p$, we denote as $dep(p)$ the \emph{dependencies} of $p$:
 \[
	dep(p) = \{q~|~\exists r \in P: pred(h(r)) = p \land \exists l \in b(r): pred(l) = q\}
\]
Intuitively, the dependencies of a predicate $p$ are all predicates of which all ground instances must be known in order to compute all ground instances of $p$.
A set $S$ of predicates $p_1,\ldots,p_n$ is a splitting set of $P$ if the following holds. 
\[
	\forall p:  p \in S \Rightarrow dep(p) \in S
\]
A non-ground splitting set $S_{ng}$ is sematically equivalent to the ground splitting set $S_{grnd}$ one gets by taking all ground instances of all predicates in $S_{ng}$ from the ground program $grnd(P)$.
\end{definition}

\begin{example}[Partial evaluation using splitting sets]
\label{ex:prelims-nonstrat-splitting-sets}
Given program $P_{col}$ from Listing~\ref{lst:prelims-3col-var} in Example~\ref{ex:prelims-nonstrat}, let $S$ be the following (non-ground) splitting set:
\begin{align*}
	S = \{&coloring\_vertex/1,~coloring\_edge/2,~vertex/1,~exclude\_vertex/1, \\
		  &edge/2,~exclude\_edge/2\}
\end{align*}
Using $S$, we can split $P_{col}$ into bottom $B_S(P_{col})$, which is shown in Listing~\ref{lst:prelims-3col-var-bot}, and top $T_S(P_{col})$ in Listing~\ref{lst:prelims-3col-var-top}.
\begin{lstlisting}[style=asp-code, label={lst:prelims-3col-var-bot}, caption={The bottom part of $P_{col}$}]
vertex(a). vertex(b). vertex(c). vertex(d).
edge(a, b). edge(a, c). edge(a, d).
edge(b, c). edge(b, d). edge(c, d).
edge(X, Y) :- edge(Y, X).
	
exclude_vertex(d).
exclude_edge(V1, V2) :- edge(V1, V2), exclude_vertex(V1).
exclude_edge(V1, V2) :- exclude_edge(V2, V1).
	
coloring_vertex(V) :- vertex(V), not exclude_vertex(V).
coloring_edge(V1, V2) :- 
	edge(V1, V2), not exclude_edge(V1, V2).
\end{lstlisting}
Looking at $B_S(P_{col})$, we can observe that this sub-program of $P_{col}$ is now stratified, and can therefore be evaluated using the straight-forward computation employed in Example~\ref{ex:prelims-strat-eval}. We arrive at the following model for $B_S(P_{col})$:
\begin{align*}
	LM(B_S(P_{col})) = \{&coloring\_vertex(b),~exclude\_vertex(d),~edge(b, d), \\
						 &edge(a, b),~edge(c, d),~edge(c, b),~coloring\_edge(b, c),\\
						 &coloring\_edge(a, c),~coloring\_edge(b, a),~edge(a, d),\\
						 &coloring\_edge(c, a),~vertex(c),~	vertex(a),~edge(d, b), \\
						 &exclude\_edge(d, b),~exclude\_edge(c, d),~exclude\_edge(a, d),\\
						 &coloring\_vertex(c),~exclude\_edge(b, d),~coloring\_vertex(a),\\
						 &edge(a, c),~edge(b, c),~edge(b, a),~edge(d, c),coloring\_edge(a, b)\\
						 &coloring\_edge(c, b),~vertex(d),~exclude\_edge(d, a),~vertex(b)\\
						 &edge(c, a),~exclude\_edge(d, c),~edge(d, a)\}
\end{align*}		
\begin{lstlisting}[style=asp-code, label={lst:prelims-3col-var-top}, caption={The top part of $P_{col}$}]
% Guess colors
red(V) :- coloring_vertex(V), not green(V), not blue(V).
green(V) :- coloring_vertex(V), not red(V), not blue(V).
blue(V) :- coloring_vertex(V), not red(V), not green(V).
	
% Filter invalid guesses
:- coloring_vertex(V1), coloring_vertex(V2), 
	coloring_edge(V1, V2), red(V1), red(V2).
:- coloring_vertex(V1), coloring_vertex(V2), 
	coloring_edge(V1, V2), green(V1), green(V2).
:- coloring_vertex(V1), coloring_vertex(V2), 
	coloring_edge(V1, V2), blue(V1), blue(V2).
\end{lstlisting}	
In order to compute the answer sets of the complete program, by the splitting set theorem, we only have to compute the answer sets of $T_S(P_{col}) \cup pr(LM(B_S(P_{col})))$. A simple (but computationally expensive) way to do this would be to perform an exhaustive search over all combinations of ground instances of the rules from lines 2 to 4 and drop all model candidates where any of the constraints fire.
We finally get the following answer sets (filtered for predicates $red/1$, $green/1$ and $blue/1$):
\begin{align*}
	A_1 &= \{blue(c),~green(b),~red(a)\}\\
	A_2 &= \{blue(b),~green(c),~red(a)\}\\
	A_3 &= \{blue(b),~green(a),~red(c)\}\\
	A_4 &= \{blue(c),~green(a),~red(b)\}\\
	A_5 &= \{blue(a),~green(b),~red(c)\}\\
	A_6 &= \{blue(a),~green(c),~red(b)\}
\end{align*}	
\end{example}	

Expanding on Example~\ref{ex:prelims-nonstrat-splitting-sets}, it is clear that a stratifiable bottom with respect to some splitting set always exists - every program has at least the empty set as a trivial splitting set. We call the maximal stratifiable sub-program that only depends on facts (i.e. is a bottom w.r.t some splitting set) \gls{cbp}, see Definition~\ref{def:prelims-asp-semantics-cbp}.

\begin{definition}[Common Base Program, adapted from~\cite{partial-eval}]
\label{def:prelims-asp-semantics-cbp}
Given a splitting set $S$ of program $P$, the bottom $B_S(P)$ is called \emph{common base program}, i.e. $CBP(P)$ if it is stratified and maximal in the sense that adding any further rule to $B_S(P)$ would destroy the property of $B_S(P)$ of being stratified.
\end{definition}

What remains is to solve the remaining, non-stratifiable part of a program. State-of-the-art two-phased solvers such as Clingo employ solving techniques based on~\gls{cdnl} for this~\cite{clasp-cdnl}. The basic idea of \gls{cdnl} is to encode a ground program as a set of constraints called \emph{nogoods}, i.e. sets of literals that must not be contained in any answer set. Based on these nogoods, truth assignments over $H_B(P)$ are "guessed" and conflicting guesses are used to potentially learn new nogoods using techniques originating in SAT-solving. However, these techniques normally require an input program to be fully ground, and therefore do not lend themselves to lazy-grounding approaches. Alpha uses a solving approach that is inspired by \gls{cdnl}, but makes a number of modifications to apply the same principles in a lazy-grounding scenario. Section~\ref{subsec:prelims-lazygrounding-alpha-cdnl} gives a highlevel overview of a \gls{cdnl}-based answer set search procedure that has first been implemented in Alpha~\cite{lazy-cdnl}.

\subsection{Employing CDNL-based answer set seach for lazy-grounding solvers}
\label{subsec:prelims-lazygrounding-alpha-cdnl}

Just like in traditional two-phased solvers, the lazy-grounding \gls{cdnl}-approach uses \emph{nogoods} to encode rules and constraints. However, since a full grounding of the program to solve is not available, some modifications are necessary. Without a full grounding, a solver must be able to distinguish between atoms in an assignment (i.e. answer set candidate) that are \emph{true because some rule fired}, and atoms that \emph{must be true because some constraint does not permit otherwise} (but haven't been derived by a rule). Lazy-grounding solvers introduce a third truth value, \emph{must-be-true} (MBT), to reflect this. For an assignment to be an answer set, it must not contain any MBT values. Definitions~\ref{def:prelims-alpha-nogood} and~\ref{def:prelims-alpha-nogood-ruleencoding} formally describe nogoods and their use to represent rules in this solving strategy.

\begin{definition}[Alpha-Assignment, Nogood~\cite{lazy-cdnl}]
\label{def:prelims-alpha-nogood}	
An \emph{Alpha-Assignment}, in the following just \emph{assignment} is a sequence $A = \{sl1,\ldots,sl_n\}$ of literals $l_1\ldots,l_n$ preceded by a \emph{sign} $s \in \{T, M, F\}$. Intuitively, a signed literal $sl_i$ assigns the truth value $s$ to literal $l$. Possible truth values are \emph{true} ($T$), \emph{false} ($F$) and \emph{must-be-true} ($M$). Must-be-true is used as truth value for literals that are deemed by the solver to necessarily be true (e.g. because of a constraint precluding any other truth value), but there is no known firing rule (yet) which would allow assigning $T$. The \emph{boolean projection} $A^{\mathcal{B}}$ of an assignment $A$ is defined as: 
\[
	A^{\mathcal{B}} = \{ Ta~|~Ta \in A \lor Ma \in A\} \cup \{ Fa~|~Fa \in A\}
\]
\\
A \emph{nogood}, in the variant used here, is a set of signed literals intuitively characterizing a set of literals that may not occur in an assignment with the given truth values:
\[
	ng = \{ sl_1,\ldots,sl_n \}, s \in \{T, F\}
\]
Different from how nogoods are defined in classic two-phased solvers, a nogood may optionally have a specifically designated "head" literal. Given a nogood where $sl_i$ is the head literal, we denote this as: $ng = \{ sl_1,\ldots,sl_n \}_i$. For a signed literal $sl$, we denote the literal with inverse sign as $\overline{sl}$.
An assignment $A$ \emph{satisfies} a nogood $ng$ if $ng \nsubseteq A^{\mathcal{B}}$. Furthermore, an assignment $A$ is a \emph{solution} to a set $\Delta$ of nogoods if $\{a~|~Ta \in A^{\mathcal{B}}\} \cap \{a~|~Fa \in A^{\mathcal{B}}\} = \emptyset$,  $\{a~|~Ta \in A^{\mathcal{B}}\} \cup \{a~|~Fa \in A^{\mathcal{B}}\} = A^{\mathcal{B}}$ and $\forall~ng \in \Delta: ng \nsubseteq A^{\mathcal{B}}$.
\end{definition}

\begin{definition}[Nogood representation of rules~\cite{lazy-cdnl}]
\label{def:prelims-alpha-nogood-ruleencoding}		
In order to represent a ground body of a rule, new atoms are introduced, denoted $\beta(r,\sigma)$ for a rule $r$ and variable substitution $\sigma$.\\
Given a rule $r$ with head $h(r) = a_0$, positive body $b^{+}(r) = a_1,\ldots,a_k$ and negative body $b^{-}(r) = a_{k+1},\ldots,a_n$, the \emph{nogood representation} of $r$, $ng(r)$ is defined as follows:
\begin{align*}
	ng(r) = \{&\{F\beta(r,\sigma),Ta_1,\ldots,Ta_k,Fa_{k+1},\ldots,Fa_n\}_1,\{Fa_0,T\beta(r,\sigma)\},\\
			  &\{T\beta(r,\sigma),Fa_1\},\ldots,\{T\beta(r,\sigma),Fa_k\},\\
			  &\{T\beta(r,\sigma),Ta_{k+1}\},\ldots,\{T\beta(r,\sigma),Ta_n\}\}
\end{align*}	
\end{definition}


\paragraph{Lazy Grounding and Solving Procedure~\cite{alpha-techniques}}
In order to find answer sets in the lazy-grounding search procedure used by Alpha, a grounding component (grounder) always calculates ground substitutions for all rules that are \emph{applicable} with respect to a (partial) assignment $A$. A rule $r$ is said to be applicable w.r.t $A$, if, given a ground substitution $\sigma$ $b^{+}(r\sigma) \subseteq A^{\mathcal{B}}$ and $b^{-}(r\sigma) \cap \{a~|~Ta \in A^{\mathcal{B}}\} = \emptyset$, i.e. a ground rule is applicable, if its positive body is satisifed and its negative body is not contradicted by an assignment.
When solving a program, the grounder constructs an initial partial assignment from all facts and the model of the previously solved \gls{cbp}. Based on this partial assignment, ground instances for applicable rules are computed and translated into nogoods. The partial assignment and generated nogoods are then passed to a solving component (solver). The solver then first applies propagation, i.e. extends assignment $A$ based on unit nogoods as described in Definition~\ref{def:prelims-alpha-nogood-unitness}. Then, if a conflict between the (possibly extended) assignment and the current set of nogoods exists, the conflict is analyzed, a new nogood describing the conflict cause is added, and the solver \emph{backtracks} (i.e. retracts enough of the assignment to get to a conflict-free state). If there are no conflicts, and applicable rules exist, the solver guesses (based on a heuristic), which rule to apply. If there are no applicable rules or unassigned atoms, and no atom is assigned \emph{must-be-true}, the solver has found an answer set. In this case, the answer set is stored, a new nogood which prevents finding the same answer set again is added, and the solver backtracks. Algorithm~\ref{alg:prelims-alpha-search} gives a highlevel summary of the answer set search procedure.

\begin{definition}[Weak and strongly unit nogoods~\cite{lazy-cdnl}]
\label{def:prelims-alpha-nogood-unitness}	
Let $ng$ be a nogood $ng = \{sl_1,\ldots,s_n\}$ and $A$ an assignment. Then
\begin{itemize}
	\item $ng$ is \emph{weakly unit} under $A$ for $sl$ if $ng \setminus A^{\mathcal{B}} = \{sl\}$ and $\overline{sl} \notin A^{\mathcal{B}}$
	\item $ng$ is \emph{strongly unit} under $A$ for $sl$ if $sl$ is designated the "head" of $ng$, $ng \setminus A = \{sl\}$ and $\overline{sl} \notin A$
\end{itemize}	
Intuitively, if a nogood is unit under an assignment $A$, it forces a truth value for the single literal $sl$ which is not yet assigned (as otherwise, the nogood would be violated). Here, the notions of weak and strong unitness are used to distinguish between nogoods forcing a literal to be set to \emph{must-be-true} (if implied by a weakly unit nogood) and \emph{true} (if implied by a strongly unit nogood)
\end{definition}	

\begin{algorithm}[!h]
\SetAlgoLined
\SetKwInOut{Input}{Input}\SetKwInOut{Output}{Output}
\SetKwRepeat{Do}{do}{while}
\Input{A non-ground program $P$}
\Output{The answer sets $AS(P)$ of $P$}
Iniitialize $AS = \emptyset$, assignment $A$, nogood storage $\Delta$.\\
Run lazy grounder, obtain initial nogoods $\Delta$ from facts.\\
\While{search space not exhausted}{
	Propagate on $\Delta$, extending $A$. \\
	\uIf{conflicting nogood exists}{
		Analyze conflict, learn new nogood, backtrack. \\
	}
	\uElseIf{propagation extended $A$}{
		Run lazy grounder w.r.t $A$ to obtain new nogoods to extend $\Delta$.\\
	}
	\uElseIf{applicable rule exists}{
		Guess rule to fire according to heuristic.\\
	}
	\uElseIf{unassigned atom exists}{
		Assign all unassigned atoms to false.\\
	}
	\uElseIf{no atom assigned must-be-true in $A$}{
		$AS \leftarrow AS \cup \{A\}$
		Answer set found, add enumeration nogood and backtrack. 
	}
	\Else{
		Backtrack.
	}
}
\textbf{return} $AS$
\caption{An answer set search procedure for lazy-grounding solvers~\cite{alpha-techniques}.}\label{alg:prelims-alpha-search}
\end{algorithm}


\chapter{The Evolog Language}
The Evolog language extends (non-disjunctive) ASP as defined in the ASP-Core2 standard~\cite{asp-core2} with facilities to communicate with and influence the "outside world" (e.g. read and write files, capture user input, etc.) as well as program modularization and reusability features, namely \emph{actions} and \emph{modules}.

\section{Actions in Evolog}
\label{sec:evolog-actions}

Actions allow for an ASP program to encode operations with \emph{side-effects} while maintaining fully declarative semantics. Actions are modelled in a functional style loosely based on the concept of monads as used in Haskell~\todo{cite something here!}. Intuitively, to maintain declarative semantics, actions need to behave as pure functions, meaning the result of executing an action (i.e. evaluating the respective function) must be reproducible for each input value across all executions. On first glance, this seems to contradict the nature of IO operations, which inherently depend on some state, e.g. the result of evaluating a function $getFileHandle(f)$ for a file $f$ will be different depending on whether $f$ exists, is readable, etc. However, at any given point in time - in other words, in a given state of the world - the operation will have exactly one result (i.e. a file handle or an error will be returned). A possible solution to making state-dependent operations behave as functions is therefore to make the state of the world at the time of evaluation part of the function's input. A function $f(x)$ is then turned into $f'(s, x)$  where $s$ represents a specific world state. The rest of this section deals with formalizing this notion of actions.

\todo{Define non-disjunctive ASP-Core2 in detail in preliminaries. Give detailed definition of all "standard ASP" elements referenced here!}

\subsection{Syntax}
\label{subsec:evolog-actions-syntax}

\begin{definition}[Action Rule, Action Program]
\label{def:action-rule-syntax}
An \emph{action rule} $R$ is of form
\[
	a_H : @t_{act} = act_{res} \leftarrow l_1,\ldots,l_n.
\]
where
\begin{itemize}
	\item $a_H$ is an atom called \emph{head atom},
	\item $t_{act}$ is a functional term called \emph{action term},
	\item $act_{res}$ is a term called \emph{(action-)result} term
	\item and $l_1,\ldots,l_n$ are literals constituting the \emph{body} of $R$.
\end{itemize}
An \emph{action program} $P$ is a set of (classic ASP-)rules and action rules.
\end{definition}

\subsection{Semantics}
\label{subsec:evolog-actions-semantics}

To properly define the semantics of an action program according to the intuition outlined at the start of this section, we first need to formalize our view of the "outside world" which action rules interact with. We call the world in which we execute a program a \emph{frame} - formally, action programs are always evaluated \emph{with respect to a given frame}. The behavior of actions is specified in terms of \emph{action functions}. The semantics (i.e. interpretations) of action functions in a program are defined by the respective frame.

\subsubsection{Action Rule Expansion}
\label{subsubsec:evolog-actions-semantics-expansion}

To get from the practical-minded action syntax from Definition~\ref{def:action-rule-syntax} to the formal representation of an action as a function of some state and an input, we use the helper construct of an action rule's \emph{expansion} to bridge the gap. Intuitively, the expansion of an action rule is a syntactic transformation that results in a more verbose version of the original rule called \emph{application rule} and a second rule only dependent on the application rule called \emph{projection rule}. A (ground) application rule's head atom uniquely identifies the ground instance of the rule that derived it. As one such atom corresponds to one action executed, we call a ground instance of an application rule head in an answer set an \emph{action witness}. 

\todo{define (classic ASP) grounding and substitutions in preliminaries}

\begin{definition}[Action Rule Expansion]
\label{def:action-rule-expansion}
Given a non-ground action rule $R$ with head atom $a_H$, action term $f_{act}(i_1,\ldots,i_n)$ and body  $B$ consisting of literals $l_1,\ldots,l_m$, the expansion of $R$ is a pair of rules consisting of an \emph{application rule} $R_{app}$ and \emph{projection rule} $R_{proj}$. $R_{app}$ is defined as
\[
	a_{res}(f_{act}, S, I, f_{act}(S, I)) \leftarrow l_1,\ldots,l_n.
\]
where $S$ and $I$ and function terms called \emph{state-} and \emph{input-}terms, respectively.
An action rule's state term has the function symbol $\mathit{state}$ and terms $fn(l_1),\ldots,fn(l_m)$, with the expression $fn(l)$ for a literal $l$ denoting a function term representing $l$. The (function-)term representation of a literal $p(t_1,\ldots,t_n)$ with predicate symbol $p$ and terms $t_1,\ldots,t_n$ uses $p$ as function symbol. For a negated literal $\mathit{not}~p(t_1,\ldots,t_n)$, the representing function term is $not(p(t_1,\ldots,p_n))$. The action input term is a "wrapped" version of all arguments of the action term, i.e. for action term $f_{act}(t_1,\ldots,t_n)$, the corresponding input term is $input(t_1,\ldots,t_n)$. The term $f_{act}(S, I)$ is called \emph{action application term}. \\
The projection rule $R_{proj}$ is defined as
\[
	a_H \leftarrow a_{res}(f_{act}, S, I, v_{res}).
\]
where $a_H$ is the head atom of the initial action rule $R$ and the (sole) body atom is the action witness derived by $R_{app}$, with the application term $f_{act}(S, I)$ replaced by a variable $v_{res}$ called \emph{action result variable}.
\end{definition}

Looking at the head of an action application rule of format $a_{res}(f_{act}, S, I, \mathit{t_{app}})$ with action $f_{act}$, state term $S$, input term $I$ and application term $t_{app}$, the intuitive reading of this atom is "The result of action function $f_{act}$ applied to state $S$ and input $I$ is $t_{app}$", i.e. the action application term $t_{app}$ is not a regular (uninterpreted) function term as in regular ASP, but an actual function call which is resolved using an interpretation function provided by a \emph{frame} during grounding.

%% NOTE: Action func. term is uninterpreted in nonground view! Application formally happens during grounding, i.e. action func is applied in grounding, ground version has action result.

\subsubsection{Grounding of Action Rules}
\label{subsubsec:evolog-actions-semantics-grounding}

Grounding, in the context of answer set programming, generally refers to the conversion of a program with variables into a semantically equivalent, variable-free, version. Action application terms as introduced in Definition \ref{def:action-rule-expansion} can be intuitively read as variables, in the sense that they represent the result of applying the respective action function. Consequently, all action application terms are replaced with the respective (ground) result terms defined in the \emph{frame} with respect to which the program is grounded.

\begin{definition}[Frame]
\label{def:evolog-frame}
Given an action program $P$ containing action application terms $A = \{a_1,\ldots,a_n\}$, a frame $F$ is an interpretation function such that, for each application term $f_{act}(S, I) \in A$ where $S \in H_{U}(P)^{*}$ and $I \in H_{U}(P)^{*}$, $F(f_{act}): H_{U}(P)^{*} \times H_{U}(P)^{*} \mapsto H_{U}(P)$.
\end{definition}

\begin{definition}[Grounding of action rules]
\label{def:evolog-grounding}
Grounding of Evolog rules (and programs) always happens \emph{with respect to a frame}. Given a frame $F$, an expanded action rule $r_a$ and a (grounding) substitution $\sigma$ over all body variables of application rule $r_{a_{app}}$, during grounding, every ground action application term $t_{app}\sigma$ resulting from applying substitution $\sigma$ is replaced with its interpretation according to $F$.
\end{definition}

Example \ref{ex:action-rule-expansion} demonstrates the expansion of an action rule as well as a compatible example frame for the respective action.

\begin{example}[Expansion and Frame]
\label{ex:action-rule-expansion}
Consider following Evolog Program $P$ which contains an action rule with action $a$:
\begin{align*}
	&p(a).~q(b).~r(c). \\
	&h(X, R) : @a(X, Z) = R \leftarrow p(X), q(Y), r(Z).
\end{align*}
The expansion of $R$ is:
\begin{align*}
	a_{res}(a, \mathit{state}(p(X), q(Y), r(Z)), \mathit{input}(X, Z), a(\mathit{state}(p(X), q(Y), r(Z)), \mathit{input}(X, Z))) \leftarrow& \\
	p(X), q(Y), r(Z).& \\
	h(X, R) \leftarrow a_{res}(a, \mathit{state}(p(X), q(Y), r(Z)), \mathit{input}(X, Z), R).&
\end{align*}
Furthermore, consider following frame $F$:
\[
	F(a) = \{a(\mathit{state}(p(a), q(b), r(c)), \mathit{input}(a, c)) \mapsto \mathit{success}(a, c)\}
\]
which assigns the result $\mathit{success}(a, c)$ to the action application term (i.e. function call $a(\mathit{state}(p(a), q(b), r(c)), \mathit{input}(a, c)))$. \\

Then, the ground program $P_{grnd}$ after action rule expansion is
\begin{align*}
	p(a).~q(b).~r(c).& \\
	a_{res}(a, \mathit{state}(p(a), q(b), r(c)), \mathit{input}(a, c), \mathit{success}(a, c)) \leftarrow p(a).~q(b).~r(c).& \\
	h(a, \mathit{success}(a, c)) \leftarrow a_{res}(a, \mathit{state}(p(a), q(b), r(c)), \mathit{input}(a, c), \mathit{success}(a, c)).&
\end{align*}
The sole model of $P$ with respect to frame $F$ is 
\begin{align*}
	M = \{p(a), q(b), r(c),& \\
	a_{res}(a, \mathit{state}(p(a), q(b), r(c)), \mathit{input}(a, c), \mathit{success}(a, c))& \\ h(a, \mathit{success}(a, c))\}&
\end{align*}
\end{example}

\subsubsection{Evolog Models}
\label{subsubsec:evolog-action-semantics-models}

Having introduced action rule expansions as well as frames, we now use these to extend the stable model semantics to Evolog programs.

\begin{definition}[Supportedness of Actions]
\label{def:evolog-supported-action}
Let $r_{app}$ be a non-ground action application rule with head $H = a_{res}(f_{act}, S, I, f_{act}(S, I))$, $F$ a frame, and $H_{grnd} = a_{res}(f_{act}, S_{grnd}, I_{grnd}, r)$ a ground instance of $H$ with $r$ being an arbitrary ground term.
Then, $H_{grnd}$ is \emph{supported by $F$}, if and only if $F$ contains a mapping of form $f_{act}(S_{grnd}, I_{grnd}) \mapsto r$, i.e. $r$ is a valid result of action function $f_{act}$ with arguments $S_{grnd}$ and $I_{grnd}$ according to frame $F$. 
We call a ground instance of an action rule \emph{supported by a frame} if the head of the corresponding application rule in the rule's expansion is supported by that frame.
\end{definition}

\begin{definition}[Evolog-Reduct]
\label{def:evolog-reduct}
Given a ground Evolog program $P$, a frame $F$ and a set of ground atoms $A$, the \emph{Evolog Reduct} of $P$ with respect to $F$ and $A$ $P_{F}^{A}$ is obtained from $P$ as follows:
\begin{enumerate}
	\item Remove all rules $r$ from $P$ that are "blocked", i.e. $A \not\models l$ for some negative body literal $l \in b^{-}(r)$.
	\item Remove all action application rules from $P$ which are not supported by $F$.
	\item Remove the negative body from all other rules.
\end{enumerate}
\end{definition}

Note that the reduct outlined in Definition \ref{def:evolog-reduct} extends the classic GL-reduct (see Definition \ref{def:prelims-asp-semantics-gl-reduct}) just by adding a check on action supportedness.

\begin{definition}[Evolog Model]
A herbrand interpretation $I$ of an Evolog Program is an \emph{Evolog Model} ("answer set") of an Evolog program $P$ with respect to a frame $F$ if and only if it is a minimal classical model of its Evolog-Reduct $P_{F}^{A}$. We denote the set of Evolog Models of a program $P$ as $\mathit{EM}(P)$.
\end{definition}

\subsection{Restrictions on Program Structure}
\label{subsec:evolog-actions-restrictions}

% Following goals are listed in introduction
%     \item declarative programs, i.e. order in which actions occur in code does not affect semantics,
%     \item actions behaving in a functional fashion, i.e. an action always gives the same result for the
%      same input. Especially, actions have to be idempotent in the sense that, for an ASP rule that is
%      associated with some action, the result of the action never changes, no matter how often the rule
%      fires.
%     \item transparent action execution, i.e. every action that is executed during evaluation of a program must be reflected in an answer set.

While the action semantics outlined so far addresses the requirements for both declarativity and functionality of actions outlined in the introduction (see \ref{goals:actions}), we haven't yet addressed the demand for transparency, i.e. that every action that is executed must be reflected in an aswer set of the respective program. With just the semantics outlined in Section \ref{subsec:evolog-actions-semantics}, it would be possible to write programs such as the one shown in Example \ref{ex:unsat-with-sideeffects} where an action rule can fire, but the program is unsatisfiable.

\begin{example}[Unsatisfiable Program with Side-effects]
\label{ex:unsat-with-sideeffects}
The program below contains an action rule that can fire (because $p(a)$ is true), but is also unsatisfiable due to the constraint in the last line.
\begin{align*}
	&p(a). \\
	&q(X) \leftarrow p(X). \\
	&act\_done(X, R) : @act(X) = R \leftarrow p(X). \\
	&\leftarrow q(X), act\_done(X, \_).
\end{align*}
\end{example}

This kind of programs raises some hard problems for implementations - given the contract that every side-effect of (i.e. action executed by) a program must be reported in an answer set, a solver evaluating the program from Example~\ref{ex:unsat-with-sideeffects} would have to "retract" action $act/1$ after finding that the program is unsatisfiable. Since it is generally not possible to "take back" side-effects (e.g. when some message is sent over a network broadcast to an unknown set of recipients), the only practical way to deal with this is to impose some conditions on programs with actions. Definition~\ref{def:evolog-actions-transparency} details this notion and introduces \emph{transparency} as a necessary condition for an Evolog program to be considered valid.

\begin{definition}[Action Transparency]
\label{def:evolog-actions-transparency}
An action rule $r_a$ of an Evolog program $P$ is \emph{transparent} if, for every expanded ground instance $gr_a$ it holds that, if $gr_a$ fires, then the head $h(gr_{a_{app}})$ of the repective application rule $gr_{a_{app}}$ is contained in an answer set of $P$. \\

For an Evolog program to be valid, all its action rules must be transparent.
\end{definition}

It follows from Definition \ref{def:evolog-actions-transparency} that only satisfiable programs can be transparent. Furthermore, note that Definition~\ref{def:evolog-actions-transparency} aims to be as permissive as possible in terms of program structure. In general, it can not be assumed that transparency of all rules in a program can be guaranteed up-front. Implementations may therefore impose further restrictions on what is considered a valid Evolog program. \todo{It might make sense to introduce the restriction P = CBP(P) for a valid program already here (anything more permissive cannot be linted without a theorem prover!)}


\section{Program Modularization in Evolog}
\label{sec:evolog-modules}

TODO

\section{Relationship between Evolog- and Stable Model Semantics}

The extensions to the usual \gls{asp} programming language described in the previous sections extend the original formalism with a notion of an outside world (through frames in the context of which, actions can be expressed) as well as a grouping mechanism for sub-programs into modules.

\begin{theorem}[Extension]
\label{thm:extension}
Every ASP program $P$ is a valid and semantically equivalent Evolog program in the sense that - for any given frame $F$, the Evolog Models of $P$ are the same as its Answer Sets according to Stable Model Semantics~\ref{def:prelims-asp-semantics-answer-set}.
\end{theorem}

\begin{proof}
First, we assert that every "regular" ASP program P is also a syntactically valid Evolog program. This follows directly from the definitions - since Evolog only adds syntactic support for actions~\ref{def:action-rule-syntax}, but does not restrict regular ASP syntax, it follows that a syntactically correct ASP program is also a syntactically correct Evolog program. \\
Next, we show that for every regular ASP program $P$ and any frame $F$, the Evolog Models of $P$ are the same as its Stable Models, i.e. $\mathit{EM}(P)=\mathit{AS}(P)$: \\
Given any set of ground Atoms $A$ from the Herbrand Base $HB_P$ of $P$, $P$ is an Answer Set according to Stable Model Semantics if it is a minimal model of the GL-reduct~\ref{def:prelims-asp-semantics-gl-reduct} $P^{A}$ of $P$ w.r.t. $A$. We recall that $P^{A}$ is constructed as follows (see~\ref{def:prelims-asp-semantics-gl-reduct}):
\begin{itemize}
	\item remove from $P$ all rules $r$ that are "blocked", i.e. $A \not\models l$ for some literal $l \in b^{-}(r)$ 
	\item and remove the negative body of all other rules.
\end{itemize}
Furthermore, consider how the Evolog-Reduct of $P$ w.r.t. $A$ and (any arbitrary) Frame $F$, $P_{F}^{A}$, is constructed (see~\ref{def:evolog-reduct}):
\begin{itemize}
	\item Remove all rules $r$ from $P$ that are "blocked", i.e. $A \not\models l$ for some negative body literal $l \in b^{-}(r)$.
	\item Remove all action application rules from $P$ which are not supported by $F$.
	\item Remove the negative body from all other rules.
\end{itemize}
Since $P$ is a "regular", i.e. non-Evolog, ASP Program it contains no action application rules by definition. It is therefore clear, that for any set answer set candidate $A$ of $P$ and any Frame $F$, the GL-Reduct $P^{A}$ and Evolog-Reduct $P_{F}^{A}$ coincide. Any minimal model of the GL-reduct is therefore also a minimal model of the Evolog-Reduct (and vice-versa) and therefore the Evolog Models of P are identical with its Satble Models, i.e. $\mathit{EM}(P)=\mathit{AS}(P)$.
\end{proof}

\backmatter

% Use an optional list of figures.
\listoffigures % Starred version, i.e., \listoffigures*, removes the toc entry.

% Use an optional list of tables.
\cleardoublepage % Start list of tables on the next empty right hand page.
\listoftables % Starred version, i.e., \listoftables*, removes the toc entry.

% Use an optional list of alogrithms.
\listofalgorithms
\addcontentsline{toc}{chapter}{List of Algorithms}

% Add an index.
\printindex

% Add a glossary.
\printglossaries

% Add a bibliography.
\bibliographystyle{alpha}
\bibliography{evolog}

\end{document}