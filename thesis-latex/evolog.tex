% Copyright (C) 2014-2020 by Thomas Auzinger <thomas@auzinger.name>

\documentclass[draft,final]{vutinfth} % Remove option 'final' to obtain debug information.

% Load packages to allow in- and output of non-ASCII characters.
\usepackage{lmodern}        % Use an extension of the original Computer Modern font to minimize the use of bitmapped letters.
\usepackage[T1]{fontenc}    % Determines font encoding of the output. Font packages have to be included before this line.
\usepackage[utf8]{inputenc} % Determines encoding of the input. All input files have to use UTF8 encoding.

% Extended LaTeX functionality is enables by including packages with \usepackage{...}.
\usepackage{amsmath}    % Extended typesetting of mathematical expression.
\usepackage{amsthm}
\usepackage{listings}
\usepackage{amssymb}    % Provides a multitude of mathematical symbols.
\usepackage{mathtools}  % Further extensions of mathematical typesetting.
\usepackage{microtype}  % Small-scale typographic enhancements.
\usepackage[inline]{enumitem} % User control over the layout of lists (itemize, enumerate, description).
\usepackage{multirow}   % Allows table elements to span several rows.
\usepackage{booktabs}   % Improves the typesettings of tables.
\usepackage{subcaption} % Allows the use of subfigures and enables their referencing.
\usepackage[ruled,linesnumbered,algochapter]{algorithm2e} % Enables the writing of pseudo code.
\usepackage[usenames,dvipsnames,table]{xcolor} % Allows the definition and use of colors. This package has to be included before tikz.
\usepackage{nag}       % Issues warnings when best practices in writing LaTeX documents are violated.
\usepackage{todonotes} % Provides tooltip-like todo notes.
\usepackage{color}
\usepackage{float}
\usepackage{boldline}
\usepackage{hyperref}  % Enables cross linking in the electronic document version. This package has to be included second to last.
\usepackage[acronym,toc]{glossaries} % Enables the generation of glossaries and lists fo acronyms. This package has to be included last.

% Define convenience functions to use the author name and the thesis title in the PDF document properties.
\newcommand{\authorname}{Michael Langowski} % The author name without titles.
\newcommand{\thesistitle}{Evolog - Actions and Modularization in Lazy-Grounding Answer Set Programming} % The title of the thesis. The English version should be used, if it exists.

% Set PDF document properties
\hypersetup{
    pdfpagelayout   = TwoPageRight,           % How the document is shown in PDF viewers (optional).
    linkbordercolor = {Melon},                % The color of the borders of boxes around crosslinks (optional).
    pdfauthor       = {\authorname},          % The author's name in the document properties (optional).
    pdftitle        = {\thesistitle},         % The document's title in the document properties (optional).
    pdfsubject      = {Subject},              % The document's subject in the document properties (optional).
    pdfkeywords     = {a, list, of, keywords} % The document's keywords in the document properties (optional).
}

\setpnumwidth{2.5em}        % Avoid overfull hboxes in the table of contents (see memoir manual).
\setsecnumdepth{subsection} % Enumerate subsections.

\nonzeroparskip             % Create space between paragraphs (optional).
\setlength{\parindent}{0pt} % Remove paragraph identation (optional).

\theoremstyle{definition}
\newtheorem{definition}{Definition}[section]

\newtheorem{theorem}{Theorem}[section]

\newtheorem{example}{Example}[section]

\newtheorem{corollary}{Corollary}[section]

\definecolor{pblue}{rgb}{0.13,0.13,1}
\definecolor{pgreen}{rgb}{0,0.5,0}
\definecolor{pred}{rgb}{0.9,0,0}
\definecolor{pgrey}{rgb}{0.46,0.45,0.48}
\definecolor{ppurple}{rgb}{0.49,0.18,0.52}

\lstdefinestyle{code}{
	basicstyle=\ttfamily
}

\lstdefinestyle{asp-code}{
	basicstyle=\ttfamily,
	frame=single,
	numbers=left,
  stepnumber=1,
  breaklines=true,
  showstringspaces=false,
  breakatwhitespace=true,  
}

\lstdefinestyle{java}{language=Java,
  showspaces=false,
  showtabs=false,
  breaklines=true,
  showstringspaces=false,
  breakatwhitespace=true,
  commentstyle=\color{pgreen},
  keywordstyle=\color{pblue},
  stringstyle=\color{pred},
  basicstyle=\ttfamily,
  moredelim=[il][\textcolor{ppurple}]{$$},
  moredelim=[is][\textcolor{ppurple}]{\%\%}{\%\%}
}

% Set persons with 4 arguments:
%  {title before name}{name}{title after name}{gender}
%  where both titles are optional (i.e. can be given as empty brackets {}).
\setauthor{}{\authorname}{BSc.}{male}
\setadvisor{Prof. Dr.}{Thomas Eiter}{}{male}

% For bachelor and master theses:
\setfirstassistant{Dr.}{Antonius Weinzierl}{}{male}
%\setsecondassistant{Pretitle}{Forename Surname}{Posttitle}{male}
%\setthirdassistant{Pretitle}{Forename Surname}{Posttitle}{male}

% For dissertations:
%\setfirstreviewer{Pretitle}{Forename Surname}{Posttitle}{male}
%\setsecondreviewer{Pretitle}{Forename Surname}{Posttitle}{male}

% For dissertations at the PhD School and optionally for dissertations:
%\setsecondadvisor{Pretitle}{Forename Surname}{Posttitle}{male} % Comment to remove.

% Required data.
\setregnumber{01426581}
\setdate{01}{07}{2022} % Set date with 3 arguments: {day}{month}{year}.
\settitle{\thesistitle}{\thesistitle} % Sets English and German version of the title (both can be English or German). If your title contains commas, enclose it with additional curvy brackets (i.e., {{your title}}) or define it as a macro as done with \thesistitle.
\setsubtitle{}{} % Sets English and German version of the subtitle (both can be English or German).

% Select the thesis type: bachelor / master / doctor / phd-school.
% Bachelor:
%\setthesis{bachelor}
%
% Master:
\setthesis{master}
\setmasterdegree{dipl.} % dipl. / rer.nat. / rer.soc.oec. / master
%
% Doctor:
%\setthesis{doctor}
%\setdoctordegree{rer.soc.oec.}% rer.nat. / techn. / rer.soc.oec.
%
% Doctor at the PhD School
%\setthesis{phd-school} % Deactivate non-English title pages (see below)

% For bachelor and master:
\setcurriculum{Logic and Computation}{Logic and Computation} % Sets the English and German name of the curriculum.

\newcommand{\IDs}{\mathit{ID}}
\newcommand{\INTs}{\mathit{INT}}
\newcommand{\VARs}{\mathit{VAR}}

\newcommand{\NOT}{\mathit{not}}

\newcommand{\DEP}[1]{\succ_{d}^{#1}}

\newacronym{asp}{ASP}{Answer Set Programming}
\newacronym{cdnl}{CDNL}{Conflict-driven Nogood Learning}
\newacronym{cbp}{CBP}{Common Base Program}
\newacronym{dtd}{DTD}{Document Type Definition}
\newacronym{dom}{DOM}{Document Object Model}

\makeindex      % Use an optional index.
\makeglossaries % Use an optional glossary.
%\glstocfalse   % Remove the glossaries from the table of contents.

\begin{document}

\frontmatter % Switches to roman numbering.
% The structure of the thesis has to conform to the guidelines at
%  https://informatics.tuwien.ac.at/study-services

\addtitlepage{naustrian} % German title page (not for dissertations at the PhD School).
\addtitlepage{english} % English title page.
\addstatementpage

\begin{danksagung*}
\todo{Ihr Text hier.}
\end{danksagung*}

\begin{acknowledgements*}
\todo{Enter your text here.}
\end{acknowledgements*}

\begin{kurzfassung}
  \emph{Answer Set Programming (ASP)} ist ein seit Jahrzehnten etablierter Formalismus in der Logikprogrammierung, der insbesondere bei der Modellierung und Lösung NP-schwerer Planungsprobleme Anwendung findet. Effiziente Solver für ASP sind seit über 20 Jahren verfügbar. Aktuelle Trends in ASP befassen sich zunehmend mit der Interaktion mit externen Daten, sowohl zum Lesen und Verarbeiten (wie im Stream Reasoning~\cite{lars} bis hin zu Interaktion mit Machine-Learning-Software~\cite{neurasp}), als auch zum Ausführen von Aktionen, d.h. der Entwicklung von ASP-Programmen mit Seiteneffekten (side-effects)~\cite{acthex, aspetris}. Das primäre Ziel der vorliegenden Arbeit ist es, eine neue Methode zur Modellierung von Regeln mit Seiteneffekten in ASP Programmen zu entwickeln, die sich leicht in einem \emph{lazy-grounding} ASP-Solver wie Alpha~\cite{alpha} implementieren lässt. Ein weiteres Ziel ist die Entwicklung eines leichtgewichtigen Konzeptes zur Programmmodularisierung, das die zuvor erwähnte Unterstützung von Aktionen ergänzt. Die Notwendigkeit hierfür begründet sich aus der Tatsache, das Modellieren von Seiteneffekten in einer vollständig deklarativen Sprache zwangsläufig einige Einschränkungen struktureller Natur für Programme mit Aktionen mit sich bringt. Die mit dem Modularisierungskonzept gegebene Möglichkeit, seiteneffekt-freie Unterprogramme als eigenständige Einheiten auszuführen, sollte diese Einschränkungen im Praxiseinsatz ausgleichen. Die resultierende erweiterte Answer-Set-Programming-Sprache wird \emph{Evolog} genannt.
  Die formale Semantik von Aktionen wird bezüglich eines \emph{Frame}, d.h. eines Modells der Seiteneffekte einer Aktion in der Außenwelt in Form einer Interpretationsfunktion für \emph{Action-Atome}, modelliert. Basierend auf diesem Modell der Außenwelt werden Kriterien definiert, nach denen ein Evolog-Programm (semantisch) gültig ist. Schließlich wird das Konzept eines \emph{Evolog-Modells} definiert, d.h. eines Stable-Models eines Evolog-Programms, das konsistent bezüglich einem gegebenen Frame ist. Darüber hinaus wird gezeigt, dass Evolog eine \emph{konservative} Erweiterung der üblichen Stable-Model-Semantik ist, insofern als jedes reguläre ASP-Programm auch ein gültiges Evolog-Programm in allen Frames ist.
  Die Modularisierung wird basierend auf Alpha's bestehendem Konzept eines \emph{externen Atoms} modelliert, d.h. eines Atoms, das eine seiteneffektfreie Funktion abbildet, die in einer beliebigen Programmiersprache (in diesem Fall Java) implementiert ist. Modul-Atome sind definiert als externe Atome, deren Interpretationsfunktion darin besteht, einen ASP-Solver mit einem fixen Programm (der Modul-Implementierung) und den Eingabetermen des externen Atoms als zusätzlichem Input aufzurufen. Um das Schreiben von Code, der Module benutzt, effizienter zu gestalten, wird das (rein syntaktische) Konzept eines \emph{Listen-Terms} eingeführt.
  Um Evolog als tatsächliche Programmiersprache zu evaluieren, wird eine einfache Kommandozeilenapplikation vorgestellt, die Graphen aus XML-Dateien einliest, 3-Colorings dieser Graphen berechnet und die berechneten Colorings in XML-Dateien serialisiert. Die Applikation wird sowohl in Bezug auf die Entwicklungsfreundlichkeit als auch auf die Ausführungsleistung diskutiert. Basierend auf der Evaluierung der XML-3-Coloring-App werden mögliche Erweiterungen der Sprache, sowie zukünftige Wege zur Steigerung der Solver-Performance formuliert.
\end{kurzfassung}

\begin{abstract}
\emph{Answer Set Programming (ASP)} has been an established formalism in logic programming, especially used for NP-hard planning problems, for decades, with efficient solvers available for more than 20 years. Current trends in ASP increasingly deal with interaction with external data, both for reading and processing (such as in stream reasoning~\cite{lars} or even interacting with machine learning software~\cite{neurasp}) as well as performing actions, i.e. influencing the state of the outside world~\cite{acthex, aspetris}. This thesis' primary concern is exploring a new way of representing actions in a manner suitable to be easily implemented in a \emph{lazy-grounding} ASP solver such as Alpha~\cite{alpha}. Secondary to that, the other goal of this work is to specify and implement a light-weight program modularization concept to go along with the aforementioned action support. The inspiration here is that, since inevitably modelling side-effects in a fully declarative language forces us to impose some restrictions on programs with actions, being able to run action-free subprograms as encapsulated units should alleviate any such strictures in software development. The resulting extended answer set programming language is called \emph{Evolog}.
We define the formal semantics of actions with regards to a \emph{frame}, i.e. a formalization of the effects of an action in the outside world, that is modelled as an interpretation function for \emph{action atoms}. Based on this model of the outside world, the notion of what constitutes a (semantically) valid Evolog program is defined. Finally, we introduce a definition of an \emph{Evolog Model}, i.e. a stable model of an Evolog program that is consistent with a given frame. We further show that Evolog is a \emph{conservative} extension to traditional ASP in the sense that every regular ASP program is also a valid Evolog program in all frames.
Modularization is modelled around Alpha's existing concept of an \emph{external atom}, i.e. an atom that models a side-effect free function which is implemented in an arbitrary programming language (Java in this case). We define module atoms as external atoms whose interpretation function is given as calling an ASP solver with a fixed program (the module body) and the input terms of the external atom as additional input. To aid in efficiently writing code around modules, the purely syntactical concept of a \emph{list term} is introduced.
In order to evaluate Evolog as an actual application programming language, a simple command-line application for reading graphs from XML files, calculating 3-colorings of said graphs and serializing calculated colorings to XML files, is showcased and discussed in terms of both ease of development, as well as evaluation performance. Based on our evaluation of the XML-3-coloring application, we formulate possible refinements to the language as well as future paths for improvement in solver performance.
\end{abstract}

% Select the language of the thesis, e.g., english or naustrian.
\selectlanguage{english}

% Add a table of contents (toc).
\tableofcontents % Starred version, i.e., \tableofcontents*, removes the self-entry.

% Switch to arabic numbering and start the enumeration of chapters in the table of content.
\mainmatter

\chapter{Introduction}
\todo{short summary what ASP is, why it is cool etc.}

The goal of this work is to address the following issues with as little "semantics overhead" as possible and deliver an easy-to-use programming language that can be used in most software engineering contexts.

\paragraph{Triggering actions from programs} Most program flows follow a chain of events, each a consequence of its predecessor, e.g. "If there exists a file A, read it. If reading was successful, do something with the content. If the operation succeeds, write the result to file B". It is highly desirable to be able to write this kind of program in a declarative, logic-based language that can leverage the strengths of ASP for the "business logic" part. Specifically, the proposed action semantics should deliver
\begin{itemize}
    \item declarative programs, i.e. order in which actions occur in code does not affect semantics,
    \item actions behaving in a functional fashion, i.e. an action always gives the same result for the same input. Especially, actions have to be idempotent in the sense that, for an ASP rule that is associated with some action, the result of the action never changes, no matter how often the rule fires.
    \item transparent action execution, i.e. every action that is executed during evaluation of a program must be reflected in an answer set.
\end{itemize}

\paragraph{Program Modularization} While not formally connected, triggering actions from programs and modularization (i.e. plugable and re-usable sub-programs), intuitively complement each other in our current high-level design. Introducing a simple, easy-to-use module system is therefore the second goal of this work. It is, however, secondary in priority to definition and prototypical implementation of action support and may be reduced to a technical design draft if required due to time constraints.

\paragraph{Incremental Evaluation and Lazy Grounding} Experiences from existing systems for ASP application development such as ASAP~\cite{aspetris} or ACTHEX~\cite{acthex} show that, in order to achieve the evaluation performance necessary for use in real-world applications, ASP application code needs to be evaluated in an incremental fashion (rather than iteratively re-evaluating the whole program) whenever possible. The lazy-grounding architecture employed by ASP systems such as Alpha~\cite{alpha} offers an intuitive solution. % A short introduction to LaTeX.

\chapter{Preliminaries}
\section{Answer Set Programming}

When speaking of \emph{answer set programming}, we nowadays mostly refer to the language specfied by the ASP-Core2 standard~\cite{asp-core2}. It uses the \emph{stable model semantics} by Gelfond and Lifschitz~\cite{stable-models} as a formal basis and enhances it with support for advanced concepts such as disjunctive programs, aggregate literals and weak constraints. This chapter describes the input language supported by the Alpha solver, which will serve as the basis on which we will define the Evolog language.

\todo{abbreviations!}

\subsection{Syntax}
\label{subsec:prelims-asp-syntax}

\begin{definition}[Integer numeral]
\label{def:prelims-asp-syntax-int}
An \emph{integer numeral} in the context of an ASP program is a string matching the regular expression:
\begin{lstlisting}[style=code]
(-)?[0-9]+
\end{lstlisting}
The set of all valid integer numerals is denoted as $\INTs$.
\end{definition}

\begin{definition}[Identifier]
\label{def:prelims-asp-syntax-id}
An \emph{identifier} in the context of an ASP program is a string matching the regular expression:
\begin{lstlisting}[style=code]
[a-z][a-zA-Z0-9\_]*
\end{lstlisting}
The set of all valid identifiers is denoted as $\IDs$.
\end{definition}

\begin{definition}[Variable Name]
\label{def:prelims-asp-syntax-var}
A \emph{variable name} in the context of an ASP program is a string matching the regular expression:
\begin{lstlisting}[style=code]
[A-Z][a-zA-Z0-9\_]*
\end{lstlisting}
The set of all valid variable names is denoted as $\VARs$.
\end{definition}

\begin{definition}[Term]
\label{def:prelims-asp-syntax-term}
A \emph{term} is inductively defined as follows:
\begin{itemize}
	\item Any \emph{constant} $c \in (\INTs \cup \IDs)$ is a term.
	\item Any \emph{variable} $v \in \VARs$ is a term.
	\item Given terms $t_1, t_2$, any \emph{artihmetic expression} $t_1 \oplus t_2$ with $\oplus \in \{+, - , *, /, **\}$ is a term.
	\item Given terms $t_1, t_2$, any \emph{interval expression} $t_1 \ldots t_2$ is a term.
	\item For function symbol $f \in \IDs$ and argument terms $t_1, \ldots, t_n$, the \emph{functional expression} $f(t_1, \ldots, t_n)$ is a term.
\end{itemize}
\end{definition}

\begin{definition}[Subterms]
\label{def:prelims-asp-syntax-subterms}
Given a term $t$, the set of \emph{subterms} of $t$, $st(t)$, is defined as follows:
\begin{itemize}
	\item If $t$ is a \emph{constant} or \emph{variable}, $st(t) = \{t\}$.
	\item If $t$ is an \emph{arithmetic expression} $t_1 \oplus t_2$, $st(t) = st(t_1) \cup st(t_2)$.
	\item If $t$ is an \emph{interval expression} $t_1 \ldots t_2$, $st(t) = st(t_1) \cup st(t_2)$.
	\item If $t$ is a \emph{functional expression} with argument terms $t_1, \ldots, t_n$, $st(t) = st(t_1) \cup \ldots \cup st(t_n)$.
\end{itemize}
A term is called \emph{ground} if it is variable-free, i.e. none of its subterms is a variable.
\end{definition}

\begin{definition}[Atom]
\label{def:prelims-asp-syntax-atom}
Given a predicate symbol $p \in \IDs$ and argument terms $t_1,\ldots,t_n$, the expression
\[
	p(t_1,\ldots,t_n)
\]
is called an \emph{atom}. An atom is ground if all of its argument terms are ground. A ground atom with predicate $p$ is called an \emph{instance} of $p$.
\end{definition}

\begin{definition}[Literal]
\label{def:prelims-asp-syntax-literal}
A literal in ASP is an atom $a$ or ("default"-)negated atom $\NOT\ a$.
\end{definition}

\begin{definition}[Rule, Program]
\label{def:prelims-asp-syntax-rule}
A \emph{rule} is an expression of form
\[
	a_H \leftarrow b_1,\ldots,b_n.
\]
where the \emph{rule head} $a_H$ is an atom and the \emph{rule body} $b_1,\ldots,b_n$ is a set of literals. An ASP \emph{program} is a set of rules
\end{definition}
\todo{Maybe add definitions for constraints and choice rules?}

\chapter{The Evolog Language}
\label{chap:language}
The Evolog language extends (non-disjunctive) ASP as defined in the ASP-Core2 standard~\cite{asp-core2} with facilities to communicate with and influence the "outside world" (e.g. read and write files, capture user input, etc.) as well as program modularization and reusability features, namely \emph{actions} and \emph{modules}.

\section{Actions in Evolog}
\label{sec:evolog-actions}

Actions allow for an ASP program to encode operations with \emph{side-effects} while maintaining fully declarative semantics. Actions are modelled in a functional style loosely based on the concept of monads as used in Haskell~\todo{cite something here!}. Intuitively, to maintain declarative semantics, actions need to behave as pure functions, meaning the result of executing an action (i.e. evaluating the respective function) must be reproducible for each input value across all executions. On first glance, this seems to contradict the nature of IO operations, which inherently depend on some state, e.g. the result of evaluating a function $getFileHandle(f)$ for a file $f$ will be different depending on whether $f$ exists, is readable, etc. However, at any given point in time - in other words, in a given state of the world - the operation will have exactly one result (i.e. a file handle or an error will be returned). A possible solution to making state-dependent operations behave as functions is therefore to make the state of the world at the time of evaluation part of the function's input. A function $f(x)$ is then turned into $f'(s, x)$  where $s$ represents a specific world state. The rest of this section deals with formalizing this notion of actions.

\todo{Define non-disjunctive ASP-Core2 in detail in preliminaires. Give detailed definition of all "standard ASP" elements referenced here!}

\subsection{Syntax}
\label{subsec:evolog-actions-syntax}

\begin{definition}[Action Rule, Action Program]
\label{def:action-rule-syntax}
An \emph{action rule} $R$ is of form
\[
	a_H : @t_{act} = act_{res} \leftarrow l_1,\ldots,l_n.
\]
where
\begin{itemize}
	\item $a_H$ is an atom called \emph{head atom},
	\item $t_{act}$ is a functional term called \emph{action term},
	\item $act_{res}$ is a term called \emph{(action-)result} term
	\item and $l_1,\ldots,l_n$ are literals constituting the \emph{body} of $R$.
\end{itemize}
An \emph{action program} $P$ is a set of (classic ASP-)rules and action rules.
\end{definition}

\subsection{Semantics}
\label{subsec:evolog-actions-semantics}

To properly define the semantics of an action program according to the intuition outlined at the start of this section, we first need to formalize our view of the "outside world" which action rules interact with. We call the world in which we execute a program a \emph{frame} - formally, action programs are always evaluated \emph{with respect to a given frame}. The behavior of actions is specified in terms of \emph{action functions}. The semantics (i.e. interpretations) of action functions in a program are defined by the respective frame. \\

To get from the practical-minded action syntax from Definition~\ref{def:action-rule-syntax} to the formal representation of an action as a function of some state and an input, we use the helper construct of an action rule's \emph{expansion} to bridge the gap. \\

Intuitively, the expansion of an action rule is a syntactic transformation that results in a more verbose version of the original rule called \emph{application rule} and a second rule only dependent on the application rule called \emph{projection rule}. A (ground) application rule's head atom uniquely identifies the ground instance of the rule that derived it. As one such atom corresponds to one action executed, we call a ground instance of an application rule head in an answer set an \emph{action witness}. 

\todo{define (classic ASP) grounding and substitutions in preliminaries}

\begin{definition}[Action Rule Expansion]
Given a non-ground action rule $R$ with head atom $a_H$, action term $f_{act}(i_1,\ldots,i_n)$ and body  $B$ consisting of literals $l_1,\ldots,l_m$, the expansion of $R$ is a pair of rules consisting of an \emph{application rule} $R_{app}$ and \emph{projection rule} $R_{proj}$. $R_{app}$ is defined as
\[
	a_{res}(f_{act}, S, I, f_{act}(S, I)) \leftarrow l_1,\ldots,l_n.
\]
where $S$ and $I$ and function terms called \emph{state-} and \emph{input-}terms, respectively.
An action rule's state term has the function symbol $\mathit{state}$ and terms $fn(a_H), fn(l_1),\ldots,fn(l_m)$, with the expression $fn(l)$ for an atom or literal $l$ denoting a function term representing $l$. The (function-)term representation of a literal $p(t_1,\ldots,t_n)$ with predicate symbol $p$ and terms $t_1,\ldots,t_n$ uses $p$ as function symbol. For a negated literal $\mathit{not}~p(t_1,\ldots,t_n)$, the representing function term is $not(p(t_1,\ldots,p_n))$. \\
The action input term is a "wrapped" version of all arguments of the action term, i.e. for action term $f_{act}(t_1,\ldots,t_n)$, the corresponding input term is $input(t_1,\ldots,t_n)$. \\
The projection rule $R_{proj}$ is defined as
\[
	a_H \leftarrow a_{res}(f_{act}, S, I, v_{res}).
\]
where $a_H$ is the head atom of the initial action rule $R$ and the (sole) body atom is the action witness derived by $R_{app}$, with the application term $f_{act}(S, I)$ replaced by a variable $v_{res}$ called \emph{action result variable}.
\end{definition}

%% NOTE: Action func. term is uninterpreted in nonground view! Application formally happens during grounding, i.e. action fnc is applied in grounding, ground version has action result.

%% TODO define frame as interpretation function for action func symbols.
%% TODO define grounding with respect to a frame - grounding eliminates action terms, i.e. action functions are applied during grounding!

\begin{definition}[Frame]
Given an action program $P$ containing action application terms $A = \{a_1,\ldots,a_n\}$, a frame $F$ is an interpretation function such that,for each application term $f_{act}(S, I) \in A$ where $S \in H_{U}(P)^{*}$ and $I \in H_{U}(P)^{*}$, $F(f_{act}): H_{U}(P)^{*} \times H_{U}(P)^{*} \mapsto H_{U}(P)$.
\end{definition}

Example \ref{ex:action-rule-expansion} demonstrates the expansion of an action rule as well as a compatible example frame for the respective action.

\begin{example}[Expansion and Frame]
\label{ex:action-rule-expansion}
Consider the (action-)rule $R$ which writes a message into a file if $dom(X)$ holds:
\[
	write\_result(F, R) : @fileWrite(F, S) = R \leftarrow file\_handle(F), message(S), dom(X).
\]
The expansion of $R$ is
\begin{align*}
	a_{res}(fileWrite, s(F, S, X), i(F, S), &fileWrite(s(F, S, X), i(F, S))) \leftarrow \\
		&file\_handle(F), message(S), dom(X). \\
	write\_result(R) \leftarrow a_{res}(fileWrite, &s(F, S, X), i(F, S), R).
\end{align*}

\end{example}




















\section{Syntax}
\label{sec:lang-syntax}

Every valid ASP-Core2 program is a valid Evolog program. In addition, Evolog programs may contain \emph{action rules} and \emph{module literals}.

\begin{definition}[Action Rule]
\label{def:action-rule}
Action Rules are ASP rules that have a body as defined by the ASP-Core2 standard~\cite{asp-core2} and an \emph{action head}, where an action head is of the following form:
\[
h : @a[i_1,\ldots,i_n] = v_r
\]
where
	\begin{itemize}
		\item the head atom $h$ is an ASP atom of form $p(t_1,\ldots,t_n)$ with $p$ and $t_1 \ldots,t_n$ being a predicate symbol and a list of terms, respectively.
		\item the function symbol $a$ is the name of an action function, i.e. an identifier starting with a lower-case letter
		\item action input terms $t_1$ through $t_n$ are a list of terms
		\item result variable $r_v$  is a variable.
	\end{itemize}
\end{definition}

Action result variables must not occur in the rule body.

\begin{definition}[Module Literals]
TBD
\end{definition}

\section{Semantics}
\label{sec:lang-semantics}

\subsection{Action Rules}
\label{sec:lang-semantics:action-rules}

\subsubsection*{Desiderata}

For every Evolog Program $P$ and answer set $A$, the following must be clearly defined:
\begin{itemize}
	\item $D1$: Which actions were executed by the program?
	\item $D2$: For every individual action $act$, what led to the action being executed, i.e. of which rule body is $act$ a consequence?
\end{itemize}
Combining $D1$ and $D2$ it follows that 
\begin{itemize}
	\item $D3$: for actions that depend on other actions, it is clearly visible in which sequence they were executed, i.e. the respective execution sequence can be unambiguously reconstructed using the answer set and program('s dependency graph).
\end{itemize}
Furthermore,
\begin{itemize}
\item $D4$: all state changes effected on the outside world by execution of $P$ are reflected in each answer set (as results of actions).
\end{itemize}

% action function defined as in following example: some_io(s, i) where s is a state tuple (t_1,...,t_n) and i is an input tuple (t_i1,...,t_in)

\begin{definition}[Expansion of action rules]
\label{def:action-rule-expansion}
\todo{This is an example, make into a proper definition}
Semantically, every action rule is equivalent to its \emph{expansion}:
\[
file1\_open(OP\_RES) : @fileInputStream[PATH] = OP\_RES :- file1(PATH). %  r1
\]
The expansion of $r1$ is:
\[
action\_result(r1, fileInputStream, PATH, fileInputStream(PATH)) :- file1(PATH). 
\]
\[
file1\_open(OP\_RES) :- action\_result(r1, fileInputStream, PATH, OP\_RES).
\]
\end{definition}
Consequently, it is ensured that for each ground instance of an action rule $R_a$ that fires, there is exactly one $action\_result$ instance in every answer set. We call this atom a \emph{witness of action $act$}. Requirement $D1$ is fulfilled through the existence of action witnesses. Furthermore, inspection of a program (or its dependency graph) and all action witnesses in an answer set yields the information demanded in $D2$.

\begin{definition}[Applicability of action rules]
\label{def:action-rule-applicability}
In order to guarantee $D1$ and $D4$, for every (ground) action rule $R_a$ that fires, it must hold that the corresponding \emph{witness atom} is part of \emph{every answer set}.
Implementations may further restrict this in order to ensure static verifiability of the condition (e.g. by restricting action rules to  the stratified part, i.e. common base program of a program).
\end{definition}

\begin{definition}[Rule Identifier]
\label{def:rule-id}
Given a non-ground Evolog rule $R$, $id(R)$ denotes a (program-wide) unique identifier of $R$.
\end{definition}

\begin{definition}[Action function]
\label{def:action-function}
An action function $f_{act}$ maps a rule id $r$, a tuple $S$, and a list of input terms $t_1,\ldots, t_n$ to a result term $t_{res}$.
\begin{itemize}
	\item Identifier $r$ references the rule (within a program) that is the \emph{action source} (i.e. that fires in order to trigger the action)
	\item State $S$ is a ground susbtitution for all body variables of the action source rule, i.e. it encodes the state of the world on which the action operates.
\end{itemize}
\end{definition}
In accordance with Definition \ref{def:action-function}, an action witness $action\_result(r_1, fileInputStream, PATH, OP\_RES)$ then reads as "Function $fileInputStream$, with action source $r_1$, applied to input $PATH$, given world state $(PATH)$, gives result $OP\_RES$".

\begin{definition}[Frame]
\label{def:evolog-frame}
A \emph{Frame} $F$ is an interpretation function 
\end{definition}

\begin{definition}[Interpretations of Evolog programs]
\label{def:evolog-interpretation}
An Evolog interpretation $I$ of a program $P$ is a tuple $(F, H)$ consisting of a \emph{Frame} $F$ and a herbrand interpretation $H$. The frame $F$ defines the action functions associated with rules in $P$.
\end{definition}

\begin{definition}[Evolog Model]
An evolog interpretation $I = (F, H)$ is a \emph{model} of evolog program $P$ iff $H$ is a \emph{stable model} of $P$, and all action witness atoms in $H$ are consistent with the action function definitions in $F$.
\end{definition}

\chapter{Evolog Reference Implementation}
\label{chap:implementation}
While Chapter~\ref{chap:language} gives a complete formal specification of the Evolog language extension, this chapter describes an implementation of said specification based on the Alpha \gls{asp} solver. All code referenced in this chapter is available on the official Github repository for Alpha~\cite{evolog-pr}.

\section{Architectural overview of Alpha}
\label{sec:alpha-arch-overview}

Alpha is a lazy-grounding \gls{asp} solver implemented in Java. In a nutshell, Alpha calculates Computation Sequences (see Definition~\ref{def:prelims-asp-semantics-compseq}) for an input program using a \gls{cdnl}-inspired solving algorithm. Starting from the set of facts contained in the program, an initial truth assignment is constructed. Based on this assignment, ground instances are calculated for all rules that \emph{could potentially fire} based on the assignment. These ground instances are converted into \emph{noGoods} and passed to the solver component which, using an adapted \gls{cdnl} approach, guesses a new assignment based on the last set of noGoods. This process is repeated until no more guesses are possible, at which point the current assignment is either returned as an answer set, or some conflict is detected, in which case a new noGood is learned, and the solver backtracks. However, the actual ground-solve-loop is - while arguably at the heart of Alpha - just a small part of the process through which programs get evaluated. Figure \ref{fig:alpha-arch} gives an overview of the building blocks making up the Alpha~\cite{alpha} system.

\begin{figure}[t]
    \includegraphics[width=\linewidth]{graphics/alpha-architecture.drawio.png}
    \caption{Alpha System Architecture}
    \label{fig:alpha-arch}
\end{figure}

\subsection{Parsing and Compilation}
\label{subsec:alpha-arch-compilation}

The core ground-and-solve component of Alpha supports only a subset of the input language described in Section~\ref{subsec:prelims-asp-syntax}. All language features supported by the parser, but not the solver itself, get compiled into equivalent constructs in the solver's internal representation. The following transformations are applied:
\begin{itemize}
    \item \emph{External Atom Linking}: Programs may only use external atoms whose implementations are known to the parser prior to parsing. Alpha provides a set of frequently used built-in external atoms out of the box, user-supplied code must be scanned through Alpha's API. Example~\ref{ex:user-supplied-externals} demonstrates the use of user-supplied atom definitions. All external atom implementations are linked during parsing, i.e. every external atom in the parsed program holds a reference to the implementing Java Method.
    \item \emph{Equality Literal Rewriting}: Literals like $A = B$ in rule bodies that establish an equality between variables are removed by replacing one variable with the other (e.g. $B$ with $A$ in the example). 
    \item \emph{Choice Head Rewriting}: Rules with a choice head get replaced by a set of rules and constraints that is semantically equivalent to the choice rule.
    \item \emph{Aggregate Literal Rewriting}: Aggregate literals like \texttt{N = \#count\{ X : interesting(X)\}} are replaced by a regular literal and a set of rules deriving instances of the replacement literal equivalent to the original aggregate literal. Subsection~\ref{subsubsec:alpha-arch-aggregate-rewriting} goes into more detail on the rewriting process.
    \item \emph{Enumeration Atom Resolution}: Alpha provides a feature where terms can be enumerated, i.e. the solver maps user-supplied terms to integers. This is used internally for Aggregate Literal Rewriting. Section~\ref{subsubsec:alpha-arch-enum-resolution} describes how the atoms in question are resolved.
    \item \emph{Interval Term Rewriting}: Interval terms, i.e. terms of form $A..B$ are transformed into regular variables that are bound through special internal literals which supply all values of the interval as ground instances of the variable.
    \item \emph{Arithmetic Term Rewriting}: In order to simplfy grounding later, terms constituting arithmetic expressions such as $p(f(X * 3, Y - 4))$ are rewritten such that no arithmetic expressions occur in nested terms, i.e. the atom from before would be rewritten to $p(f(R1, R2)), R1 = X * 3, R2 = Y - 4$.
\end{itemize}
The result of the above list of transformations is what is called a \emph{normalized} program in Alpha, which can directly be passed to the evaluation component and solved. Rewriting of Aggregate Literals and Resolution of Enumeration Atoms are of interest in the context of the Evolog reference implementation, and shall therefore be described in more detail.

\subsubsection{Enumeration Atoms}
\label{subsubsec:alpha-arch-enum-resolution}
Alpha permits use of an "enumeration" solver directive, which allows programs to associate terms with consecutive integer keys.
Listing~\ref{lst:enum-usage} demonstrates using an enumeration to assign integer ordinals to a set of colors. 
\begin{lstlisting}[style=asp-code, label={lst:enum-usage}, caption={Using the Enumeration Directive to enumerate color symbols.}]
#enumeration_predicate_is ordinal.

color(white). color(red). color(magenta).
color(yellow). color(green). color(cyan).
color(blue). color(black).
    
numbered_color(COL, NUM) :- 
    color(COL), ordinal(colors, COL, NUM).    
\end{lstlisting}    
The directive \texttt{\#enumeration\_predicate\_is} designates the predicate \texttt{ordinal/3} as an \emph{Enumeration Predicate}. All occurrences of the an enumeration predicate get replaced with a special internal predicate \texttt{\_Enumeration/3}. Listing~\ref{lst:enum-rewritten} shows the program from before after transformation.
\begin{lstlisting}[style=asp-code, label={lst:enum-rewritten}, caption={Transformed color numbering.}]
color(white). color(red). color(magenta).
color(yellow). color(green). color(cyan).
color(blue). color(black).

numbered_color(COL, NUM) :- 
    color(COL), _Enumeration(colors,COL,NUM).
\end{lstlisting} 
Alpha's grounding component calculates valid ground substitutions for enumeration atoms as follows: \\
Given an enumeration atom $a_e$ with terms $t_{enum}, t_{value}$ and $t_{ord}$ and a partial substitution $\sigma$, assigning ground values to $t_{enum}$ and $t_{value}$,
\begin{itemize}
    \item If the value $\sigma t_{enum}$ is encountered for the first time, initialize a new empty map (i.e. set of pairs with unique first elements), and associate it with  term $\sigma t_{enum}$.
    \item If the map for $\sigma t_{enum}$ does not contain a mapping for key $\sigma t_{value}$, extend $\sigma$ by the mapping $t_{ord} \mapsto o$, where $ = s + 1$ and $s$ denotes the current map size for $\sigma t_{enum}$. Add the mapping $(\sigma t_{ord}, o)$ to the map and return the extended version of $\sigma$.
    \item If a mapping for $\sigma t_{enum}$ and $\sigma t_{value}$ exists, read the associated ordinal $o$, add it to $\sigma$ and return the extended substitution.
\end{itemize}    
From a semantics point of view, enumeration literals can intuitively be seen as "lazily assigned fixed-interpretation literals"  (see Definition~\ref{def:prelims-asp-semantics-fixedinterpretation-literals}) in the way that every enumeration atom is true for exactly the ground substitution generated upon grounding it for the first time. The program from Listing~\ref{lst:enum-rewritten} yields the following answer set:
\begin{align*}
    A = \{ &color(black), color(blue), color(cyan), color(green), color(magenta), color(red),\\
           &color(white), color(yellow), numbered\_color(black, 1), numbered\_color(blue, 2),\\
           &numbered\_color(cyan, 3), numbered\_color(green, 4), numbered\_color(magenta, 6),\\
           &numbered\_color(red, 7), numbered\_color(white, 8), numbered\_color(yellow, 5)\}
\end{align*}    

\subsubsection{Aggregate Atoms}
\label{subsubsec:alpha-arch-aggregate-rewriting}

Alpha supports \emph{Aggregate Literals} as defined in~\cite[p.~3]{asp-core2} by rewriting programs containing aggregate literals into sematically equivalent aggregate-free programs. A detailed description of Alpha's implementation of Aggregate Rewriting is available at~\cite{alpha-aggregate-support}, but would exceed the scope of this thesis. We will therefore focus on general concepts applying to all aggregate functions that are rewritten, and outline the rewriting procedure for aggregate literals where the minimum or maximum over a set of terms is calculated, which is the starting point for compilation of the newly introduced $\#list$ aggregate described in Section~\ref{def:list-aggregate}.

In the context of this section, we consider literals of form $X \odot \#\mathit{func}\{t_1,\ldots,t_n : l_1,\ldots,l_m\}$, where $X$ is a term, $\odot \in \{=,\leq\}$ and $\mathit{func} \in \{\mathit{min},\mathit{max}\}$, $t_1,\ldots,t_n$ are terms and $l_1,\ldots,l_m$ literals, respectively.

\begin{definition}[Aggregate Terms, Elements, Local and Global Variables, Dependencies]
In the following, we use the notation $var(l_1,\ldots,l_n)$ for literals $l_1,\ldots,l_n$ to denote the set of variable terms occuring in said literals. Consider a rule $r$ containing aggregate literal $l_{agg} =  X \odot \#\mathit{func}\{t_1,\ldots,t_n : l_1,\ldots,l_m\}$:
\[
    H \leftarrow l_{agg}, b_1,\ldots,b_k.
\]
Then, the set $var(l_1,\ldots,l_n) \cap var(b_1,\ldots, b_k)$ is called \emph{global variables} of $l_{agg}$, denoted $glob(l_{agg})$. Roughly speaking, global variables of an aggregate literal are all variables occurring within the aggregate literal as well as other body literals of $r$. Given a set $V = glob(l_{agg})$, the set of literals $dep(l_{agg})$ is the mnimal set of literals that, given a substitution $\sigma$, must be ground after application of $\sigma$, in order for $\sigma$ to also ground $l_{agg} \cup dep(l_{agg})$.
Intuitively, global variables of an aggregate literals are all variables for which Alpha's lazy grounding component needs a ground value, in order to be able to calculate a ground instance of the aggregate literal itself. Dependencies of an aggregate literal $l_{agg}$ are all literals of which the grounder needs ground instances in order to calculate a grounding of all global variables of $l_{agg}$.
\end{definition} 

In order to translate a rule $r$ containing an aggregate literal $l_{agg} = X \odot \#\mathit{func}\{t_1,\ldots,t_n : l_1,\ldots,l_m\}$ with global variables $glob(l_{agg})$, dependencies $dep(l_{agg})$ into a semantically equivalent set of rules, the following steps are taken:
\begin{itemize}
    \item Generate a unique identifier $id(l_{agg})$ (typically some integer) for $l_{agg}$
    \item Construct rule $r´$ in which $l_{agg}$ is replaced by a literal $aggregate\_result(id(l\_{agg})\_args, X)$
    \item Generate an \emph{element rule} which derives one atom per element that is being aggregated over. Given the aggregate element $t_1,\ldots,t_n : l_1,\ldots,l_m$, the corresponding element rule is $id(l_{agg})\_element\_tuple(id(l\_{agg})\_args, t_1,\ldots,t_n) \leftarrow l_1,\ldots,l_m, dep(l_{agg})$, i. e. the element rule body consists of all literals of the aggregate element together with all dependencies of the aggregate literal.
    \item Generate a set of \emph{encoding rules} that encode the actual aggregate function over all elements as derived by the element rule and derives instances of the $aggregate\_result/2$ predicate.
\end{itemize}     

Example~\ref{ex:aggregate-rewriting-min} demonstrates how an aggregate literal for the minimum function gets rewritten by Alpha.

\begin{example}
\label{ex:aggregate-rewriting-min}
Consider the program from Listing~\ref{lst:aggregate-rewriting-min-source}. Based on some facts of type $employee/3$ which assert that an employee works in some department and earns a given salary, we use a $\#min$-aggregate to find the employee with the lowest salary in each department.
\begin{lstlisting}[style=asp-code, label={lst:aggregate-rewriting-min-source}, caption={ASP program to find the worst paid employee per department.}]
employee(bob, sales, 2000).
employee(alice, development, 6000).
employee(dilbert, development, 4500).
employee(jane, sales, 3500).
employee(carl, controlling, 5000).
employee(bill, controlling, 4000).
employee(claire, development, 5000).
employee(mary, sales, 3000).
employee(joe, controlling, 5500).
    
department(DEP) :- employee(_, DEP, _).
    
min_salary(SAL, DEP) :- 
    SAL = #min{S : employee(_, DEP, S)}, 
    department(DEP).
worst_paid(DEP, EMP) :- 
    min_salary(S, DEP), employee(EMP, DEP, S).    
\end{lstlisting} 
Listing~\ref{lst:aggregate-rewriting-min-rewritten} shows a rewritten version of the original program.
\begin{lstlisting}[style=asp-code, label={lst:aggregate-rewriting-min-rewritten}, caption={The program from Listing~\ref{lst:aggregate-rewriting-min-source} in its rewritten version.}]
employee(bob, sales, 2000).
employee(alice, development, 6000).
employee(dilbert, development, 4500).
employee(jane, sales, 3500).
employee(carl, controlling, 5000).
employee(bill, controlling, 4000).
employee(claire, development, 5000).
employee(mary, sales, 3000).
employee(joe, controlling, 5500).

department(DEP) :- employee(_0, DEP, _1).
worst_paid(DEP, EMP) :- 
    min_salary(S, DEP), employee(EMP, DEP, S).
min_salary(SAL, DEP) :- 
    min_1_result(min_1_args(DEP), SAL), department(DEP).
min_1_element_tuple_less_than(ARGS, LESS, THAN) :- 
    min_1_element_tuple(ARGS, LESS), 
    min_1_element_tuple(ARGS, THAN), LESS <THAN.
min_1_element_tuple_has_smaller(ARGS, TPL) :- 
    min_1_element_tuple_less_than(ARGS, _3, TPL).
min_1_min_element_tuple(ARGS, MIN) :- 
    min_1_element_tuple(ARGS, MIN), 
    not min_1_element_tuple_has_smaller(ARGS, MIN).
min_1_result(ARGS, M) :- 
    min_1_min_element_tuple(ARGS, M).
min_1_element_tuple(min_1_args(DEP), S) :- 
    employee(_2, DEP, S), department(DEP).
\end{lstlisting}   
In the rewritten version, the rule in line 14, which derives $\mathit{min\_salary/2}$ has the aggregate literal replaced with a regular literal, instances for which are derived by newly added rules that together encode the aggregate function. 
The individual elements over which a minimum is being calculated are derived as instances of the \texttt{min\_1\_element\_tuple/2} predicate using the rule in line 26. In order to find the minimal instance of \texttt{min\_1\_element\_tuple/2}, we first establish a "less than"-relation (i.e. a partial order based on numeric comparison of the second term of the element tuple instances) using the rule in line 16. The minimum element is then the one for which we cannot find a "smaller" element. This element is derived by the rule in line 21.
The single answer set of the rewritten program (filtered for instances of \texttt{worst\_paid/2}) is $A =\{worst\_paid(controlling,bill),~worst\_paid(development,dilbert),~worst\_paid(sales,bob)\}$.
\end{example}    

\subsection{Evaluation}

Once a program has been parsed and compiled using the steps described in~\ref{subsec:alpha-arch-compilation}, the compiled program can be evaluated (i.e. solved). Evaluation consists of two steps - first, the stratified bottom (\emph{"common base program, CBP"}, see~\ref{def:prelims-asp-semantics-cbp}) is evaluated using a bottom-up evaluation algorithm based on iterative fixpoint calculation as described in Definition~\ref{def:prelims-asp-semantics-stratified-compseq}. The remainder of the program, i.e. the part which cannot be handled using stratified evaluation is then solved using Alpha's \gls{cdnl}-based answer set search described in Section~\ref{subsec:prelims-lazygrounding-alpha-cdnl}. Since the stratified evaluation component is where Evolog Actions are handled, Section~\ref{subsubsec:impl-stratified-eval} gives an overview of how stratified evaluation is implemented in Alpha.

\subsubsection{Stratified Evaluation in Alpha~\cite{partial-eval}}
\label{subsubsec:impl-stratified-eval}

Alpha's stratified evaluation component takes a normalized program as described in Section~\ref{subsec:alpha-arch-compilation} as its input. A dependency graph for the program in question is calculated based on predicate dependencies based on Definition~\ref{def:prelims-asp-semantics-nonground-splitting-set} where each dependency is labelled either "+" or "-" to distinguish dependencies through positive and negative body literals, respectively. Calculating the component graph (i.e. the graph resulting from condensing the dependency graph into its stronlgy connected components), one ends up with a directed acyclic graph, which is labelled such that all component nodes containing cycles through negative (i.e. labelled "-") dependencies get labelled as "unstratifiable". The \gls{cbp} is then the program consisting of all rules from components that are not reachable by any path containing an unstratifiable component node. Based on the resulting subset of the component graph, a partition satisfying the criteria for a stratification according to Definition~\ref{def:prelims-asp-semantics-stratification} is calculated.\\
\\
The stratified part of the input program is then evaluated in order of ascending stratum using Algorithm~\ref{alg:cbp-eval}. For each stratum, we first calculate applicable ground rules based on all facts in the program as well as those derived when evaluating lower strata. As long as rule application yields new atoms, additional ground rules are computed and evaluated, until no new information can be derived, in which case a fixpoint has been reached on the current stratum. The result of the evaluation procedure is a program consisting of the rules of the non-stratified part of the input program, plus original facts as well as all new facts dervied during stratified evaluation.

\begin{algorithm}[!h]
\SetAlgoLined
\SetKwInOut{Input}{Input}\SetKwInOut{Output}{Output}
\SetKwRepeat{Do}{do}{while}
\Input{Stratification $S = \{S_0,\ldots,S_n\}$}
\Input{Input Program $P = (F_{in}, R_{in})$ consisting of facts and rules $F_{in}$, $R_{in}$}
\Output{partially evaluated program $P_{eval}$}
Facts $F_{out}$ = \emph{facts($F_{in}$)} \\
\ForEach{stratum $S_i$ \emph{\textbf{in}} $S_0 \ldots S_n$}{
    initialize derived atoms in stratum $A = F_{out}$ \\
    initialize derived atoms in individual run $A_{new} = A$ \\
    \Do{new atoms derived in last run, i.e. $A_{new} \ne A$}{
        $A \leftarrow A \cup A_{new}$ \\
        $A_{new} = \emptyset$ \\
        calculate applicable ground rules $R_{app}$ in $P_{S_i}$ based on $A$\\
        \ForEach{rule $r \in R_{app}$}{
            $A_{new} \leftarrow A_{new} \cup fireRule(r)$ \\
        }
    }
    add newly derived atoms to output facts, $F_{out} \leftarrow F_{out} \cup A$ \\
}
\textbf{return} $P_{eval} = (F_{out}, (P \setminus S))$\\
\caption{Stratified up-front evaluation of \gls{cbp} in Alpha}\label{alg:cbp-eval}
\end{algorithm}

\section{Implementing the Evolog extension in Alpha}

After the technical overview of Alpha in Section~\ref{sec:alpha-arch-overview}, we now turn to how Alpha has been adapted to implement the Evolog language extension specified in Chapter~\ref{chap:language}. Specifically, additions were made in the following areas:
\begin{itemize}
    \item \emph{Parsing} - Alpha's program parser has been extended to be able to handle rules with action heads, module definitions, and references to modules from rule bodies.
    \item \emph{Aggregate Rewriting} - In order to accomodate \emph{list aggregation literals}, a concise construction syntax for list terms, additional logic to compile this new type of aggregate literal has been added.
    \item \emph{Action Execution Service} - A new component responsible for executing actions has been added. It ensures idempotency of action rule application and serves as an abstraction layer between actual action implementations and Alpha's stratified evaluation component.
    \item \emph{Module Atom Compilation} - A new program preprocessing component has been added in order to compile module atoms, i.e. link name-based references to modules with the actual implementation code.
\end{itemize}    
Figure~\ref{fig:alpha-evolog-arch} shows an updated overview of Alpha's architecture, with new components highlighted.

\begin{figure}[t]
    \includegraphics[width=\linewidth]{graphics/alpha-evolog-architecture.drawio.png}
    \caption{Alpha for Evolog System Architecture}
    \label{fig:alpha-evolog-arch}
\end{figure}

\subsection{Implementing Action Support}
\label{subsec:implementation-actions}

In this section, we give a detailed description of how Action support, conforming to definitions from Section~\ref{sec:evolog-actions} is implemented in Alpha.

\paragraph{Implemented Actions} 
The actual actions provided in the reference implementation are intended to be sufficient for basic file (i.e. stream-)based input- and output operations. Specifically, Alpha offers actions to
\begin{itemize}
    \item open a file handle for reading, i.e. obtaining an input stream,
    \item read from an input stream,
    \item close an input stream
    \item open a file handle for writing, i.e. obtaining an output stream,
    \item write to an output stream
    \item close an output stream
\end{itemize}
The idea of this selection of actions is that these can be used as the basic building blocks from which many more complex IO-Tasks, such as parsing specific file formats, waiting for user input, implementing command shells, etc. can be composed.

\subsubsection{Supported Programs}
\label{subsubsec:implementation-actions-support}
As stated in Section~\ref{subsec:evolog-actions-restrictions}, Evolog programs are required to be \emph{transparent} in terms of action execution, i.e. for every action that gets executed, there must be a corresponding witnessing atom in the respective answer set. This requirement is formalized in Definition~\ref{def:evolog-actions-transparency}. While the formal definition only states that if an action rule fires (i.e. an action is executed), the corresponding head must be contained in an answer set - which, aside from the effect that is applied on the outside world, is not any different from the general meaning of a rule that, if a the body is true, the head is as well - this leads directly to the practical problem of how to enforce this in an implementation. Since actions taken in the "outside world" (such as sending data over a network interface, writing into a file, ending a process, etc.) can not be retracted, an Evolog interpreter must be able to establish action rule transparency \emph{prior to actual execution}. In general, any answer set search algorithm based on guesses and propagation through nogoods as described in Section~\ref{subsec:prelims-lazygrounding-alpha-cdnl} does not know up-front whether an atom that gets assigned as true at some point in search, will be true in a resulting answer set - every later propagation step could lead to a violation of one or more nogoods and subsequent backtracking. With that in mind, we establish that it is not feasible to have any form of guesses depend on the result of an action, since whenever a solver makes a guess, there is a chance the program might be unsatisfiable, in which case the action would not be transparent anymore. While it may certainly possible to identify classes of programs with guesses where action transparency can be assured through clever static analysis, our implementation takes a simple approach. In the reference implementation based on Alpha, an Evolog program $P$ is supported, if and only if $P = cbp(e)$, using the notion of the common base program as given in Definition~\ref{def:prelims-asp-semantics-cbp}. Since the \gls{cbp} of a program is by definition stratified, we can be sure there will be an answer set, and therefore have a guarantee of action transparency as well. At the same time, programs using the guess-and-check pattern can still be executed under this restriction, by encapsulating the guess-and-check part in a module. Consider the program from Example~\ref{ex:bin-packing-module}, which has a single answer set, in which individual atoms correspond to answer sets of the used modules. In a program structured like that, one can use modules to encapsulate arbitrarily complicated calculations with any number of guesses while maintaining stratifiability of the top-level program.
Example~\ref{ex:implementation-actions-unsupported} demonstrates an unsupported program as well as a supported version of it based on a short interactive application.

\begin{example}
\label{ex:implementation-actions-unsupported}    
The program in Listing~\ref{lst:implementation-actions-unsupported1} is intended to be a very short "greeter" application. The user is asked to enter their name, which is then prependen with "Hello" and printed to the terminal. However, the constraints \texttt{:- usr\_input\_res(success(line(""))).} and \texttt{:- usr\_input\_res(error(\_)).} render the program \emph{unsupported}. In fact, the program is unsatisfiable with respect to any Frame (see Definition~\ref{def:evolog-frame}) where the user input is either empty or reading from \texttt{stdin} results in an error. Since, however, in order to evaluate the constraints in question, two actions with side-effects have to be performed first - namely one line needs to be written to \texttt{stdout} and one read from \texttt{stdin} - we would inevitably end up in a state where some side-effects have been applied to the outside world, without there being a corresponding action witness in an answer.

\begin{lstlisting}[style=asp-code, label={lst:implementation-actions-unsupported1}, caption={An unsupported "greeter" application.}]
    prompt_text("Hello user, enter your name: ").
        
    write_prompt_res(R) : @streamWrite[STDOUT, PROMPT] = R :- 
        prompt_text(PROMPT), &stdout(STDOUT).
    usr_input_res(INP) : @streamReadLine[STDIN] = INP :- 
        write_prompt_res(success(_)), &stdin(STDIN).

    :- usr_input_res(success(line(""))).
    :- usr_input_res(error(_)).  

    write_greeting_res(R) : @streamWrite[STDOUT, GREETING] = R :- 
        usr_input_res(success(line(TEXT)), _),
        &stdlib_string_concat["Hello ", TEXT](GREETING).
        &stdout(STDOUT).   
\end{lstlisting}       

Listing~\ref{lst:implementation-actions-supported1} shows a supported version of the program from Listing~\ref{lst:implementation-actions-unsupported1}. Here, rather than constraints, regular rules are used for error checking, and in case of an error, a corresponding message is written to \texttt{stdout}. In this version, the program might not run in the "intended" fashion, but there will always be an answer set, i.e. effects the program applies on the outside world will always be made transparent.

\begin{lstlisting}[style=asp-code, label={lst:implementation-actions-supported1}, caption={A supported version of the "greeter" application.}]
    prompt_text("Hello user, enter your name: ").
        
    write_prompt_res(R) : @streamWrite[STDOUT, PROMPT] = R :- 
        prompt_text(PROMPT), &stdout(STDOUT).
    usr_input_res(INP) : @streamReadLine[STDIN] = INP :- 
        write_prompt_res(success(_)), &stdin(STDIN).

    error("Input String empty") :- usr_input_res(success(line(""))).
    error(MSG):- usr_input_res(error(MSG)).
    error_occurred :- error(_).  

    write_greeting_res(R) : @streamWrite[STDOUT, GREETING] = R :- 
        usr_input_res(success(line(TEXT)), _), not error_occurred.
        &stdlib_string_concat["Hello ", TEXT](GREETING),
        &stdout(STDOUT).
        
    write_errmsg_res(R) : @streamWrite[STDOUT, ERRMSG] = R :-
        error(ERR),
        &stdlib_string_concat["An error occurred: ", ERR](ERRMSG),
        &stdout(STDOUT).
\end{lstlisting} 
\end{example}     

\subsubsection{Action Implementation}
\label{subsucsec:implementation-actions-execution}

Action execution in Alpha is tied into the stratified evaluation component described in Section~\ref{subsubsec:impl-stratified-eval}. Specifically, whenever a ground instance of an action rule is fired, the corresponding action function is evaluated. The expansion of action rules described in Definition~\ref{def:action-rule-expansion} happens implicitly:
Instead of actually splitting a rule with an action head into an application- and projection-rule as described in the formal definition, the implementation constructs an internal representation of the ground rule in question. This internal representation is equivalent to an \emph{action application term} $f_{act}(S, I)$ (see Definition~\ref{def:action-rule-expansion}) in that it uniquely identifies a ground instance of an action rule within a program. After calculating the application term for the ground rule to be fired, Alpha's action execution component queries an in-memory table called \emph{action record} for existing entries, where the key is equal to the calculated action application term. If such a mapping exists, the cached action result is returned, otherwise, the Java method associated with the action function is executed and a new mapping is added to the action record. The action record, i.e. action result cache, is used to make sure actions only get executed once, regardless of how often a certain ground instance of a rule is "fired" during program evaluation. Listing~\ref{lst:implementation-actions-execution-service} shows an (abbreviated) version of Alpha's action execution component.

\begin{lstlisting}[style=java, label={lst:implementation-actions-execution-service}, caption={Alpha's action execution logic}]
public class ActionExecutionServiceImpl implements ActionExecutionService {

	private final ActionImplementationProvider actionProvider;
	private final Map<ActionInput, ActionWitness> actionRecord = new HashMap<>();

	// [Constructors and utility methods omitted for brevity]

	@Override
	public ActionWitness execute(String actionName, int sourceRuleId, Substitution sourceRuleInstance, List<Term> inputTerms) {
		ActionInput actInput = new ActionInput(actionName, sourceRuleId, sourceRuleInstance, inputTerms);
		return actionRecord.computeIfAbsent(actInput, this::execute);
	}

	private ActionWitness execute(ActionInput input) {
		Action action = actionProvider.getSupportedActions().get(input.name);
		ActionResultTerm<?> result = action.execute(input.inputTerms);
		return new ActionWitness(input.sourceRule, input.instance, input.name, input.inputTerms, result);
	}

	private static class ActionInput {

		private final String name;
		private final int sourceRule;
		private final Substitution instance;
		private final List<Term> inputTerms;

        // [Constructors and utility methods omitted for brevity]

	}

}
\end{lstlisting}    

Example~\ref{ex:implementation-actions-loop} demonstrates a program where actions are used in recursive rules where ground instances potentially have to be evaluated multiple times and the aforementioned caching of action results comes into play.

\begin{example}
\label{ex:implementation-actions-loop}
Listing~\ref{lst:implementation-actions-loop} shows the "echo" program. It asks the user to enter text on the command line, and "echoes", i.e. prints, the previous input until the user enters \texttt{EXIT}, at which point the program terminates. The "loop-like" behavior is realized through a positive recursive dependency cycle over predicates \texttt{write\_prompt\_res/2}, \texttt{usr\_input\_res/2} and \texttt{write\_echo\_res/2}.

\begin{lstlisting}[style=asp-code, label={lst:implementation-actions-loop}, caption={An "echo" application which echoes user input written using Evolog actions.}]
prompt_text("Hello user, tell me something: ").
cancel_cmd("EXIT").
    
write_prompt_res(R, 0) : @streamWrite[STDOUT, PROMPT] = R :- 
    prompt_text(PROMPT), &stdout(STDOUT).
write_prompt_res(R, N) : @streamWrite[STDOUT, PROMPT] = R :- 
    prompt_text(PROMPT), write_echo_res(success(_), K), 
    N = K + 1, &stdout(STDOUT).
usr_input_res(INP, N) : @streamReadLine[STDIN] = INP :- 
    write_prompt_res(success(_), N), &stdin(STDIN).
write_echo_res(R, N) : @streamWrite[STDOUT, ECHO] = R :- 
    usr_input_res(success(line(TEXT)), N), TEXT != CANCEL_CMD, 
    cancel_cmd(CANCEL_CMD), &stdout(STDOUT),
    &stdlib_string_concat["You said: ", TEXT](ECHO).
        
write_goodbye_res(R) : @streamWrite[STDOUT, MSG] = R :- 
    usr_input_res(success(line(TEXT)), _), cancel_cmd(CMD), 
    TEXT = CMD, MSG = "Goodbye, user! Sad to see you leave", 
    &stdout(STDOUT).   
\end{lstlisting}    

\end{example}    


\paragraph{Action Result Terms} The way action results are represented in Alpha takes its inspiration from the \texttt{Either} type present in many functional programming languages or libraries such as Vavr~\cite{vavr-io}. For example, an instance of type \texttt{Either<InputStream, Exception>}
can either hold an instance of the "left" type, i.e. \texttt{InputStream} or an instance of the "right" type, i.e. \texttt{Exception}. Similarly, \emph{action result terms} as defined in Definition~\ref{def:action-rule-syntax} are function terms of form $r_t(r_v)$, where $r_t \in \{success, error\}$ is the \emph{result type} and $r_v$ the \emph{result value}. Result type $success$ indicates the action completed normally, and in this case, $r_v$ holds the result of the successful action, e.g. a term representing an input stream. If the result type is $error$, for example when trying to obtain a read handle on a non-existing file, $r_v$ typically holds an error message.
Listing~\ref{lst:implementation-actions-actionfunc} shows the Java Interface specifying an Action result term in Alpha.

\begin{lstlisting}[style=java, label={lst:implementation-actions-actionfunc}, caption={Java definition of an action result term.}]
    public interface ActionResultTerm<T extends Term> extends FunctionTerm {
        public static final String SUCCESS_SYMBOL = "success";
        public static final String ERROR_SYMBOL = "error";
        /**
         * True if the action that generated this 
         * result was successful (i.e. executed normally).
         */
        boolean isSuccess();
        /**
         * True if the action that generated this 
         * result failed (i.e. threw an error in execution).
         */
        boolean isError();
        /**
         * Gets the actual value wrapped in this result.
         * Either a term representing the action return 
         * value or a string term representing an error
         * message.
         */
        T getValue();
    }    
\end{lstlisting}   

\subsection{Implementing Program Modularization}

Since, formally, module atoms are just a special type of external atoms, implementing support for Modularization as described in~\ref{sec:evolog-modules} mainly has to deal with how to parse module definitions and how to construct external atoms according to Alpha's internal representation from a set of parsed module definition for atoms which reference these definitions. While not strictly necessary, this section also discusses the newly added \texttt{\#list\{...\}} aggregate, which has been added as syntactic sugar for combining a set of terms into a single list term according to Definition~\ref{def:asp-list-term}.

\subsubsection{List aggregation}
\label{sec:implementation-list-aggregation}

External atoms are defined as having input- and output-\emph{terms} (but not multisets of terms). At the same time, it is self-evident that - in most cases - one has to pass multiple input facts to a module program - for instance, a graph has sets of edges and vertices, all of which need to be passed to a module dealing with graph problems. This is where lists come in. By establishing the convention that the \emph{single} input fact to a module program always has one or more \emph{list terms} as arguments, we are able to encode all information that needs to be passed to the module into a single atom, while still staying true to external atom semantics as supported by Alpha.
The practical problem arising from this is that constructing list terms in ASP is rather cumbersome. One has to write rules to
\begin{itemize}
    \item specify \emph{what} goes into the list, i.e. one or more rules encoding eligible \emph{list elements},
    \item specify \emph{in what sequence} elements go into the list, i.e. rules which establish a total order between list elements,
    \item actually construct the list, i.e. starting from the end, recursively construct the term holding all list elements.
\end{itemize}
Listing~\ref{lst:graph-list-aggregation} shows an example of this approach, where a list term is constructed for vertices and edges of a graph, respectively.
\begin{lstlisting}[style=asp-code, label={lst:graph-list-aggregation}, caption={ASP code to create vertex and edge lists for a given graph.}]
%% pack vertices into a vertex list
vertex_element(E) :- vertex(E).
% First, establish ordering of elements
vertex_element_less(N, K) :- 
    vertex_element(N), vertex_element(K), N < K.
vertex_element_not_predecessor(N, K) :- 
    vertex_element_less(N, I), vertex_element_less(I, K).
vertex_element_predecessor(N, K) :- 
    vertex_element_less(N, K), 
    not vertex_element_not_predecessor(N, K).
vertex_element_has_predecessor(N) :- 
    vertex_element_predecessor(_, N).
% Now build the list as a recursively nested function term
vertex_lst_element(IDX, list(N, list_empty)) :- 
    vertex_element(N), 
    not vertex_element_has_predecessor(N), 
    IDX = 0.
vertex_lst_element(IDX, list(N, list(K, TAIL))) :- 
    vertex_element(N), 
    vertex_element_predecessor(K, N), 
    vertex_lst_elemen(PREV_IDX, list(K, TAIL)), 
    IDX = PREV_IDX + 1.
has_next_vertex_element(IDX) :- 
    vertex_lst_element(IDX, _), 
    NEXT_IDX = IDX + 1, 
    vertex_lst_element(NEXT_IDX, _).
vertex_lst(LIST) :- 
    vertex_lst_element(IDX, LIST), 
    not has_next_vertex_element(IDX).

%% pack edges into an edge list
edge_element(edge(V1, V2)) :- edge(V1, V2).
% First, establish ordering of elements
edge_element_less(N, K) :- 
    edge_element(N), edge_element(K), N < K.
edge_element_not_predecessor(N, K) :- 
    edge_element_less(N, I), edge_element_less(I, K).
edge_element_predecessor(N, K) :- 
    edge_element_less(N, K), 
    not edge_element_not_predecessor(N, K).
edge_element_has_predecessor(N) :- 
    edge_element_predecessor(_, N).
% Now build the list as a recursively nested function term
edge_lst_element(IDX, list(N, list_empty)) :- 
    edge_element(N), 
    not edge_element_has_predecessor(N), 
    IDX = 0.
edge_lst_element(IDX, list(N, list(K, TAIL))) :- 
    edge_element(N), 
    edge_element_predecessor(K, N), 
    edge_lst_element(PREV_IDX,list(K, TAIL)), 
    IDX = PREV_IDX + 1.
has_next_edge_element(IDX) :- 
    edge_lst_element(IDX, _), 
    NEXT_IDX = IDX + 1, 
    edge_lst_element(NEXT_IDX, _).
edge_lst(LIST) :- 
    edge_lst_element(IDX, LIST), 
    not has_next_edge_element(IDX).
\end{lstlisting} 
Given facts $vertex(a),~vertex(b),~vertex(c),~edge(a,b),~edge(b,c),~edge(c,a)$, we get the following answer set (filtered for predicates $edge\_lst/1$ and $vertex\_lst/1$):
\begin{align*}
    A = \{&edge\_lst(list(edge(c,a), list(edge(b,c), list(edge(a,b), \\
           &list\_empty)))), vertex\_lst(list(c, list(b,list(a,list\_empty))))\}
\end{align*}    

Considering Listing~\ref{lst:graph-list-aggregation}, it is easy to see that construction of a list term looks always the same, regardless of the actual elements going into the list. Furthermore, we can observe a distinct similarity to the code generated to rewrite a $\#min$-aggregate outlined in Section~\ref{subsubsec:alpha-arch-aggregate-rewriting}. In the following, we define a new aggregate function, $\#list$, which Alpha rewrites into a generalized version of the list encoding from Listing~\ref{lst:graph-list-aggregation}.

\begin{definition}[List Aggregate]
\label{def:list-aggregate}    
A \emph{list aggregate} is an aggregate atom of the following form
\[
    t_{res} = \#list\{ t_{elem} : l_{1},\ldots,l_{n}\}
\]
where $t_{res}$ and $t_{elem}$ are terms called result- and element-term, respectively, and $l_{1},\ldots,l_{n}$ are literals.
\end{definition}

Definition~\ref{def:list-aggregate} formally defines the syntax of a list aggregate. Note that - in contrast to general aggregate atoms, only equality comparisons, i.e. $X = \#list\{...\}$, are allowed, and we permit only a single element term rather than arbitrary tuples. Listing~\ref{lst:graph-list-aggregation-lstagg} shows a much shorter version of the program from Listing~\ref{lst:graph-list-aggregation}, in which list aggregation is used instead of step-by-step construction of list terms.
\begin{lstlisting}[style=asp-code, label={lst:graph-list-aggregation-lstagg}, caption={Creating vertex- and edge-lists using list aggregates.}]
%% pack vertices into a vertex list
vertex_lst(LIST) :- 
    LIST = #list{V : vertex(V)}.
    
%% pack edges into an edge list
edge_lst(LIST) :- 
    LIST = #list{edge(V1, V2) : edge(V1, V2)}.    
\end{lstlisting}

In order to compile list-aggregates into their semantically equivalent aggregate-free versions, the following handling of list aggregates has been added to Alpha's aggregate rewriting logic:
\begin{itemize}
    \item Substitute list aggregates in rules according to the general rules for aggregates outlined in~\ref{subsubsec:alpha-arch-aggregate-rewriting}.
    \item Element rules for list aggregates get constructed the same as for other aggregates, but permit only one element term rather than a tuple.
    \item In order to encode list construction, rules are added to construct a total order over all elements that should go into the list.
    \item List construction starts with the last element, which is defined to be the maximum element of the constructed order, successor elements are recursively added to the front of the list.
    \item The list is complete when a list has been constructed, such that there is no element smaller than the current head of the list.
\end{itemize}
Listing~\ref{lst:graph-list-aggregation-rewritten} shows the rewritten version of the program form Listing~\ref{lst:graph-list-aggregation-lstagg} (note that in the hand-crafted example from Listing~\ref{lst:graph-list-aggregation}, lists are constructed in descending element order, rather than ascending as is the case for the aggregate-based version).
\begin{lstlisting}[style=asp-code, label={lst:graph-list-aggregation-rewritten}, caption={Rewritten list aggregates from Listing~\ref{lst:graph-list-aggregation-lstagg}}]  
vertex_lst(LIST) :- 
    list_1_result(list_1_no_args, LIST).
edge_lst(LIST) :- 
    list_2_result(list_2_no_args, LIST).
list_1_element_greater(ARGS, N, K) :- 
    list_1_element(ARGS, N), 
    list_1_element(ARGS, K), 
    N > K.
list_1_element_not_successor(ARGS, N, K) :- 
    list_1_element_greater(ARGS, N, I), 
    list_1_element_greater(ARGS, I, K).
list_1_element_successor(ARGS, N, K) :- 
    list_1_element_greater(ARGS, N, K), 
    not list_1_element_not_successor(ARGS, N, K).
list_1_element_has_successor(ARGS, N) :- 
    list_1_element_successor(ARGS, _0, N).
list_1_lst_element(ARGS, IDX, lst(N, lst_empty)) :- 
    list_1_element(ARGS, N), 
    IDX = 0,
    not list_1_element_has_successor(ARGS, N).
list_1_lst_element(ARGS, IDX, lst(N, lst(K, TAIL))) :- 
    list_1_element(ARGS, N),
     list_1_element_successor(ARGS, K, N), 
     list_1_lst_element(ARGS, PREV_IDX, lst(K, TAIL)), 
     IDX = PREV_IDX + 1.
list_1_has_next_element(ARGS, IDX) :- 
    list_1_lst_element(ARGS, IDX, _1), 
    NEXT_IDX = IDX + 1, 
    list_1_lst_element(ARGS, NEXT_IDX, _2).
list_1_result(ARGS, LIST) :- 
    list_1_lst_element(ARGS, IDX, LIST), 
    not list_1_has_next_element(ARGS, IDX).
list_1_element(list_1_no_args, V) :- 
    vertex(V).
list_2_element_greater(ARGS, N, K) :- 
    list_2_element(ARGS, N), 
    list_2_element(ARGS, K), 
    N > K.
list_2_element_not_successor(ARGS, N, K) :- 
    list_2_element_greater(ARGS, N, I), 
    list_2_element_greater(ARGS, I, K).
list_2_element_successor(ARGS, N, K) :- 
    list_2_element_greater(ARGS, N, K), 
    not list_2_element_not_successor(ARGS, N, K).
list_2_element_has_successor(ARGS, N) :- 
    list_2_element_successor(ARGS, _3, N).
list_2_lst_element(ARGS, IDX, lst(N, lst_empty)) :- 
    list_2_element(ARGS, N), 
    IDX = 0, 
    not list_2_element_has_successor(ARGS, N).
list_2_lst_element(ARGS, IDX, lst(N, lst(K, TAIL))) :- 
    list_2_element(ARGS, N), 
    list_2_element_successor(ARGS, K, N), 
    list_2_lst_element(ARGS, PREV_IDX, lst(K, TAIL)), 
    IDX = PREV_IDX + 1.
list_2_has_next_element(ARGS, IDX) :- 
    list_2_lst_element(ARGS, IDX, _4), 
    NEXT_IDX = IDX + 1, 
    list_2_lst_element(ARGS, NEXT_IDX, _5).
list_2_result(ARGS, LIST) :- 
    list_2_lst_element(ARGS, IDX, LIST), 
    not list_2_has_next_element(ARGS, IDX).
list_2_element(list_2_no_args, edge(V1, V2)) :- 
    edge(V1, V2).   
\end{lstlisting}    

\subsubsection{Parsing Module Definitions}

Module definitions are parsed together with regular ASP code. Given a file containing a set of facts and rules and an arbitrary number of module definitions, Alpha's parser will group all rules and facts outside any module definition into one ASP program (i.e. the "main program"), and emit a set containing all encountered module definitions alongside with the parsed main program. Definition~\ref{def:module-definition-technical} details the technical representation of a module definition (which is basically a data structure storing all elements specified in the syntactical definition of a module, see~\ref{def:module-definition}). The actual code used in Alpha for parsing and storing module definitions can be found in the Appendix, see Example~\ref{ex:alpha-module-parsing}.

\begin{definition}[Internal module representation]
\label{def:module-definition-technical}
A parsed module definition gets stored as a tuple $M = (\mathit{name},\mathit{p_{in}},\mathit{OUT},\mathit{PROG})$ where
\begin{itemize}
    \item $\mathit{name}$ is a string representing the name of the module. This must be unique, i.e. Alpha currently has no concept of namespaces or similar means of distinguishing equally named modules,
    \item $\mathit{p_{in}}$ is a predicate (represented as a string of form $\mathit{name}/\mathit{arity}$) called \emph{input specification},
    \item $\mathit{OUT}$ denotes a (possibly empty) set of predicates called \emph{output specification},
    \item and $\mathit{PROG}$ is the ASP program holding the actual implementation code for the module.
\end{itemize}
\end{definition}   

\subsubsection{Module Literal Compilation}

Once a program and all contained module definitions have been parsed, every module atom (i.e. call to a module from a rule body) needs to be linked to its implementation. In Alpha, external atoms are evaluated during grounding. Definition~\ref{def:external-atom-interpretation} formally defines the notion of an \emph{external predicate interpretation}, i.e. an interpretation function for ground instances of external atoms. 

\begin{definition}[External Predicate Interpretation]
\label{def:external-atom-interpretation}    
 Given an external atom \\ $\&\mathit{ext}[i_1,\ldots,i_n](o_1,\ldots,o_m)$ with input terms $i_1,\ldots,i_n$ and output terms $o_1,\ldots,o_m$, and a substitution $\sigma$ assigning ground values to at least all input terms of $ex$, the \emph{predicate interpretation function} $f: HU^n \mapsto 2^{HU^m}$ is a function which, for a given $n$-tuple over the Herbrand Universe $HU$, returns all $m$-tuples over $HU$ for which the ground atom $\&\mathit{ext}[\sigma(i_1),\ldots,\sigma(i_n)](r_1,\ldots,r_m)$ where $(r_1\ldots,r_m) \in f(\sigma(i_1),\ldots,\sigma(i_n))$ is true.
\end{definition}

Implementation-wise, predicate interpretations are Java methods with a specific signature. Listing~\ref{lst:predicate-interpretation-interface} shows the corrseponding interface definition. Note that, for an implementation of \texttt{PredicateInterpretation} to be valid, it must be pure functional, i.e. not have any side-effects.
\begin{lstlisting}[style=java, label={lst:predicate-interpretation-interface}, caption={Java Interface for Predicate Interpretations}]
@FunctionalInterface
public interface PredicateInterpretation {
    
    Set<List<Term>> evaluate(List<Term> terms);

}    
\end{lstlisting}

\paragraph{Constructing interpretations for Module Atoms}
Listing~\ref{lst:module-interpretation-impl} shows the (abbreviated) Java code used in Alpha to construct a \texttt{PredicateInterpretation} for a module atom. Boundary checks and similar validations as well as auxiliary method calls are left out for easier readability. The interpretation itself is constructed using the Lambda-expression \texttt{PredicateInterpretation interpretation = terms -> \{...\}}, where \texttt{terms} is the list of input terms, which gets wrapped into an atom of the predicate defined in \texttt{inputSpec} (i.e. the module input specification). Answer sets get filtered based on predicate names according to the module's output specification. Finally, each answer set is converted to a list of list terms using function \texttt{answerSetToTerms}, and the resulting set of term lists is returned.
\begin{lstlisting}[style=java, label={lst:module-interpretation-impl}, caption={Constructing module interpretations}]
private ExternalAtom translateModuleAtom(ModuleAtom atom, Map<String, Module> moduleTable) {
    ...
    Predicate inputSpec = definition.getInputSpec();
    ...
    Set<Predicate> outputSpec = definition.getOutputSpec();
    Set<Predicate> expectedOutputPredicates;
    if (outputSpec.isEmpty()) {
    	expectedOutputPredicates = calculateOutputPredicates(normalizedImplementation);
    } else {
    	expectedOutputPredicates = outputSpec;
    }
    ...
    PredicateInterpretation interpretation = terms -> {
    	BasicAtom inputAtom = Atoms.newBasicAtom(inputSpec, terms);
    	NormalProgram program = Programs.newNormalProgram(
            normalizedImplementation.getRules(),
    		ListUtils.union(List.of(inputAtom), 
            normalizedImplementation.getFacts()), 
            normalizedImplementation.getInlineDirectives(), 
            Collections.emptyList());
    	java.util.function.Predicate<Predicate> filter = outputSpec.isEmpty() ? p -> true : outputSpec::contains;
    	Stream<AnswerSet> answerSets = moduleRunner.solve(program, filter);
    	if (atom.getInstantiationMode()
            .requestedAnswerSets().isPresent()) {
    		answerSets = answerSets.limit(
                atom.getInstantiationMode()
                    .requestedAnswerSets().get());
    	}
    	return answerSets.map(as -> answerSetToTerms(as, expectedOutputPredicates))
            .collect(Collectors.toSet());
    };
    return Atoms.newExternalAtom(atom.getPredicate(), interpretation, atom.getInput(), atom.getOutput());
}

private static List<Term> answerSetToTerms(AnswerSet answerSet, Set<Predicate> moduleOutputSpec) {
    List<Term> terms = new ArrayList<>();
    for (Predicate predicate : moduleOutputSpec) {
        if (!answerSet.getPredicates().contains(predicate)) {
            terms.add(Terms.EMPTY_LIST);
        } else {
            terms.add(Terms.asListTerm(
                    answerSet.getPredicateInstances(predicate).stream()
                        .map(Atoms::toFunctionTerm)
                        .collect(Collectors.toList())));
        }
    }
    return terms;
}
\end{lstlisting}    

\chapter{Results}
\label{chap:results}
In this chapter, we apply Alpha's Evolog extension to a larger example. In order to showcase the kind of applications that can be written in pure Evolog (without the need to resort to any additional programming language), we implement an application that:
\begin{itemize}
    \item asks the user to enter a path to a file containing a graph represented in XML format,
    \item reads and parses the XML content of the given files,
    \item calculates 3-colorings of the graph from the XML file,
    \item and finally writes the list of obtained colorings to a user-supplied file path.
\end{itemize}    

We first describe the program itself and then move on to a discussion of observations that can be made from the example.

\section{Interactive XML-based Graph Coloring}
\label{sec:results-xml-graphcol}

TODO: Reference full program in appendix. Maybe lose a few words about overall structure again.

\subsection{Reading files from user-supplied paths}

First, the user is asked to enter a file path. The content of the given file is then read line-by-line and aggregateed into a list term. Listing~\ref{lst:results-xml-graphcol-userinput} shows this part of the program.

\begin{lstlisting}[style=asp-code, label={lst:results-xml-graphcol-userinput}, caption={Reading a file based on user input.}] 
enter_input_prompt("Please enter a path to read graphs from: ").

write_input_prompt_res(R) : @streamWrite[STDOUT, PROMPT] = R :- 
    enter_input_prompt(PROMPT), &stdout(STDOUT).
usr_input_res(INP) : @streamReadLine[STDIN] = INP :- 
    write_input_prompt_res(success(_)), &stdin(STDIN).

infile(PATH) :- usr_input_res(success(line(PATH))).
infile_open(PATH, HD) : @fileInputStream[PATH] = HD :- 
    infile(PATH).

% Handle file opening error
error(io, MSG) :- infile_open(_, error(MSG)).

% Read all lines from infile 
readline_result(PATH, 0, RES) : 
    @streamReadLine[STREAM] = RES :- 
    infile(PATH), infile_open(PATH, success(stream(STREAM))).
readline_result(PATH, LINE_NO, RES) : 
    @streamReadLine[STREAM] = RES :- 
    infile(PATH), 
    infile_open(PATH, success(stream(STREAM))), 
    readline_result(PATH, PREV_LINE_NO, PREV_LINE_RES), 
    PREV_LINE_RES != success(line(eof)), 
    LINE_NO = PREV_LINE_NO + 1.

% close stream after getting eof
infile_closed(PATH, RES) : 
    @inputStreamClose[STREAM] = RES :- 
    infile(PATH), 
    infile_open(PATH, success(stream(STREAM))),
    readline_result(PATH, _, ok(eof)).

% Now create a list of content lines
file_lines(PATH, LST) :- 
    LST = #list{ 
        line(LINE_NO, LINE) : 
            readline_result(PATH, LINE_NO, success(line(LINE))), 
            LINE != eof },
    infile(PATH). 
\end{lstlisting}    

The rule on line 3 derives the predicate \texttt{write\_input\_prompt\_res/1}, which wraps the result of a stream output operation - which in this case writes a prompt text to stdout. Note that the \emph{standard output stream} \texttt{STDOUT} does not need to be opened using an Evolog action, but is always available to an application. A reference to \texttt{STDOUT} can be obtained through the external atom \texttt{stdout/1}.

If writing the prompt text succeeds, i.e. in case the result term of the write action has structure \texttt{success(\_)}, user input is read from the \emph{standard input stream} \texttt{STDIN}. The result of the read operation is derived on line 5 using  predicate \texttt{usr\_input\_res/1}. Note that, as with stdout, \texttt{STDIN} does not have to be opened, but is obtained using the external atom \texttt{stdin/1}.

In case the read operation succeeded, we attempt to open an input stream on the file entered by the user. The rule on line 9 derives the result of this operation in the \texttt{infile\_open/2} predicate. The first term is the path of the respective file, the second is an action result term which either takes the form \texttt{error(MSG)} or holds a reference to a file input stream \texttt{STREAM} wrapped in a function term \texttt{success(stream(STREAM))}.

If opening succeeds, the openend file is read line-wise, using the rules on lines 16 and 19. Lines are numbered in order to keep positional information of file content. Reading stops once the result from the last \texttt{@streamReadLine} action is \texttt{success(line(eof))} where \texttt{eof} denotes \emph{end-of-file}.

\subsection{Parsing XML data}
\label{subsec:results-xml-parsing}

The interactive graph coloring program reads XML files which describe graphs. Input files are expected to conform to the \gls{dtd} in Listing\ref{lst:results-xml-graphcol-input}.

\begin{lstlisting}[style=asp-code, label={lst:results-xml-graphcol-input}, caption={\gls{dtd} for graph XML files.}]  
<!ELEMENT graph (vertices, edges)>
<!ATTLIST graph directed (true | false) #REQUIRED>

<!ELEMENT vertices (vertex+)>
<!ELEMENT vertex (#PCDATA)>

<!ELEMENT edges (edge+)>
<!ELEMENT edge (source, target)>
<!ELEMENT source (#PCDATA)>
<!ELEMENT target (#PCDATA)>    
\end{lstlisting}    

Listing~\ref{lst:results-xml-graphcol-input-ex} shows an example input file describing the complete graph $K_3$.

\begin{lstlisting}[style=asp-code, label={lst:results-xml-graphcol-input}, caption={The complete graph $K_3$ represented according to the \gls{dtd} from Listing~\ref{lst:results-xml-graphcol-input}.}]
<graph directed="false">
    <vertices>
        <vertex>a</vertex>
        <vertex>b</vertex>
        <vertex>c</vertex>
    </vertices>
    <edges>
        <edge>
            <source>a</source>
            <target>b</target>
        </edge>
        <edge>
            <source>b</source>
            <target>c</target>
        </edge>
        <edge>
            <source>c</source>
            <target>a</target>
        </edge>
    </edges>    
</graph>
\end{lstlisting} 

\chapter{Discussion}
\label{chap:discussion}
In this chapter, we discuss observations from the expirements conducted in Chapter~\ref{chap:application-experiment}. It is split into two main parts: In the first part, we discuss recurring patterns in code written in Evolog as observed on the model application written in Chapter~\ref{chap:application-experiment}. In the second part of this chapter, we take a look at the performance metrics gained from the testing described in Section~\ref{sec:results-performance-tests}.

\section{Observations from writing Evolog code}
\label{sec:discussion-language-improvements}

Considering the example application discussed in Section~\ref{sec:results-xml-graphcol}, this section aims to highlight potential areas for improvement of the language.

\subsection{Unwrapping list terms}
As the examples in Section~\ref{sec:results-xml-graphcol} clearly show, there is a certain amount of boilerplate code involved whenever one has to "unpack" a list term obtained as an output term of a module literal. Since in most cases where list terms come into play, one actually wants to work on individual values on the list, we end up with many iterations of similar-looking code that is needed to transform one atom with a single list term into a set of "element atoms", each referring to one element of the list. An example of this is the subset of the 3-Coloring module (see Example~\ref{ex:3col-module}) where the module input (an atom referencing a list of vertices and a list of edges) is deconstructed in order to represent a graph using instances of predicates \texttt{vertex/1} and \texttt{edge/2}. The relevant part of the program is shown in Listing~\ref{lst:discussion-graph-unwrap}.

\begin{lstlisting}[style=asp-code, label={lst:discussion-graph-unwrap}, caption={Unwrapping list terms representing a graph}]
	vertex_element(V, TAIL) :- graph(lst(V, TAIL), _).
	vertex_element(V, TAIL) :- vertex_element(_, lst(V, TAIL)).
	vertex(V) :- vertex_element(V, _).
	edge_element(E, TAIL) :- graph(_, lst(E, TAIL)).
	edge_element(E, TAIL) :- edge_element(_, lst(E, TAIL)).
	edge(V1, V2) :- edge_element(edge(V1, V2), _).
\end{lstlisting}

The six lines of code shown in Listing~\ref{lst:discussion-graph-unwrap} deal with deconstructing two list terms, a vertex list and and edge list, respectively, into individual atoms such that each atom corresponds to one element of the list.
The transformation always follows the same pattern:
\begin{itemize}
    \item One rule derives an instance of an "element atom" from the head of the list, which is obtained by deconstructing the list term into the form \texttt{lst(V, TAIL)}, where \texttt{TAIL} is the rest of the list without the head value represented as variable \texttt{V}.
    \item A second rule recursively derives further list elements from element atoms by further deconstructing the \texttt{TAIL} terms of the individual element atoms. Repeated evaluation of this rule eventually leads to one element atom for every element of the original list term.
    \item Finally, a third rule derives the actual domain atoms which are of interest by projecting away the partial lists from the individual element atoms.
\end{itemize}    

Example~\ref{ex:discussion-graph-unwrapping} demonstrates this method of unwrapping list terms based on a list representation of the complete graph with 3 vertices $K_3$.

\begin{example}[List Unwrapping]
\label{ex:discussion-graph-unwrapping} 
Listing~\ref{lst:discussion-graph-unwrapping-example} shows the "graph unwrapping" program from Listing~\ref{lst:discussion-graph-unwrap} including a fact representing the complete graph $K_3$ encoded as two list terms.
\begin{lstlisting}[style=asp-code, label={lst:discussion-graph-unwrapping-example}, caption={Unwrapping the list representation of $K_3$.}]
	graph(
		lst(a, lst(b, lst(c, lst_empty))),
		lst(edge(a, b), lst(edge(b, c), lst(edge(c, a), lst_empty)))).
	
	vertex_element(V, TAIL) :- graph(lst(V, TAIL), _).
	vertex_element(V, TAIL) :- vertex_element(_, lst(V, TAIL)).
	vertex(V) :- vertex_element(V, _).
	edge_element(E, TAIL) :- graph(_, lst(E, TAIL)).
	edge_element(E, TAIL) :- edge_element(_, lst(E, TAIL)).
	edge(V1, V2) :- edge_element(edge(V1, V2), _).
\end{lstlisting}
Listing ~\ref{lst:discussion-graph-unwrapped-example} shows the (single) answer set of the program given in Listing~\ref{lst:discussion-graph-unwrap}
\begin{lstlisting}[style=asp-code, label={lst:discussion-graph-unwrapped-example}, caption={The unwrapped list representation of $K_3$.}]
Answer set 1:
	{ edge(a, b)
	edge(b, c)
	edge(c, a)
	edge_element(edge(a, b), lst(edge(b, c), lst(edge(c, a), lst_empty)))
	edge_element(edge(b, c), lst(edge(c, a), lst_empty))
	edge_element(edge(c, a), lst_empty)
	graph(
		lst(a, lst(b, lst(c, lst_empty))), 
		lst(edge(a, b), lst(edge(b, c), lst(edge(c, a), lst_empty))))
	vertex(a)
	vertex(b)
	vertex(c)
	vertex_element(a, lst(b, lst(c, lst_empty)))
	vertex_element(b, lst(c, lst_empty))
	vertex_element(c, lst_empty) }
\end{lstlisting}
\end{example}    

\paragraph{Generalizing list-unrwapping}
As Example~\ref{ex:discussion-graph-unwrapping} demonstrates, unwrapping a list term always takes 3 rules that only differ in predicate names. Definition~\ref{def:discussion-list-extraction} introduces a potential "syntacic sugar" solution we call an \emph{unwrap rule}.

\begin{definition}[List-Unwrapping-Rule]
\label{def:discussion-list-extraction}
Given an Atom $a$ with terms $t_1,\ldots,t_k,\ldots,t_n$, where $t_k$ is a list term, we define the following rule to be the \emph{unwrap rule for $t_k$}:
\[
	e(V)~\leftarrow~\#unwrap\{ V~|~t_k \},~a(t_1,\ldots,t_k,\ldots,t_n).
\]
In the unwrap rule, $e(V)$ is called an \emph{element atom}, where $V$ refers to one value from the list $t_k$.
We define the semantics of the above rule to be equivalent to the following program:
\begin{align*}
	\mathit{list\_element}(V,\mathit{TAIL})~&\leftarrow~a(t_1,\ldots,\mathit{lst}(V,\mathit{TAIL}),\ldots,t_n).\\
	\mathit{list\_element}(V,\mathit{TAIL})~&\leftarrow~\mathit{list\_element}(\_,\mathit{lst}(V,\mathit{TAIL})).\\
	e(V)~&\leftarrow~\mathit{list\_element}(V,\_).
\end{align*}	
\end{definition}

Listing~\ref{lst:discussion-sugared-unwrap} shows a potential implementation of the 3-Coloring Module from Example~\ref{ex:3col-module} using the shorthand notation for list unwrapping proposed in Definition~\ref{def:discussion-list-extraction}.

\begin{lstlisting}[style=asp-code, label={lst:discussion-sugared-unwrap}, caption={3-Coloring Module with shorthand list unwrapping rules.}]
#module threecol(graph/2 => {col/2})  {
	% Unwrap input
	vertex(V) :- #unwrap{V | VLST}, graph(VLST, _).
	edge(V1, V2) :- #unwrap{edge(V1, V2) | ELST}, graph(_, ELST).

	% Make sure edges are symmetric
	edge(V2, V1) :- edge(V1, V2).

	% Guess colors
	red(V) :- vertex(V), not green(V), not blue(V).
	green(V) :- vertex(V), not red(V), not blue(V).
	blue(V) :- vertex(V), not red(V), not green(V).

	% Filter invalid guesses
	:- vertex(V1), vertex(V2), edge(V1, V2), red(V1), red(V2).
	:- vertex(V1), vertex(V2), edge(V1, V2), green(V1), green(V2).
	:- vertex(V1), vertex(V2), edge(V1, V2), blue(V1), blue(V2).

	col(V, red) :- red(V).
	col(V, blue) :- blue(V).
	col(V, green) :- green(V).
}
\end{lstlisting}	

\subsection{Repetitive File-IO Actions}

As we can observe in Listings~\ref{lst:results-xml-graphcol-userinput} and~\ref{lst:results-xml-graphcol-write-output}, respectively, reading the content of a file, as well as writing to a file require a set of rules that looks the same regardless of the specific file or content in question. Listing~\ref{lst:discussion-fileio-readlines} shows the "standard program" to read all lines from a file:
\begin{itemize}
	\item A file path (i.e. location in storage), represented using the fact \texttt{infile(PATH)} is opened.
	\item Assuming opening the file produced no error, the first line of content is read using the rule on line 11.
	\item Based on the first line read (line number 0), we recursively continue reading as long as reading the last line was successful and did not yield \texttt{eof} (i.e. a pseudo-term denoting the end of content) as a result.
	\item If \texttt{eof} has been read, the file is closed, and lines are represented as atoms of form \texttt{line(LINE\_NO, LINE)}.
\end{itemize}	

\begin{lstlisting}[style=asp-code, label={lst:discussion-fileio-readlines}, caption={Reading all lines from a file.}]
infile_open(PATH, HD) : @fileInputStream[PATH] = HD :- 
	infile(PATH).

% Handle file opening error
error(io, MSG) :- infile_open(_, error(MSG)).

% Read all lines from infile 
readline_result(PATH, 0, RES) : 
	@streamReadLine[STREAM] = RES :- 
	infile(PATH), infile_open(PATH, success(stream(STREAM))).
readline_result(PATH, LINE_NO, RES) : 
	@streamReadLine[STREAM] = RES :- 
	infile(PATH), 
	infile_open(PATH, success(stream(STREAM))), 
	readline_result(PATH, PREV_LINE_NO, PREV_LINE_RES), 
	PREV_LINE_RES != success(line(eof)), 
	LINE_NO = PREV_LINE_NO + 1.

% close stream after getting eof
infile_closed(PATH, RES) : 
	@inputStreamClose[STREAM] = RES :- 
	infile(PATH), 
	infile_open(PATH, success(stream(STREAM))),
	readline_result(PATH, _, ok(eof)).

% Extract actual lines and numbers
line(LINE_NO, LINE) :- 
	readline_result(PATH, LINE_NO, success(line(LINE))), 
	LINE != eof.
\end{lstlisting}	

The code to write a set of lines to a file looks similar to how a file is read. Listing~\ref{lst:discussion-fileio-writelines} demonstrates the typical implementation. The following steps are needed:
\begin{itemize}
	\item First, the file to which content is to be written is opened.
	\item Assuming lines to be written are consecutively numbered and available as atoms of form \texttt{line(LINE\_NO, LINE)}, the first line (assumed to have an index of 0) is written.
	\item If the first line has been successfully written, all other lines are written using a recursive rule.
	\item We derive that all lines have been written when the line with maximum index has been written.
	\item Once \texttt{all\_lines\_written} has been derived, the file is closed.
	\item The file is also closed in case an error occurs during writing.
\end{itemize}	

\begin{lstlisting}[style=asp-code, label={lst:discussion-fileio-writelines}, caption={Writing lines to a file.}]
% Open the output file
open_result(PATH, RES) : @fileOutputStream[PATH] = RES :- outfile(PATH).

% Write lines in order of ascending line number
write_result(0, RES) : @streamWrite[STREAM, LINE] = RES :- open_result(PATH, success(stream(STREAM))), outfile(PATH), line(0,LINE).
write_result(LINE_NO, RES) : @streamWrite[STREAM, LINE] = RES :-  
	open_result(PATH, success(stream(STREAM))), 
	write_result(LINE_NO - 1, success(ok)), 
	outfile(PATH), line(LINE_NO, LINE).
all_lines_written :- write_result(LINE_NO, success(ok)), LINE_NO = #max{NUM : line(NUM, _)}.
	
% The file should be closed once all lines were successfully written	
should_close(PATH, STREAM) :- all_lines_written, open_result(PATH, success(stream(STREAM))), outfile(PATH).

% The file should also be closed when an IO error occurs during writing
should_close(PATH, STREAM) :- 
	write_result(_, error(_)),
	open_result(PATH, success(stream(STREAM))), outfile(PATH).

% Close the file
close_result(PATH, RES) : @outputStreamClose[STREAM] = RES :- 
	should_close(PATH, STREAM), 
	open_result(PATH, success(stream(STREAM))), 
	outfile(PATH).
\end{lstlisting}	

\paragraph{Module-based abstraction of actions}
As we've demonstrated in Listings~\ref{lst:discussion-fileio-readlines} and~\ref{lst:discussion-fileio-writelines}, frequently used file operations like reading all lines from and writing a set of lines to a file, form repetitive code patterns that would greatly benefit from some shortened abstraction. The obvious solutions seems to be to simply permit actions in Evolog Modules. However, this approach presents some challenges with regards to transparent semantics.

Recalling the demand for \emph{action transparency} (see Definition~\ref{def:evolog-actions-transparency}), i.e. that the result of every firing (ground) action rule must be part of an answer set, modules containing actions can not simple be used in module literals as they're described in Section~\ref{sec:evolog-modules}. Instead, we'd have to apply the definitions of action functions in rule heads to an Evolog Module and have a module function as an action interpretation function. Example~\ref{ex:discussion-fileio-actmodules} demonstrates this approach.

\begin{example}[Action Modules]
\label{ex:discussion-fileio-actmodules}	
In this example, we show a potential implementation for a module that takes a list of numbered strings and a string describing a file location as input, and writes all lines to the referenced file. Listing~\ref{lst:discussion-fileio-actmodule-impl} gives an example of how a module definition for a module containing actions might look.
\begin{lstlisting}[style=asp-code, label={lst:discussion-fileio-actmodule-impl}, caption={Prototypical definition of a module containing Evolog actions.}]
#action write_lines(lines/1, path/1 => (success: lines_written/1 | error: io_error/1)) {
	% Unwrap line list
	line(IDX, LINE) :- #unwrap{line(IDX, LINE) | LST}, lines(LST).
	% Unwrap 1-element-list path/1
	outfile(PATH) :- #unwrap{P | PLST}, path(PLST).
	
	% Open the output file
	open_result(PATH, RES) : @fileOutputStream[PATH] = RES :- outfile(PATH).

	% Write lines in order of ascending line number
	write_result(0, RES) : @streamWrite[STREAM, LINE] = RES :- open_result(PATH, success(stream(STREAM))), outfile(PATH), line(0, LINE).
	write_result(LINE_NO, RES) : @streamWrite[STREAM, LINE] = RES :-  
		open_result(PATH, success(stream(STREAM))), 
		write_result(LINE_NO - 1, success(ok)), 
		outfile(PATH), line(LINE_NO, LINE).
	all_lines_written :- write_result(LINE_NO, success(ok)), LINE_NO = #max{NUM : line(NUM, _)}.
		
	% The file should be closed once all lines were successfully written	
	should_close(PATH, STREAM) :- all_lines_written, open_result(PATH, success(stream(STREAM))), outfile(PATH).

	% The file should also be closed when an IO error occurs during writing
	should_close(PATH, STREAM) :- 
		write_result(_, error(_)),
		open_result(PATH, success(stream(STREAM))), outfile(PATH).

	% Close the file
	close_result(PATH, RES) : @outputStreamClose[STREAM] = RES :- 
		should_close(PATH, STREAM), 
		open_result(PATH, success(stream(STREAM))), 
		outfile(PATH).

	io_error("Failed opening file: " + MSG) :-
		open_result(_, error(MSG)).
	io_error("Failed writing line " + IDX + ": " + MSG) :-
		write_result(IDX, error(MSG)).
	io_error("Failed closing file: " + MSG) :-
		close_result(_, error(MSG)).

	lines_written(CNT) :- CNT = #count{N, R : write_result(N, R)}, not io_error(_).		
}	
\end{lstlisting}	
The action module proposed in Listing~\ref{lst:discussion-fileio-actmodule-impl} differs from a "regular" module in the following ways:
\begin{itemize}
	\item The module output definition \texttt{(success: lines\_written/1 | error: io\_error/1)} distinguishes between a success and an error case. 
	\item Since we assume action result terms to always be unary function terms with symbol either $\mathit{success}$ or $\mathit{error}$, the atoms of the given predicates would then be arguments of the respective (success or error) term in their termified form according to Definition~\ref{def:module-output-translation}.
	\item The above items imply that an action module result can only be success \emph{or} error, but not both. In an actual implementation of this concept, ways of enforcing this rule would need to be adressed. 
\end{itemize}
Listing~\ref{lst:discussion-fileio-actmodule-usage} shows a prototypical example of how an action module could be used.
\begin{lstlisting}[style=asp-code, label={lst:discussion-fileio-actmodule-usage}, caption={Prototypical usage of an action module providing an abstraction to write content to a file.}]
line(0, "lorem ipsum dolor").
line(1, "sit amet consectetur").
line(2, "adipiscing  elit")

write_result(R) : @write_lines[LINES, "/tmp/out.txt"] = R 
	:- LINES = #list{line(I, L) : line(I, L)}.
\end{lstlisting}		
\end{example}

While the approach to writing composite actions using a special form of modules sketched in Example~\ref{ex:discussion-fileio-actmodules} looks promising at first glance, we note that it also, potentially, projects away information about side-effects: Consider two distinct Frames $F_1$ and $F_2$, where in $F_1$ the \texttt{stream\_write} action in module \texttt{write\_lines} gives an error for line 1, and in $F_2$, the same action yields an error for line 2. In both frames, the result of the action rule in the top-level program is going to be an error, specifically \texttt{write\_result(io\_error("Failed writing line 1: error"))} for $F_1$, and \texttt{write\_result(io\_error("Failed writing line 2: error"))} (note that wrapping list terms mandated by module output translation were ignored since there is only one answer set holding one instance of \texttt{io\_error} anyway). While in this case the line on which writing failed is made transparent through the error message, in general, it would be possible to project away the information how many lines were written, i.e. two frames might yield different outcomes for the actions inside the module while yielding the exact same result for the overall program. Potential implementations of this concept would have to make allowances for this.


\section{Potential improvements to solving performance}

The performance test results from Section~\ref{subsec:results-performance-numbers} show a strong increase in runtime of the XML-3-Coloring application with increasing vertex count of the input graph. Intuitively, this seems easily explained - the XML-parsing module described in Section~\ref{sec:appendix-xml-dom-parsing} has a number of predicates whose (ground) instance count in an answer set is $O(n^2)$ with regards to the number $n$ of instances of the predicates they depend on. A typical example are rules like in Listing~\ref{lst:results-xml-parse-tag-pairs} where, in order to find pairs of corresponding opening and closing XML tags, we construct a partial ordering of tokens, resulting in $\frac{n^2 - n}{2} = \binom{n}{2}$ atoms for the relation predicate. Similar constructs occur multiple times in the XML parsing module. While information about the position of tokens in the input text relative to each other is necessary for the program, it would not have to be kept in memory at all times - the program in the XML parsing module is stratified, and the output predicates which are actually reported back to the calling program all reside in the uppermost stratum.

Based on the observations above, we could reasonably adapt Alpha's stratified evaluation algorithm (see Section~\ref{subsubsec:impl-stratified-eval}) to discard predicates from working memory that are no longer needed for evaluation (i.e. all instances of all directly dependent predicates have been calculated) and are not part of a module's output predicates, thereby reducing memory footprint and in turn reduce the amount of time spent on (fruitless) garbage collection cycles in the JVM running Alpha.

\chapter{Conclusion}
\label{chap:conclusion}
TODO: related work\\
TODO: discussion of contribuitions and results,\\
TODO: turning into future work, what Evolog would need to get from prototype to product.

\appendix
\chapter{Additional Material}

Some more samples that would be too much inline. \todo{do proper text}

\section{Installing Alpha}

TODO: Brief how-to to install an Alpha build made for this thesis and run the examples (Windoze/Linux).

\section{Running ASP code with Alpha}

TODO: Debugging features of CLI application, basic API usage.

\section{Examples}
\begin{example}[Fibonacci-Numbers using external atoms]
\label{ex:user-supplied-externals}
The following code snippet demonstrates how to run a program which includes user-supplied external atoms using Alpha. Since the Alpha Commandline-App currently does not support loading Atom Definitions from jar files, the solver is directly called from Java code in this example.\\
\\
In this example, we use an external predicate definition \texttt{fibonacci\_number/2} to efficiently calculate Fibonacci numbers. The actual ASP program generates all Fibonacci numbers up to index 40 that are even. The ASP program is shown in Listing~\ref{lst:user-supplied-externals-asp}, while Listing~\ref{lst:user-supplied-externals-java} shows the Java implementation of the external predicate.
\begin{lstlisting}[style=asp-code, label={lst:user-supplied-externals-asp}, caption={ASP program to find even Fibonacci numbers.}]
%% Find even fibonacci numbers up to F(40).
fib(N, FN) :- &fibonacci_number[N](FN), N = 0..40.
even_fib(N, FN) :- fib(N, FN), FN \ 2 = 0.    
\end{lstlisting}    
The Java implementation of the \texttt{fibonacci\_number/2} predicate makes use of \emph{Binet's Formula}\todo{can we cite something here?} to efficiently compute Fibonacci numbers using a closed form expression of the sequence.
\begin{lstlisting}[style=java, label={lst:user-supplied-externals-java}, caption={Fibonacci number computation in Java.}]
public class CustomExternals {

	private static final double GOLDEN_RATIO = 1.618033988749894;
	private static final double SQRT_5 = Math.sqrt(5.0);

	/**
	 * Calculates the n-th Fibonacci number using a variant of Binet's Formula
	 */
	$$@Predicate(name="fibonacci_number")
	public static Set<List<ConstantTerm<Integer>>> fibonacciNumber(int n) {
		return Set.of(List.of(Terms.newConstant(binetRounding(n))));
	}

	public static int binetRounding(int n) {
		return (int) Math.round(Math.pow(GOLDEN_RATIO, n) / SQRT_5);
	}

}    
\end{lstlisting}    
The Alpha solver is invoked from a simple Java method, which uses Alpha's API to first compile the external atom definition, and then run the solving process for the parsed ASP program. This is illustrated in Listing~\ref{lst:user-supplied-externals-main}
\begin{lstlisting}[style=java, label={lst:user-supplied-externals-main}]
public class CustomExternalsApp {

	public static void main(String[] args) throws IOException {
		Alpha alpha = AlphaFactory.newAlpha();
		Map<String, PredicateInterpretation> customExternals =
			Externals.scan(CustomExternals.class);
		String aspCode =
			Files.readString(
				Paths.get("src/main/resources/customExternals.asp"));
				InputProgram program =
					alpha.readProgramString(aspCode,
				customExternals);
		alpha.solve(program).forEach(as -> {
			System.out.println("Answer set:\n" + as);
		});
	}

}
\end{lstlisting}
\end{example}

\begin{example}[Module Parsing]
\label{ex:alpha-module-parsing}	
Listing~\ref{lst:apdx-alpha-module-grammar} shows the actual ANTLR-grammar used in Alpha to parse module definitions. Parsed Module Definitions are stored in instances of the \texttt{Module} type. Listing~\ref{lst:apdx-alpha-module-def} shows the corresponding Java interface definition.
\begin{lstlisting}[style=asp-code, label={lst:apdx-alpha-module-grammar}, caption={ANTLR grammar for module definitions}]
directive_module: 
	SHARP DIRECTIVE_MODULE id 
		PAREN_OPEN module_signature PAREN_CLOSE 
			CURLY_OPEN statements CURLY_CLOSE;

module_signature : 
	predicate_spec ARROW CURLY_OPEN 
		('*' | predicate_specs) CURLY_CLOSE;	

predicate_specs: 
	predicate_spec (COMMA predicate_specs)?;

predicate_spec: id '/' NUMBER;
\end{lstlisting}
Note how the \texttt{module\_signature} rule permits the shorthand \texttt{*} instead of an output predicate list.
This is used as a shorthand to express the list of all predicates derived by any rule in the module (i.e. emit full, unfiltered answer sets).	
\begin{lstlisting}[style=java, label={lst:apdx-alpha-module-def}, caption={Java interface specifying Alpha's internal representation of a module definition.}]
public interface Module {

	String getName();

	Predicate getInputSpec();

	Set<Predicate> getOutputSpec();

	InputProgram getImplementation();

}	
\end{lstlisting}	
\end{example}	


\backmatter

% Use an optional list of figures.
\listoffigures % Starred version, i.e., \listoffigures*, removes the toc entry.

% Use an optional list of tables.
\cleardoublepage % Start list of tables on the next empty right hand page.
\listoftables % Starred version, i.e., \listoftables*, removes the toc entry.

% Use an optional list of alogrithms.
\listofalgorithms
\addcontentsline{toc}{chapter}{List of Algorithms}

% Add an index.
\printindex

% Add a glossary.
\printglossaries

% Add a bibliography.
\bibliographystyle{alpha}
\bibliography{evolog}

\end{document}