This thesis deals with extending an existing solver for \gls{asp} programs, namely \emph{Alpha}~\cite{alpha}, with facilities to both execute \emph{actions} from within \gls{asp} programs, as well as to re-use program parts by means of a modularization system. Answer Set Programming is a logic programing formalism that goes back to the seminal work of Gelfond and Lifschitz in 1988~\cite{stable-models}. Through its declarative semantics and concise syntax, it enables short and expressive programs to describe even NP-hard problems. Since its inception, \gls{asp} has been used with great success in diverse domains~\cite{survey-2016} such as logistics~\cite{gioia-tauro}~\cite{train-scheduling}, automated music composition~\cite{blues-composition} and spaceflight~\cite{space-shuttle} as well as in various industry settings~\cite{industrial-asp}. While \gls{asp} has enjoyed great success as a domain-specific language in the sense that it can express and efficiently solve complex reasoning problems, most applications require some kind of translation code written in a general-purpose programming language such as Python or Java to transform real-world input data into facts of an \gls{asp} program. Similary, the answer sets an \gls{asp} solver outputs need to be translated back into the data exchange format of the problem domain, e.g. HTTP responses, XML files, etc. The aim of this work is to provide an extension to \gls{asp} that enables the development of complete applications entirely within the extended \gls{asp}, eliminating the need to use additional programming languages or write translation code. The vision is to, for example, write an application that receives XML encoded input, solves an NP-hard reasoning problem based on that input, and writes its results in an XML-based encoding, all in \gls{asp}. This chapter gives an intuitive introduction to \gls{asp} through examples, examines the state of the art on \gls{asp} extensions dealing with actions and side-effects, as well as program modularization, and finally formulates the specific goals for this thesis.

\section{Answer Set Programming}

At its core, an \gls{asp} program is a collection of conditional rules along the lines of \emph{"if A holds true, then B must also hold"} as well as negative rules, so-called \emph{constraints}, which prohibit certain conditions, e.g. \emph{"If A has property p, then it cannot have property q"}. Example \ref{ex:asp-first-intro} demonstrates the basic idea of \gls{asp} based on a program for course planning in a (very simplified) school setting.
\begin{example}
\label{ex:asp-first-intro}
Listing \ref{lst:school-planning} shows a knowledge base written in \gls{asp}, which encodes facts - things that are always true, such as \emph{"There is a subject called maths"} - and rules, e.g. \emph{A teacher T that is qualified to teach subject S, can be assigned to teach S}, about a simplified school domain.
\begin{lstlisting}[style=asp-code, caption=School planning in ASP, label={lst:school-planning}]
    % The curriculum consists of different courses.
    subject(german).
    subject(english).
    subject(maths).
    subject(biology).
    subject(history).

    % Each teacher can teach one or more subjects
    teacher(bob).
    can_teach(bob, english).
    can_teach(bob, maths).
    teacher(alice).
    can_teach(alice, maths).
    can_teach(alice, history).
    teacher(claire).
    can_teach(claire, german).
    can_teach(claire, history).
    teacher(joe).
    can_teach(joe, biology).
    can_teach(joe, history).

    % Assign subjects to teachers
    { teaches(T, S) } :- teacher(T), can_teach(T, S).
    % such that..
    % Every teacher teaches at least one subject
    :- teacher(T), not teaches(T, _).
    % Every subject is taught by exactly one teacher
    :- subject(S), not teaches(_, S).
    :- teaches(T1, S), teaches(T2, S), T1 != T2.
\end{lstlisting}    
Evaluating the above program using an \emph{answer set solver}, i.e. an interpreter for the \gls{asp} language, yields the following collection of so-called \emph{answer sets}:
\begin{itemize}
    \item $A_1 = \{teaches(bob,english), teaches(bob,maths), teaches(alice,history),\\teaches(claire,german), teaches(joe,biology)\}$
    \item $A_2 = \{teaches(bob,english), teaches(alice,maths), teaches(claire,german),\\teaches(claire,history), teaches(joe,biology)\}$
    \item $A_3 = \{teaches(bob,english), teaches(alice,maths), teaches(claire,german),\\teaches(joe,biology), teaches(joe,history)\}$
    \item $A_4 = \{teaches(bob,english), teaches(alice,maths), teaches(alice,history),\\teaches(claire,german), teaches(joe,biology)\}$
\end{itemize}
Intuitively, each answer set constitues a valid solution to the problem specified in the program in the sense that, if one adds the propositions from the answer set to the original program, all rules are fulfilled, while no constraint is violated. What sets \gls{asp} apart from many other logic programming formalisms such as Prolog~\cite{prolog-50y} is its \emph{multi-model semantics}, i.e. that it can express more than one solution to a problem, even when there are different solutions that are mutually exclusive.
\end{example}    

Apart from the established use of \gls{asp} as a formalization language for decision  or optimization problems, more recent developments in the field are increasingly targeted towards working with external data from different sources in \gls{asp} programs. A prominent example from this direction of research is the stream reasoning framework LARS~\cite{lars}. Another recent development, NeurASP~\cite{neurasp} even generates facts out of perception results from a neural network, thereby enabling advanced artificial intelligence applications like reasoning about content in an image. Closely related to the concept of processing external data is the notion of actually influencing the outside world, for instance by writing data to a network buffer, from an \gls{asp} program, while still preserving declarative semantics. The DLV-extension Acthex~\cite{acthex} is an example of a system with basic action support. The goal of this thesis is to implement action support in the lazy-grounding solver Alpha~\cite{alpha}, along with a basic modularization concept, thus enabling the development of arbitrary programs fully within \gls{asp}.

\section{Actions and Modularization in ASP - Motivation}

To get a clearer idea of the value of being able to write a complete application exclusively in ASP, consider the program from Example \ref{ex:asp-binary-parsing}. It demonstartes a (very simple) parser component that may be easier to implement in a declarative language than some (typically imperative) general purpose language such as Java or Python. With most current \gls{asp} solvers, if one wanted to write such a declarative parsing component, reading of parser inputs (typically from some file or stream) and writing of parser output would have to be done in some other language. Especially in applications where input and output are relatively simple but parsing and data transformation logic are more complex, being able to implement the entire application in a single language would greatly streamline development.
Furthermore, to re-use code parts -- for example some generic parser as exhibited in Example \ref{ex:asp-binary-parsing} -- one would currently have to keep that code in a separate \gls{asp} file and manually avoid conflicts in naming of predicates without any language-level support for code encapsulation and modularization. 

\begin{example}
\label{ex:asp-binary-parsing}
Listing \ref{lst:binaryparsing} shows a simple \gls{asp} program that parses binary strings and calculates the corresponding decimal numbers. It uses external atoms implemented in Java for basic string operations: The predicate $str\_x\_xs$ takes inspiration from list syntax in Haskell and splits off the first character of a given string, e.g $str\_x\_xs["abc"]("a", "bc")$, while $stdlib\_string\_length$ and $stdlib\_string\_matches\_regex$ test the length of a string and whether it matches some regular expression, respectively. 
\begin{lstlisting}[style=asp-code, caption=Parsing binary strings in ASP, label={lst:binaryparsing}]
    encoding_scheme("1", "0", "[01]+").

    % Helper - binstring_intm_decoded is the "internal" 
    % predicate we're using to recursively add up the 
    % bit values
    binstring_intm_decoded(START_STR, START_STR, 0) :- 
        binstr(START_STR). 
    
    % Handle the case where the current bit is set
    binstring_intm_decoded(START_STR, CURR_STR, CURR_VALUE) :- 
        binstring_intm_decoded(START_STR, LAST_STR, LAST_VALUE),
        &str_x_xs[LAST_STR](HIGH_CODE, CURR_STR),
        &stdlib_string_length[LAST_STR](LAST_LEN),
        CURR_VALUE = LAST_VALUE + 2 ** (LAST_LEN - 1),
        encoding_scheme(HIGH_CODE, _, REGEX),
        &stdlib_string_matches_regex[START_STR, REGEX].
    
    % Handle the case where the current bit is not set
    binstring_intm_decoded(START_STR, CURR_STR, CURR_VALUE) :- 
        binstring_intm_decoded(START_STR, LAST_STR, LAST_VALUE),
        &str_x_xs[LAST_STR](LOW_CODE, CURR_STR),
        &stdlib_string_length[LAST_STR](LAST_LEN),
        CURR_VALUE = LAST_VALUE,
        encoding_scheme(_, LOW_CODE, REGEX),
        &stdlib_string_matches_regex[START_STR, REGEX].
    
    % These are the final values
    bin_number(BIN, DEC) :- 
        binstring_intm_decoded(BINSTR, "", DEC).    
\end{lstlisting}
\end{example}    

In order to overcome the aforementioned hurdles and be able to write applications with input from and output to external data sources as one piece in a single \gls{asp}-based langugage, we propose an extension of "classical" \gls{asp} in Section \ref{sec:problem-statement}. Section \ref{sec:state-of-the-art} gives an overview of the state of the art on action support and approaches to program modularizations in current \gls{asp} solving systems. Concluding this introduction, Section \ref{sec:thesis-roadmap} gives an outline of the rest this thesis.

\section{Problem Statement}
\label{sec:problem-statement}

The main goal of this thesis is for \gls{asp} programs to be able to express interactions with external data sources which potentially have an effect on the state of the outside world. Closely related to that goal is the demand for mechanisms enabling code reuse and structuting of applications - the larger an application gets, the more the necessity for some form of modularization grows. To that end, this thesis aims to deliver the contributions listed in the following paragraphs.

\paragraph{Triggering actions from programs} \label{goals:actions}Most program flows follow a chain of events, each a consequence of its predecessor, e.g. "If there exists a file A, read it. If reading was successful, do something with the content. If the operation succeeds, write the result to file B". It is highly desirable to be able to write this kind of program in a declarative, logic-based language that can leverage the strengths of \gls{asp} for the "business logic" part. Specifically, the proposed action semantics should deliver
\begin{itemize}
    \item declarative programs, i.e. order in which actions occur in code does not affect semantics,
    \item actions behaving in a functional fashion, i.e. an action always gives the same result for the same input. Especially, actions have to be idempotent in the sense that, for an ASP rule that is associated with some action, the result of the action never changes, no matter how often the rule fires.
    \item transparent action execution, i.e. every action that is executed during evaluation of a program must be reflected in an answer set.
\end{itemize}

\paragraph{Program Modularization} While not formally connected, triggering actions from programs and modularization (i.e. plugable and re-usable sub-programs), intuitively complement each other in the sense that modules enable us to define clear boundaries between "data acquisition" (i.e. I/O-centered) and "data processing" parts of a program. Introducing a simple, easy-to-use module system is therefore the second goal of this work.

\paragraph{Incremental Evaluation and Lazy Grounding} Experiences from existing systems for \gls{asp} application development such as ASAP~\cite{aspetris} or ACTHEX~\cite{acthex} show that, in order to achieve the evaluation performance necessary for use in real-world applications, \gls{asp} application code needs to be evaluated in an incremental fashion (rather than iteratively re-evaluating the whole program) whenever possible. The lazy-grounding architecture employed by ASP systems such as Alpha~\cite{alpha} offers an intuitive solution.

\paragraph{Application-programming Proof of Concept} Based on the extended version of the Alpha solver incorporating Action and Modularization support, the final goal for this thesis is to test the extended solver by deveploping an application making use of the newly introduced concepts.

\section{State of the Art}
\label{sec:state-of-the-art}
This work aims to blend action support with modularization in the context of lazy-grounding ASP solving - all three of these areas have seen a substantial amount of research in the past.

The question of how to formally represent actions and consequences of actions in logic programs has long been investigated, giving rise to so-called \emph{action languages}, which relate outcomes of actions to their causes and preconditions. Most interesting from an Answer Set Programming  point of view is the work of Gelfond and Lifschitz in~\cite{action-and-change}, where a translation from an action language into stable model programs is introduced. In terms of actually performing (rather than just modelling) actions from \gls{asp} programs, both Clingo~\cite{clingo4} and DLVHEX, through the ACTHEX~\cite{acthex} extension, offer their own flavours of support for triggering actions from programs. While Clingo does not directly support actions as a dedicated feature - and therefore offers no strictly enforced semantics for this - similar behavior can be achieved using external functions and the reactive solving features first introduced in oClingo~\cite{oclingo}. ACTHEX has thoroughly defined semantics for actions. In the ACTHEX model, answer set search and action execution are separate steps, where executability of actions is only determined after answer sets are calculated. While this gives users a high degree of flexibility in working with actions, it does not directly lend itself to the idea of a general purpose language where program behavior may be influenced by continuous two-way communication between a program and its environment.

With regards to Modularity, i.e. the process of "assembling" an ASP program from smaller building blocks (i.e. modules), a comprehensive semantics for so-called ~\emph{nonmonotonic modular logic programs} has been introduced in~\cite{mlp-krennw} and~\cite{mlp-2009}. While it does not impose any restrictions on language constructs used in modules and recursion within and between modules, it also comes with rather high computational complexity and no easily available implementations so far. A more "lightweight" approach to modularization are \emph{Templates}~\cite{templates}. As the name implies, this purely syntactic approach aims to define isolated sub-programs that can be generically used to avoid code duplication throughout an application and is conceptually similar to the well-known generics in object-oriented languages such as C++ and Java. Templates are rewritten into regular ASP rules using an "explosion" algorithm which basically "instantiates" the template by generating the needed body atoms (and rules deriving them) wherever templates are used. While easy to implement and flexible, a potential disadvantage of this concept is that - due to its purely syntactical nature - programmers need to be on the watch for potential bugs arising from unintended cyclic dependencies or recursive use of templates (leading to potential non-termination of the explosion algorithm) themselves.
Yet another powerful toolset for modular application development is provided by Clingo's \emph{multi-shot-solving}~\cite{clingo-multishot} features. Clingo allows for parameterized sub-programs which are then repeatedly grounded in a process that is conceptually similar to the notion of module instantiation in ~\cite{modules-compositionality} and solved during solving of the overall program. However, as this "contextual grounding" needs to be programmatically controlled by an external application through Clingo's API, the inherent flexibility and usefulness for incremental solving of this approach is counterweighed by a high level of proficiency with and knowledge of the Clingo system necessary to leverage these capabilites.
The concept of \emph{lazy grounding}, i.e. interleaving of the - traditionally sequential - grounding and solving steps present in most prevalent ASP solvers, is relatively new. It has been spearheaded by the GASP~\cite{gasp} and ASPERIX~\cite{asperix-fw-chain} solvers which avoid calculating the full grounding of an input program by performing semi-naive bottom-up evaluation along the input's topologically sorted (non-ground) dependency graph. While efficient in terms of memory use, this approach cannot stand up to the solving performance of systems like DLV or Clingo which employ their knowledge of all possible ground rules to perform conflict-driven nogood learning (CDNL) as part of their solving algorithm to great effect. Alpha~\cite{alpha}, a more recent lazy-grounding solver, aims to bridge this gap in performance by employing CDNL-style solving techniques~\cite{lazy-cdnl} incrementally on partially ground program parts as part of its central ground-and-solve loop.

Conceptually, the common ingredient linking the - on first glance not directly connected - areas of actions in ASP, program modularization, and lazy grounding is a need for detailed static program analysis prior to solving, be it to detect potentially invalid action sequences, calculate module instantiation orders, or for up-front evaluation of stratified program parts in a lazy-grounding context. In addition, both actions and modularization can greatly benefit from - or even depend on - incremental evaluation facilities of a solver for efficient operation. Since lazy-grounding by its very definition embodies an incremental evaluation approach, it seems only natural to incorporate actions and modularization into a lazy-grounding solver's input language in order to provide ASP programmers with a powerful tool for application development. Alpha, with its good solving performance compared to other lazy-grounding systems, support for a large part of the current ASP-Core2 language standard, and active development status, appears the natural choice as the technical backbone of this work.

\section{Thesis Roadmap}
\label{sec:thesis-roadmap}

The core part of this work is the formal specification of the Evolog language in Chapter \ref{chap:language}, where we formally define Evolog's action semantics and modularization concept and make some observations on the relationship between Evolog programs versus regular \gls{asp} programs. Chapter \ref{chap:implementation} gives an overview of how the formal specifications from Chapter \ref{chap:language} were implemented in the Alpha \gls{asp} solver, along with some examples of actual programs written in Evolog. In Chapter~\ref{chap:application-experiment}, we showcase an example application that was developed using Alpha with its Evolog extension. The application reads graphs encoded as XML data from user-supplied files, calculates three-colorings of given graphs, and writes the obtained colorings to XML files. Finally, Chapter \ref{chap:discussion} reflects on the experiences gained for hands-on software development from developing the aforementioned showcase application. We try to gauge the practical applicability of the implementation and highlight challenges yet to be adressed. Chapter~\ref{chap:conclusion} wraps up the thesis, reflecting on the goals achieved and potential paths for future development.