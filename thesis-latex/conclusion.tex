The idea of this work was to extend traditional answer set programming with a means to apply "side-effects", i.e. influence the outside world, from a program, while still preserving fully declarative semantics. 

To that end, we introduced the language extension \emph{Evolog}, which is a conservative extension to the ASP language under stood by the Alpha ASP solver. Evolog adds a formal representation of actions to ASP that is inspired by the Monad concept from funtional programming in that it treats the state of the world side-effects are applied on as input and output of an action function. In Evolog, this is captured in the notion of a \emph{Frame}, which describes the interactions of a program with the outside world by means of an interpretation function for action rules. In order to enable a practical implementation of the action semantics that actually enables programs with actions that can be run on a physical computer, we define necessary restrictions on the structure those programs are allowed to have. The most important such restriction is that, whenever an action rule fires, the resulting atom must be part of an answer set, thereby ensuring that for every effect applied to the outside world, there is a witnessing atom in an answer set. In our reference implementation of said action semantics, in order to get to a condition that can actually be statically verified by an interpreter, we narrow this to only allowing actions in stratified programs. While this may seem like a rather large limitation, it neatly lines up with the fact that - even though ASP has multiple-model-semantics - we only run a given program on one computer at a time, so attempting to apply contradicting actions from multpile models would not be practical.

The second part of the Evolog extension is a program modularization concept that treats modules which encapsulate sub-programs as a special kind of external atoms.

TODO: related work\\
TODO: discussion of contributions and results,\\
TODO: turning into future work, what Evolog would need to get from prototype to product.